\chapter{Introduzione}
\label{cap:introduzione}

\intro{In questo primo capitolo viene descritta brevemente l'azienda in cui si è svolto lo stage e l'idea alla base del progetto.}\\

% Introduzione al contesto applicativo.\\

% \noindent Esempio di utilizzo di un termine nel glossario \gls{api}. \\

% \noindent Esempio di citazione in linea \\
% \cite{site:agile-manifesto}. \\

% \noindent Esempio di citazione nel pie' di pagina \\
% citazione\footcite{womak:lean-thinking} \\

\section{Convenzioni tipografiche}
\label{sec:convenzioni-tipografiche}

Riguardo la stesura del testo, relativamente al documento sono state adottate le seguenti convenzioni tipografiche:
\begin{itemize}
	\item gli acronimi, le abbreviazioni e i termini ambigui o di uso non comune menzionati vengono definiti nel glossario, situato alla fine del presente documento;
	\item per i termini riportati nel glossario viene utilizzata la seguente nomenclatura: parola\glsoccur;
	\item i termini in lingua straniera o facenti parti del gergo tecnico sono evidenziati con il carattere \emph{corsivo}.
	\item per i termini salienti negli elenchi puntati viene utilizzato il carattere \textbf{grassetto}.
\end{itemize}

\section{L'azienda}
\label{sec:azienda}

RiskApp Srl è un'azienda, fondata nel 2015 e situata a Conselve (PD) che si occupa di fornire piattaforme digitali per la gestione del rischio nel settore assicurativo.\\
\indent Il core business dell'azienda è quello fornire consulenze e sviluppare software, che attraverso la raccolta e l'analisi di dati, permette di valutare i rischi aziendali. \\
\indent La principale soluzione di punta dell'azienda è la piattaforma \emph{RiskAPP}, composta da diversi moduli, ognuno dei quali è dedicato ad un servizio specifico, alcuni di questi sono:
\begin{itemize}
    \item \textbf{monitoraggio del cliente}: attraverso \emph{Cacciatore di Dati}\copyright, una tecnologia proprietaria che si occupa di raccogliere tutte le informazioni disponibili online sul cliente e con l'ausilio di un motore di \gls{iag}\glsoccur è possibile identificare la causa di quello che potrebbe essere un fatto percursore di un sinistro;
    \item \textbf{valutazione delle somme assicurative}: una serie di strumenti, permettono di aiutare ad effettuare una corretta valutazione delle somme assicurative, per evitare di incorrere in sovra o sottostime;
    \item \textbf{analisi dei rischi dovuti a calamità naturali}: possibilità di prevenire conseguenze negative attraverso la valutazione dei rischi dovuti a calamità naturali;
    \item \textbf{analisi delle esigenze}: attraverso lo strumento \emph{Insurance Advisor} è possibile creare un report di consulenza per il cliente, in cui vengono evidenziate le sue esigenze e le possibili soluzioni.
\end{itemize}

\section{L'idea}
\label{sec:idea}

L'idea del progetto di stage è nata dall'esigenza, da parte dell'azienda, di avere uno strumento digitale che permettesse di tracciare gli incontri con i possibili e/o esistenti \glspl{cliente}\glsoccur.
Questi incontri sono molto importanti, in quanto permettono di raccogliere informazioni utili per la creazione di un report di consulenza.\\
\indent Dunque lo scopo finale è quello di avere un'applicazione, integrata nel \emph{Tool trattative}, strumento incluso nella piattaforma \emph{RiskAPP}, che permetta di raccogliere le informazioni più importanti di un incontro in una \gls{meetingnote}\glsoccur, in modo da monitorare il \gls{funnel}\glsoccur di tutte le trattative di una compagnia assicurativa, l'andamento ed eventualmente mettere in atto delle azioni migliorative o correttive in caso di necessità.\\
\indent L'applicazione è rivolta ad un sotto insieme di utenti, i commerciali di una compagnia assicurativa, che svolgono il loro lavoro in mobilità, dunque deve essere usufruibile da dispositivi \emph{mobile}, in particolare gli \emph{smartphone}.\\
Si tratta quindi di sviluppare un'applicazione \gls{crossplatform}\glsoccur, in modo da poter essere utilizzata sia su dispositivi \emph{Android} che \emph{iOS}.

% \section{Organizzazione del testo}

% \begin{description}
%     \item[{\hyperref[cap:processi-metodologie]{Il secondo capitolo}}] descrive ...
    
%     \item[{\hyperref[cap:descrizione-stage]{Il terzo capitolo}}] approfondisce ...
    
%     \item[{\hyperref[cap:analisi-requisiti]{Il quarto capitolo}}] approfondisce ...
    
%     \item[{\hyperref[cap:progettazione-codifica]{Il quinto capitolo}}] approfondisce ...
    
%     \item[{\hyperref[cap:verifica-validazione]{Il sesto capitolo}}] approfondisce ...
    
%     \item[{\hyperref[cap:conclusioni]{Nel settimo capitolo}}] descrive ...
% \end{description}