\chapter{Progettazione}
\label{cap:progettazione}

\intro{In questo capitolo verrà illustrata la fase di progettazione del prodotto, partendo dalla realizzazione di un mockup, passando per la definizione dell'architettura e le tecnologie da utilizzare.}\\

\section{Mockup}
\label{sec:mockup}

Il primo passo per la progettazione dell'applicazione è stato quello di realizzare un \gls{mockup}\glsoccur con lo scopo di definirne il più dettagliatamente possibile l'interfaccia grafica: il numero di viste necessarie, la loro struttura, i componenti grafici e la loro disposizione, la \emph{palette} dei colori, come l'utente interagirà con l'applicazione e come questa dovrà rispondere a tali interazioni, simulabili attraverso un prototipo. Infatti, per l'implementazione della \gls{uig}\glsoccur, si è rivelato essere di notevole utilità, in quanto ha reso più semplice e veloce la realizzazione di quest'ultima, avendo compreso a priori come questa dovesse essere strutturata.\\
Inoltre il \gls{mockup} è stato utilizzato per definire le funzionalità che l'applicazione deve offrire, in modo da avere un'idea più chiara di come queste debbano essere implementate e, infine, come riportato nella Sezione \ref{subsec:variazione-pianificazione}, è stato indispensabile per la fase di \emph{analisi dei requisiti}.\\ 
Si specifica che nel corso della fase di implementazione sono state apportate delle modifiche all'interfaccia grafica per migliorarne l'usabilità, mantenendo però invariata la struttura generale dell'applicazione e prestando particolare attenzione all'esperienza utente, in modo da rendere l'utilizzo dell'applicazione, in mobilità, il più semplice e intuitivo possibile.

\section{Architettura}
\label{sec:architettura}
% State Management -> RIVERPOD + Diagramma illustrativo
% https://codewithandrea.com/articles/flutter-app-architecture-riverpod-introduction/

\subsection{Architettura Flutter}
\label{subsec:architettura-flutter}

Per comprendere al meglio la peculiarità di \emph{Flutter}, descritto nella Sezione \ref{subsec:flutter}, è necessario analizzare l'architettura \cite{site:flutter-architecture} delle applicazioni realizzate con esso, che sono composte dagli elementi illustrati nella Figura \ref{fig:architettura-flutter}, tra i quali, i principali sono descritti di seguito:
\begin{itemize}
    \item \textbf{Embedder}: fornisce un punto d'ingresso con il sistema operativo ospitante per accedere ai servizi forniti da esso. Questo consente dunque l'esecuzione dell'applicazione su diverse piattaforme e la possibilità di utilizzare librerie per accedere a funzionalità esclusive di determinati sistemi operativi, oppure viceversa, integrare il codice \emph{Flutter} in un'applicazione nativa già esistente;
    \item \textbf{Flutter engine}: è il componente centrale che fornisce l'implementazione di basso livello dell'\gls{apig}\glsoccur principale di \emph{Flutter}, inclusa la grafica, il layout di testo, operazioni di \emph{I/O}, rete, ecc.;
    \item \textbf{Flutter framework}: componente con il quale lo sviluppatore interagisce attraverso un insieme di librerie, che partendo dal basso sono:
    \begin{itemize}
        \item \textbf{foundation}: fornisce un'astrazione delle funzionalità di animazione, grafica e \gls{gesture}\glsoccur;
        \item \textbf{rendering}: fornisce un'astrazione per gestire il layout. Con questo livello è possibile costruire un albero di oggetti renderizzabili.
        \item \textbf{widget}: ogni oggetto renderizzabile ha una classe corrispondente in questa libreria, definiti appunto \emph{widget}, l'unità fondamentale in \emph{Flutter}, che permettono la realizzazione dell'interfaccia grafica;
        \item \textbf{Material e Cupertino}: forniscono un'implementazione di alto livello dei \emph{widget} per la realizzazione di interfacce grafiche seguendo le linee guida di \emph{Material Design} e \emph{Cupertino}, rispettivamente per le piattaforme \emph{Android} e \emph{iOS}.
    \end{itemize}
    \item \textbf{Dart App}: codice sorgente sviluppato dall'utente, contenente i \emph{widget} per l'implementazione della \gls{uig}\glsoccur e la logica dell'applicazione.
\end{itemize}

Inoltre, essendo un \gls{framework}\glsoccur \gls{open-source}\glsoccur, è possibile utilizzare librerie di terze parti, sviluppate dalla \emph{community}, per aggiungere funzionalità all'applicazione, come ad esempio librerie per la gestione dello stato dell'applicazione, librerie per la gestione delle richieste HTTP, ecc. \\

\begin{figure}[!h] 
    \centering 
    \includegraphics[width=0.4\columnwidth]{images/flutter-app-anatomy.png} 
    \caption{Struttura di un'applicazione \cite{site:flutter-architecture} in \emph{Flutter}.}
    \label{fig:architettura-flutter}
\end{figure}

Lo stato posseduto dai \lstinline{StatefulWidget} (vedi Sezione \ref{subsec:flutter}) può essere considerato come \emph{ephemeral} (ing. effimero) o \emph{app state} (ing. stato dell'applicazione). \\
\emph{Ephemeral} è uno stato che può essere opportunamente confinato all'interno di un singolo \emph{widget} e gestito attraverso la primitiva \lstinline{setState()}, che permette di definire come aggiornarlo.\\
Mentre \emph{app state} è uno stato che viene condiviso tra più \emph{widget} e per la sua gestione, più complessa utilizzando solamente la primitiva sopra citata, esistono diverse librerie di terze parti, ciascuna con le proprie peculiarità a seconda del caso d'uso.\\
La problematica di come gestire lo stato dell'applicazione è definito in gergo \emph{state management} \cite{site:flutter-state-mgmt}, e dopo un'attenta analisi delle librerie disponibili, si è scelto di utilizzare \emph{Riverpod} \cite{site:riverpod}.
\subsection{Riverpod}
\label{subsec:riverpod}

\emph{Riverpod} semplifica notevolmente lo \emph{state management} e si basa su un concetto evoluto da \emph{Provider} \cite{site:provider}, libreria da cui deriva.\\
A differenza di \lstinline{InheritedWidget} \cite{site:inheritw} che permette di condividere lo stato tra più \emph{widget}, \emph{Riverpod}, che ne è una reimplementazione, fornisce \emph{providers} che sono indipendenti dai \emph{widget}, in quanto una volta dichiarato il \lstinline{ProviderScope} a livello globale, possono essere letti ovunque, a condizione che la classe richiamante venga estesa con \lstinline{ConsumerWidget} o \lstinline{ConsumerStatefulWidget} \cite{site:reading-provider}. \\
Da questo ne consegue che si evita di aggiornare l'interfaccia grafica, ovvero di richiamare il metodo \lstinline{build()} di un \emph{widget} quando non è necessario, poichè è un'operazione costosa.
Ne esistono di varie tipologie, per le quali si rimanda alla documentazione ufficiale \cite{site:riverpod}, ma quelle utilizzate in questo progetto sono:
\begin{itemize}
    \item \lstinline{FutureProvider} \cite{site:future-provider}: utilizzato per ricevere un valore generato da un'operazione asincrona (es: richiesta \gls{httpg}\glsoccur);
    \item \lstinline{StateNotifierProvider} \cite{site:state-notifier-provider}: utilizzato per gestire una classe che mantiene uno stato, più complesso di una semplice variabile primitiva, e di esporre dei metodi per monitorarlo o aggiornarlo.
\end{itemize}

\subsection{Architettura dell'applicazione}
\label{subsec:architettura-app}

Per l'architettura dell'applicazione si è scelto di utilizzare un \emph{pattern} architetturale basato su \gls{mvcg}\glsoccur \cite{site:app-architecture}, che permette di separare la logica dell'applicazione dalla sua rappresentazione grafica, in modo da rendere più semplice la manutenzione e l'aggiunta di nuove funzionalità.\\
Nel dettaglio l'applicazione è composta da quattro livelli:
\begin{itemize}
    \item \textbf{Data Layer}: contiene le classi che si occupano di recuperare i dati dal server, attraverso richieste \gls{httpg}\glsoccur, e di convertirli in oggetti rappresentati nel \emph{domain layer};
    \item \textbf{Domain Layer}: contiene le classi che rappresentano i dati dell'applicazione;
    \item \textbf{Application Layer}: contiene le classi che si occupano di gestire la logica dell'applicazione;
    \item \textbf{Presentation Layer}: contiene le classi che si occupano di gestire l'interfaccia, ovvero di eseguire il rendering dei \emph{widget} e di gestire gli eventi generati dall'utente.
\end{itemize}
Questa architettura permette inoltre di avere la possibilità di definire eventualmente più sorgenti da cui recuperare i dati senza dover modificare il codice relativo ai livelli superiori, in quanto è sufficiente modificare il \emph{data layer}.

\section{Struttura del progetto}
\label{sec:struttura-progetto}
% Struttura del progetto: cartelle, file, ecc. -> LAYER FIRST
Un altro aspetto importante da considerare per la realizzazione di un progetto software è la sua struttura, ovvero come organizzare i file e le cartelle che lo compongono.\\
Dopo opportune ricerche ed analisi, si è scelto di adottare, tra le due alternative disponibili, la struttura \emph{layer first}, che prevede di organizzare i file e le cartelle in base al livello a cui appartengono, in modo da rendere più semplice la manutenzione e l'aggiunta di nuove funzionalità.\\
\emph{Feature first}, l'altra alternativa, organizza invece i file e le cartelle, mantendendo la separazione tra i livelli, in base alle funzionalità che l'applicazione offre.\\
Il motivo principale per cui la scelta non è ricaduta su quest'ultima è che, nonostante sia quella che garantisca un'organizzazione e manutenzione del codice migliore, risulta essere più complessa da implementare in quanto adatta per progetti di dimensione e complessità maggiore.\\
Nella Figura \ref{fig:project-structures} verranno illustrati entrambi gli approcci, in modo da poterli confrontare e comprendere meglio le loro differenze \cite{site:project-structure}.

\begin{figure}[!h] 
    \centering 
    \includegraphics[width=0.8\columnwidth]{images/project_structures.png} 
    \caption{Esempi di approccio \emph{Layer First} (a sinistra) e \emph{Feature First} (a destra).}
    \label{fig:project-structures}
\end{figure}

% \section{Diagrammi UML}
% \label{sec:uml}
% dove inserirli?
% UML -> Diagrammi delle classi

% \subsubsection{Namespace 1} %**************************
% Descrizione namespace 1.

% \begin{namespacedesc}
%     \classdesc{Classe 1}{Descrizione classe 1}
%     \classdesc{Classe 2}{Descrizione classe 2}
% \end{namespacedesc}

\section{Ambiente di sviluppo}
\label{sec:ambiente-sviluppo}
% SEGUI ISSUE #4 E #5 PER LA STESURA DI QUESTA SEZIONE
Di seguito viene descritto l'ambiente di sviluppo e data una panoramica delle tecnologie e strumenti utilizzati.

\subsection*{Figma}
\label{subsec:figma}

\emph{Figma} \cite{site:figma} è un software di editor di grafica vettoriale che permette di progettare interfacce grafiche per applicazioni \emph{web} e \emph{mobile}.\\
È stato utilizzato per la realizzazione del \gls{mockup}\glsoccur dell'applicazione, in quanto vi è la possibilità di creare un prototipo interattivo, che simula l'interazione dell'utente con l'applicazione, e di condividerlo con il team di sviluppo, in modo da avere un'idea più chiara di come l'applicazione debba essere strutturata e di come debba funzionare.

\subsection*{Git}
\label{subsec:git}

\emph{Git} \cite{site:git} è un sistema di controllo di versione, finalizzato al tracciamento del codice sorgente e delle sue modifiche, inoltre ne permette la condivisione e dunque la collaborazione tra più sviluppatori.\\
Inoltre è possibile, in caso di errori, poter ripristinare una versione precedente del codice sorgente.

\subsection*{GitHub}
\label{subsec:github}

% DA DEFINIRE: ISSUE, MILESTONE
\emph{GitHub} \cite{site:github} è un servizio di hosting per il codice sorgente di progetti software che utilizza \emph{Git}.\\
Per questo progetto, è stato utilizzato per la condivisione e gestione del codice sorgente attraverso una repository dedicata, fornita dall'azienda.\\
Per la pianificazione della fase di implementazione, è stato utilizzato il sistema di \gls{issuetracking}\glsoccur integrato, creando delle \gls{milestone}\glsoccur, per ogni classe di obbiettivi da raggiungere in base alla loro priorità (vedi Sezione \ref{sec:obiettivi}).\\
In ciascuna \gls{milestone}\glsoccur, sono state create delle \emph{issue}, in base ai requisiti o ad un insieme di questi, necessarie per il raggiungimento di ciascun obiettivo, garantendo così una maggiore organizzazione e tracciabilità del lavoro svolto e dei progressi fatti.

\subsection*{VSCode}
\label{subsec:vscode}

\emph{VSCode} è un editor di codice sorgente \gls{open-source}\glsoccur che oltre a fornire le funzionalità di base necessarie per lo sviluppo (ad esempio: controllo di sintassi, \emph{debugging}, analisi statica del codice, ecc.), supporta molti linguaggi di programmazione ed è possibile estendere le sue funzionalità o il numero di linguaggi supportati attraverso delle estensioni.\\
Di fatto per questo progetto si è reso necessario l'installazione di alcune estensioni, tra cui:
\begin{itemize}
    \item \textbf{Flutter} \cite{site:flutter-extension};
    \item \textbf{Dart} \cite{site:dart-extension}.
\end{itemize}

\subsection*{Flutter}
\label{subsec:flutter}

\emph{Flutter} \cite{site:flutter} è un \gls{framework}\glsoccur che consente di sviluppare applicazioni native per diverse piattaforme, come Android, iOS, web e desktop, utilizzando un unico linguaggio di programmazione, riducendo i tempi e i costi di produzione, senza compromettere le prestazioni dell'applicazione.\\
Il concetto centrale di \emph{Flutter} è quello dei \emph{widget}, oggetti che descrivono come deve essere visualizzata una parte dell'interfaccia grafica. Questi possono essere di due tipi:
\begin{itemize}
    \item \textbf{StatelessWidget}: non hanno uno stato interno, ovvero non cambiano nel tempo, e sono definiti da un insieme di proprietà, chiamate \emph{proprietà immutabili}, che vengono passate al costruttore del \emph{widget};
    \item \textbf{StatefulWidget}: al contrario, possiedono uno stato interno mutabile, e viene usato quando una parte dell'interfaccia utente può cambiare dinamicamente.
\end{itemize}

\subsection*{StarUML}
\label{subsec:staruml}

\emph{StarUML} \cite{site:staruml} è uno strumento di modellazione per sistemi software che sono sviluppati secondo il paradigma \emph{orientato agli oggetti}, attraverso la creazione di diagrammi \gls{umlg}\glsoccur.\\
È stato utilizzato per la realizzazione dei casi d'uso (vedi Sezione \ref{sec:usecase}) e dei diagrammi delle classi.

\subsection*{Emulatori Android e iOS}
\label{subsec:emulatori}

Per effettuare i test dell'applicazione su dispositivi \emph{Android} e \emph{iOS}, è stato utilizzato rispettivamente \emph{Android Studio} \cite{site:android-studio} e \emph{Xcode} \cite{site:xcode}, che forniscono degli emulatori per le rispettive piattaforme.\\

\subsection*{API della piattaforma RiskAPP}
A completare l'ambiente di sviluppo, l'azienda mi ha fornito l'accesso alle \gls{apig}\glsoccur del \gls{backend}\glsoccur della piattaforma \emph{RiskAPP} attraverso lo \emph{Swagger} \cite{site:swagger}, strumento che, tra le altre cose, consente di consultare la documentazione delle \gls{apig}\glsoccur e di testarle attraverso un'interfaccia grafica \emph{web}.\\
Si specifica che, per lo sviluppo dell'applicazione, l'utilizzo di tali \gls{apig}\glsoccur è stato svolto su un \emph{server} di collaudo.