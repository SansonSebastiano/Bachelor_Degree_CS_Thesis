\chapter{Conclusioni}
\label{cap:conclusioni}

\intro{In questo ultimo capitolo verranno illustrati gli obiettivi raggiunti, il consultivo delle attività e infine verrà fornita una valutazione personale sul tirocinio svolto.}\\

\section{Obiettivi raggiunti}
\label{sec:raggiungimento-obiettivi}

La Tabella \ref{tab:obbiettivi-raggiunti} illustra gli obiettivi raggiunti durante al termine del tirocinio.

\begin{table}
    \centering
    \begin{tabular}{|l|l|l|}
        \hline
        \textbf{Obiettivo} & \textbf{Priorità} & \textbf{Stato} \\ \hline
        \hyperref[O01]{O01}                & Obbligatorio      & Soddisfatto    \\ \hline
        \hyperref[O02]{O02}                & Obbligatorio      & Soddisfatto    \\ \hline
        \hyperref[O03]{O03}                & Obbligatorio      & Soddisfatto    \\ \hline
        \hyperref[O04]{O04}                & Obbligatorio      & Soddisfatto    \\ \hline
        \hyperref[O05]{O05}                & Obbligatorio      & Soddisfatto    \\ \hline
        \hyperref[D01]{D01}                & Obbligatorio      & Soddisfatto    \\ \hline
        \hyperref[D02]{D02}                & Obbligatorio      & Non soddisfatto    \\ \hline
        \hyperref[F01]{F01}                & Obbligatorio      & Non soddisfatto    \\ \hline
    \end{tabular}%
\caption{Tabella con gli obbiettivi raggiunti}
\label{tab:obbiettivi-raggiunti}
\end{table}

\noindent Il mancato soddisfascimento degli obbiettivi \emph{D02} e \emph{F01}, è stato dovuto, in generale, alla mancanza di tempo, in quanto, come si può notare dalla variazione di pianificazione (sezione \ref{subsec:variazione-pianificazione}), la fase di sviluppo è stata più lunga del previsto.\\
Malgrado non siano stati progettati e scritti i test per la verifica e validazione dell'applicazione, si è comunque cercato di mantenere un codice pulito e ben strutturato, in modo da facilitare eventuali modifiche e manutenzioni future, anche se come garanzia di qualità non può considerarsi sufficiente.
Si specifica, inoltre, che la verifica delle funzionalità dell'applicazione è stata effettuata dal tutor aziendale o da altri colleghi, lungo tutto il periodo di sviluppo del progetto, fornendo un \emph{feedback} immediato, permettendo così di apportare eventuali modifiche correttive e miglioramenti.\\
Per quanto riguarda il \gls{deploy}\glsoccur, il principale motivo per cui non è stato effettuato, è da imputarsi al fatto che l'utenza a cui è rivolta l'applicazione è molto ristretta, risultava quindi superfluo e non necessario pubblicarla sulle piattaforme \emph{Google Play Store} e \emph{Apple App Store}. Tale decisione è stata presa dal tutor aziendale che ha ritenuto più opportuno pensare ad un modo alternativo per distribuire l'applicazione direttamente agli utenti finali.\\
La Tabella \ref{tab:ore-pianificate-effettive} evidenzia la differenza tra le ore pianificate (presenti nel documento \emph{Piano di lavoro}) ed effettive per ogni attività del tirocinio.

\begin{table}
    \centering
    \begin{tabular}{|l|l|l|}
        \hline
        \textbf{Attività} & \textbf{Ore pianificate} & \textbf{Ore effettive} \\ \hline
        Formazione sulle tecnologie adottate                & 80      & 60    \\ \hline
        Definizione architettura di riferimento e relativa documentazione                & 40      & 40    \\ \hline
        Sviluppo                & 160      & 200    \\ \hline
        Verifica e validazione                & 40      & 20    \\ \hline
    \end{tabular}%
\caption{Consutivo della durata delle attività}
\label{tab:ore-pianificate-effettive}
\end{table}

\section{Valutazione personale}
\label{sec:valutazione-personale}

Il tirocinio è stato un'esperienza molto formativa, in quanto mi ha permesso di mettere in pratica le conoscenze e metodologie acquisite durante il percorso di studi in un contesto lavorativo.\\
Come già menzionato in precedenza, nonostante la mia familiarità con alcune delle tecnologie adottate, ho avuto modo di approfondirle e di imparare nuovi concetti, soprattutto per quanto riguarda il \emph{framework} \emph{Flutter}.\\
Inoltre, per la prima volta ho avuto modo di lavorare ad un progetto in maniera più professionale, seguendo anche alcuni principi affrontati durante il corso di \emph{Ingegneria del Software}, come ad esempio la pianificazione della fase di sviluppo, fissando delle \gls{milestone}\glsoccur, mantenendo traccia dei progressi per il raggiungimento degli obiettivi del progetto, e la documentazione del codice, in modo da rendere più facile la comprensione e la manutenzione del codice.\\  
L'azienda mi ha lasciato molta autonomia durante lo sviluppo del progetto e mi ha fornito un \emph{feedback} costante, permettendomi di migliorare e di imparare nuove cose. \\
In conclusione, ritengo che il tirocinio sia stato un'esperienza molto positiva, che mi ha permesso di crescere sia dal punto di vista professionale che personale.