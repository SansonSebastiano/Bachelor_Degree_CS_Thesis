\chapter{Implementazione}
\label{cap:implementazione}

\intro{In questo capitolo si discuterà dell'implementazione (o codifica) dell'applicazione in conseguenza alle scelte progettuali descritte nel capitolo precedente. Inoltre verranno descritte le librerie di terze parti utilizzate, motivandone la scelta.}\\

Seguendo quanto descritto nel capitolo \ref{cap:progettazione}, si è proceduto con l'implementazione del prodotto software.\\
Di seguito verrà illustratta l'effettiva struttura del progetto basata sull'approccio \emph{layer first} (vedi sezione \ref{sec:struttura-progetto}) implementata, descrivendone poi la varie classi contenute in ciascuna cartella. \\

\begin{verbatim}
    assets/
    lib/
        components/
        constants/
        data/
            model/
            service/
        provider/
        screens
        styles/
        utils/
        main.dart
\end{verbatim}

L'architettura menzionata nella sezione \ref{subsec:architettura-app} è stata implementata nel seguente modo: il \emph{data layer} e \emph{domain layer} sono stati implementati rispettivamente in \lstinline{service} e \lstinline{model}, contenute all'interno della cartella \lstinline{data}, mentre l'\emph{application layer} è stato implementato nella cartella \lstinline{provider} e il \emph{presentation layer} nella cartella \lstinline{screens}.\\
Le restanti cartelle sono servite come ausilio per l'implementazione delle funzionalità dei \emph{layer} sopraccitati, come ad esempio \lstinline{components} per la creazione di \emph{widget} personalizzati, \lstinline{constants} per la definizione di costanti, \lstinline{styles} per la definizione di stili e del tema dell'applicazione e \lstinline{utils} per la definizione di alcune funzioni di utilità.\\

\section{Components}
\label{sec:components}

INSERIRE LE FIGURE???\\
SPIEGARE O RIMANDARE ALLA DOCUMENTAZIONE DI COME FUNZIONANO I WIDGET\\

In questa cartella sono raccolti tutti i \emph{widget} personalizzati, utilizzati dalle schermate presenti in \emph{screens} (vedi sezione \ref{sec:screens}).\\
Di seguito verranno descritti i vari \emph{widget} implementati, suddivisi in base alla loro funzionalità.\\
Si specifica inoltre che la maggior parte di questi abbiano una visibilità pubblica, in quanto devono essere richiamabile dalle schermate, mentre alcuni hanno una visibilità privata, in quanto sono stati creati per essere utilizzati esclusivamente all'interno di altri \emph{widget}.\\

\subsection{Alert Dialogs}
\label{subsec:alert-dialogs}

Nel file denominato \lstinline{alert_dialogs.dart} sono stati implementati dei \emph{widget}, estendendo \lstinline{StatelessWidget}, che consentono di personalizzare un \emph{alert dialog}.\\

\subsubsection*{CustomBaseAlertDialog}
\label{subsubsec:custom-base-alert-dialog}

È una classe che implementa un \lstinline{AlertDialog}\cite{site:alert-dialog} e ne definisce l'aspetto base, fissando alcuni elementi, come \lstinline{Text}\cite{site:text} per il titolo, \lstinline{Widget} per l'icona posta sotto il titolo, il suo colore e \gls{padding}\glsoccur.\\
Mentre è possibile scegliere se aggiungere o meno un testo descrittivo e dei bottoni di conferma e/o annulla, in base all'operazione del contesto.\\

\subsubsection*{IconAlertDialog}
\label{subsubsec:icon-alert-dialog}

Classe che implementa \lstinline{CustomBaseAlertDialog} personalizzandolo ulteriormente impostando con dei valori fissi sia il \gls{padding}\glsoccur che la dimensione dell'icona.\\
Attraverso il costruttore è obbligatorio passare il \emph{widget} di tipo \lstinline{IconData}\cite{site:icon-data} che rappresenta l'icona, il suo colore e un titolo, mentre è opzionale passare un testo descrittivo, e se aggiungere o meno dei bottoni di conferma e/o annulla.\\

\subsubsection*{LoadingAlertDialog}
\label{subsubsec:loading-alert-dialog}

Personalizzazione di \lstinline{IconAlertDialog} in cui viene impostata come icona un \lstinline{CircularProgressIndicator}\cite{site:circular-progress-indicator}, che rappresenta un indicatore di caricamento.\\
È possibile decidere, attraveso il costruttore, il colore dell'indicatore e il titolo da visualizzare.\\
Lo scopo di questa classe è quella di essere utilizzata per mostrare un indicatore di caricamento durante l'esecuzione di un'operazione asincrona.\\

\subsubsection*{WarningAlertDialog}
\label{subsubsec:warning-alert-dialog}

Personalizzazione di \lstinline{IconAlertDialog} dove si richiede, nel costruttore, di passare un'icona e il suo colore, un titolo, il contenuto del testo descrittivo, l'azione che deve compiere il bottone di conferma alla sua pressione e il suo stile grafico (vedi sezione \ref{sec:styles}), mentre il pulsante di annullamento è stato fissato.\\
Lo scopo di questa classe è quella di essere utilizzata per mostrare un messaggio di avvertimento all'utente riguardante una scelta e la sua conferma.\\
\subsubsection*{ResponseDialog}
\label{subsubsec:response-dialog}

Personalizzazione di \lstinline{IconAlertDialog} dove si richiede, nel costruttore, di passare un'icona, il suo colore e un titolo.\\
Lo scopo di questa classe è quella di essere utilizzata per mostrare un messaggio di risposta all'utente riguardante l'esito di un'operazione.\\

\subsubsection*{NetworkAlertDialog}
\label{subsubsec:network-alert-dialog}

La sua implementazione è simile a \lstinline{WarningAlertDialog}, con la differenza che non è presente il pulsante di annullamento, in quanto il suo scopo è quello di notificare l'utente la mancanza di connessione ad internet e di cliccare il pulsante di conferma per chiudere l'\emph{alert dialog}.\\

\subsection{App Bars}
\label{subsec:app-bars}

Nel file denominato \lstinline{app_bars.dart} sono stati implementati dei widget, estendendo \lstinline{StatelessWidget}, che consentono di personalizzare un \emph{app bar}.\\

\subsubsection*{CustomAppBar}
\label{subsubsec:custom-app-bar}

Classe che implementa \lstinline{AppBar}\cite{site:app-bar} definendone il colore di background, applicato dal tema dell'applicazione (vedi sezione \ref{sec:styles}) e il titolo, in cui si tratta dell'applicazione di un'immagine vettoriale contenuta nella cartella \lstinline{assets} la quale rappresenta il logo dell'azienda.\\
Infine è possibile scegliere se aggiungere o meno un'icona, che rappresenta il pulsante di accesso alla schermata dell'account utente.\\
La motivazione di quest'ultima scelta è a causa del fatto che questa personalizzazione dell'\lstinline{AppBar} viene utilizzata per la quasi totalità delle viste, anche in quella dell'account utente, dove non si rende necessaria l'icona menzionata precedentemente poichè ci si trova già in tale vista.

\subsubsection*{LoginAppBar}
\label{subsubsec:login-app-bar}

Classe che implementa \lstinline{AppBar}, personalizzandolo appositamente per la schermata di \emph{login} (vedi sezione \ref{sec:screens}), che è simile a quella precedente, differenziandosi per la non presenza dell'icona che funge da pulsante di accesso alla schermata dell'account utente.

\subsection{Biomteric Switch}
\label{subsec:biometric-switch}

Nel file denominato \lstinline{biometric_switch.dart} è stato implementato un \emph{widget} che estende \lstinline{StatefulWidget} per costruire un componente in grado di permettere all'utente di abilitare il riconoscimento biometrico. \\
Questo \emph{widget} è composto da un \lstinline{Switch}\cite{site:switch} e un \lstinline{Text}\cite{site:text} che rappresenta il testo descrittivo.\\
Inoltre ne viene definito l'aspetto e il comportamento attraverso la definizione di un \lstinline{State} che estende \lstinline{StatefulWidget} implementando il metodo \lstinline{build} per la costruzione del \emph{widget} e il metodo \lstinline{onChanged} per la gestione dell'evento di cambiamento di stato del \lstinline{Switch}.\\
Stato che è rappresentato da due variabili di tipo \lstinline{bool}, una che si occupa di gestire l'abilitazione del componente e l'altro che indica se nel dispositivo in cui viene eseguita l'applicazione è supportato il riconoscimento biometrico.\\
La prima tra queste, ad ogni suo cambiamento di stato, viene salvata in locale attraverso una libreria di terze parti (di cui verrà discussa nella sezione \ref{sec:utils} - SPECIFICA LA PARTE PRECISA) per mantenere in memoria la preferenza dell'utente. \\
Il caso in cui la seconda variabile menzionata precedentemente sia \lstinline{false}, dunque non vi è il supporto per il riconoscimento biometrico, attraverso un \emph{widget} \lstinline{Text} si notifica l'utente di tale mancanza e si disabilita il \lstinline{Switch}.

\subsection{Date Picker}
\label{subsec:date-picker}

Nel file denominato \lstinline{date_picker.dart} sono stati implementati dei \emph{widget} che estendono sia \lstinline{StatelessWidget} che \lstinline{StatefulWidget} per costruire dei componenti in grado di permettere all'utente di selezionare una data o un intervallo di date.\\

\subsubsection*{DateButton}
\label{subsubsec:date-button}

Questo componente estende \lstinline{StatelessWidget} e si occupa di costruire un bottone che visualizza la data, o un intervallo di date, selezionata/e dall'utente.\\
Ha visibilità privata, poichè è stato creato per essere utilizzato esclusivamente all'interno dei \emph{widget} contenuti nello stesso file.\\
È composto da un \lstinline{TextButton}\cite{site:text-button}, in cui viene definito l'aspetto grafico, e un \lstinline{Text} che visualizza la data o un intervallo di date. \\
Attraverso il suo costruttore è possibile definire il testo da visualizzare e il comportamento del bottone alla sua pressione.\\

\subsubsection*{CustomDateRangePicker}
\label{subsubsec:custom-date-range-picker}

Questo componente estende \lstinline{StatefulWidget} e si occupa di costruire un \emph{widget} che permette all'utente di selezionare un intervallo di date.\\
È composto da \lstinline{DateButton}, al quale il primo parametro che viene passato è l'intervallo di date da visualizzare, di default o selezionato dall'utente, precedentemente formattato da un metodo privato \lstinline{dateFormatter}: riceve in input la data iniziale e finale dell'intervallo (nel caso in cui si intenda selezionare un singolo giorno è sufficiente impostarne con esso entrambi i cambi) e restituisce una stringa che rappresenta l'intervallo di date formattato opportunamente.\\
Mentre come secondo parametro viene passato un altro metodo privato \lstinline{show} che si occupa di visualizzare il calendario e di consentire all'utente di seleziona un intervallo. \\
Inoltre è presente un metodo pubblico \lstinline{onDateRangeSelected} che si occupa di gestire l'evento di selezione dell'intervallo di date, aggiornando lo stato del \emph{widget}. \\
Lo scopo di questo \emph{widget} è quello di fornire la possibilità all'utente di selezionare una singola data o intervallo di date per filtrare la lista di \emph{Meeting Note} (vedi requisiti \hyperref[RFN-18]{RFN-18} e \hyperref[RFN-19]{RFN-19}).

\subsubsection*{CustomDatePicker}
\label{subsubsec:custom-date-picker}

L'implementazione di questo componente è del tutto analoga a quella di \lstinline{CustomDateRangePicker}, con la differenza che permette all'utente di selezionare una singola data.\\
Mentre il suo scopo è quello di fornire all'utente, nel momento di revisione dei dati estratti dall'elaborazione del testo da parte di un algoritmo di \emph{intelligenza artificiale}, di modificare la data di una \emph{Meeting Note} da creare (vedi requisito \hyperref[RFN-53]{RFN-53}).

\subsection{Filter Panel}
\label{subsec:filter-panel}

Nel file denominato \lstinline{filter_panel.dart} è stato implementato un \emph{widget} che estende \lstinline{ConsumerStatefulWidget}\cite{site:reading-provider}, il cui comportamento è il medesimo di uno \lstinline{StatefulWidget}, con l'aggiunta che è in grado di leggere i dati forniti da un \emph{Provider} (vedi sezione \ref{subsec:riverpod}).
Tale componente è composto da quattro \emph{widget}, ciascuno dei quali consente di filtrare la lista di \emph{Meeting Note}  secondo i criteri definiti in fase di analisi dei requisiti (vedi requisiti dal \hyperref[RFN-15]{RFN-15} al \hyperref[RFN-19]{RFN-19} e dal \hyperref[RFN-31]{RFN-31} al \hyperref[RFN-33]{RFN-33}). \\
Di seguito verranno elencati i vari \emph{widget} che compongono il \emph{Filter Panel}: 
\begin{itemize}
    \item \lstinline{CustomObjectPicker} (componente la cui implementazione è discussa nella sezione \ref{subsec:object-picker});
    \item \lstinline{CustomAutocomplete} (componente la cui implementazione è discussa nella sezione \ref{subsubsec:custom-autocomplete});
    \item \lstinline{CustomDateRangePicker} (componente la cui implementazione è discussa nella sezione \ref{subsubsec:custom-date-range-picker});
    \item \lstinline{CustomToogleButtons} (componente la cui implementazione è discussa nella sezione \ref{subsec:toogle-buttons}).
\end{itemize}
Come da prassi, ne viene definito l'aspetto grafico e il comportamento che deve avere.\\
Per quanto riguarda il comportamento e dunque la gestione dello stato, che sostanzialmente consiste nel memorizzare e passare alla schermata dedicata (RIMANDARE ALLA SEZIONE PRECISA) quali filtri sono stati attivati dall'utente, tale operazione viene effettuata attraverso uno \lstinline{StateNotifierProvider} (vedi sezione \ref{subsec:riverpod}) e una classe \lstinline{FilterinOptions} (RIMANDARE ALLA SEZIONE PRECISA) che si occupa appunto di memorizzare e passare i filtri.\\

\subsection{Login Form}
\label{subsec:login-form}

Nel file denominato \lstinline{login_form.dart} è stato implementato un \emph{widget} che estende \lstinline{ConsumerStatefulWidget}.\\
Lo scopo di questo componente è quello di consentire all'utente di autenticarsi all'applicazione, per farlo ci sono due modi: inserendo manualmente le credenziali oppure utilizzando il riconoscimento biometrico.\\
Nel primo caso il componente è composto da due \lstinline{TextFormField}\cite{site:text-form-field} e un \lstinline{ElevatedButton}\cite{site:elevated-button} di cui vengono definiti l'aspetto grafico.\\
Il \lstinline{TextFormField} riguardante l'inserimento della password presenta anche un pulsante per mostrarla in chiaro o nasconderla.\\
Il pulsante di conferma è disabilitato fintanto che non vengono inserite entrambe le credenziali.\\
Successivamente attraverso l'ausilio del \emph{provider} \lstinline{authProvider} (SPECIFICA LA SEZIONE PRECISA) viene effettuata la richiesta di autenticazione, che in caso di successo porta alla schermata principale dell'applicazione, salvando il \emph{token} di autenticazione ricevuto (RIMANDA ALLA SEZIONE SPECIFICA), altrimenti viene mostrato un messaggio di errore.\\ 
Mentre nel secondo caso, si effettua un controllo per verificare se l'utente in precedenza aveva abilitato tale funzionalità, in caso affermativo viene eseguito il riconoscimento biometrico (RIMANDA ALLA SEZIONE SPECIFICA), altrimenti si prosegue con l'inserimento manuale delle credenziali.\\
L'esito dell'autenticazione attraverso il riconoscimento biometrico è il medesimo di quello descritto per l'autenticazione manuale.\\
SPECIFICARE I REQUISITI CORRELATI.

\subsection{Meeting Note Card}
\label{subsec:meeting-note-card}

\subsubsection*{MeetingNoteItem}
\label{subsubsec:meeting-note-item}

\subsubsection*{MeetingNotePreview}
\label{subsubsec:meeting-note-preview}

\subsubsection*{MeetingNoteCard}
\label{subsubsec:meeting-note-card}

\subsubsection*{MeetingNoteTitle}
\label{subsubsec:meeting-note-title}

\subsection{Object Picker}
\label{subsec:object-picker}

\subsection{Recording Button}
\label{subsec:recording-button}

\subsection{Screens Template}
\label{subsec:screens-template}

\subsubsection*{BaseScreen}
\label{subsubsec:base-screen}

\subsubsection*{WizardScreen}
\label{subsubsec:wizard-screen}

\subsubsection*{WizardHeader}
\label{subsubsec:wizard-header}

\subsubsection*{WizardStepper}
\label{subsubsec:wizard-stepper}

\subsubsection*{Step}
\label{subsubsec:step}

\subsection{Scroll Date Picker}
\label{subsec:scroll-date-picker}

\subsection{Search Bar}
\label{subsec:search-bar}

\subsubsection*{CustomSearchBar}
\label{subsubsec:custom-search-bar}

\subsubsection*{CustomAutocomplete}
\label{subsubsec:custom-autocomplete}

\subsection{Text Box}
\label{subsec:text-box}

\subsection{Toogle Buttons}
\label{subsec:toogle-buttons}

\subsection{Warning Alert}
\label{subsec:warning-alert}

\section{Constants}
\label{sec:constants}

\section{Data}
\label{sec:data}

\section{Provider}
\label{sec:provider}

\section{Screens}
\label{sec:screens}

\section{Styles}
\label{sec:styles}

% SPIEGARE COME FUNZIONANO I TEMI, OVVERO CHE UNA VOLTA DEFINITO ADEGUATAMENTE, AI WIDGET IMPLEMENTATI VIENE, generalmente, APPLICATO in maniera AUTOMATICA, altre volte, se si vuole una ulteriore personalizzazione, si usa Theme.of(context).<nome_elemento>.<colore>\\

\section{Utils}
\label{sec:utils}

\section{main.dart}
\label{sec:main}

% Listato delle varie classi principali implementate
% Fai riferimento allo storico dei commits su GitHub

% \section{Design Pattern utilizzati}
