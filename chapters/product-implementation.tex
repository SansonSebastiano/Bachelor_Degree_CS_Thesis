\chapter{Implementazione}
\label{cap:implementazione}

\intro{Breve introduzione al capitolo}\\

Seguendo quanto descritto nel capitolo \ref{cap:progettazione}, si è proceduto con l'implementazione del prodotto software.\\
Di seguito verrà illustratta l'effettiva struttura del progetto (vedi sezione \ref{sec:struttura-progetto}), descrivendone poi la varie classi principali contenute in ciascuna cartella. \\

\begin{verbatim}
    lib/
        components/
        constants/
        data/
            model/
            service/
        provider/
        screens
        styles/
        utils/
        main.dart
\end{verbatim}

\section{Components}
\label{sec:components}

INSERIRE LE FIGURE???\\
SPIEGARE O RIMANDARE ALLA DOCUMENTAZIONE DI COME FUNZIONANO I WIDGET\\

In questa cartella sono raccolti tutti i \emph{widget} personalizzati, utilizzati dalle schermate presenti in \emph{screens} (vedi sezione \ref{sec:screens}).\\

\subsection{Alert Dialogs}
\label{subsec:alert-dialogs}

Nel file denominato \lstinline{alert_dialogs.dart} sono stati implementati i \emph{widget} che permetto di visualizzare un \emph{alert dialog} personalizzabile.\\

\subsubsection*{CustomBaseAlertDialog}
\label{subsubsec:custom-base-alert-dialog}

È una classe astratta che implementa un \lstinline{AlertDialog}\cite{site:alert-dialog}  definisce l'aspetto base di un \emph{alert dialog}, in cui sono sono presenti degli elementi fissi, come il titolo, un'icona posta sotto il titolo, il suo colore e \gls{padding}\glsoccur.\\
Mentre è possibile scegliere se aggiungere o meno un testo descrittivo, e se aggiungere o meno dei bottoni di conferma e/o annulla, in base all'operazione del contesto.\\

\subsubsection*{IconAlertDialog}
\label{subsubsec:icon-alert-dialog}

Classe concreta che implementa \lstinline{CustomBaseAlertDialog} ed imposta con dei valori fissi sia il \gls{padding}\glsoccur che la dimensione dell'icona.\\
Attraverso il costruttore è obbligatorio passare il \emph{widget} che rappresenta l'icona, il suo colore e un titolo, mentre è opzionale passare un testo descrittivo, e se aggiungere o meno dei bottoni di conferma e/o annulla.\\

\subsubsection*{LoadingAlertDialog}
\label{subsubsec:loading-alert-dialog}

Classe concreta che implementa \lstinline{IconAlertDialog} ed imposta come icona un \lstinline{CircularProgressIndicator}\cite{site:circular-progress-indicator}, che rappresenta un indicatore di caricamento.\\
È possibile decidere, attraveso il passaggio di parametri, il colore dell'indicatore e il titolo da visualizzare.\\

\subsubsection*{WarningAlertDialog}
\label{subsubsec:warning-alert-dialog}

\subsubsection*{ResponseDialog}
\label{subsubsec:response-dialog}

\subsubsection*{NetworkAlertDialog}
\label{subsubsec:network-alert-dialog}
\section{Constants}
\label{sec:constants}

\section{Data}
\label{sec:data}

\section{Provider}
\label{sec:provider}

\section{Screens}
\label{sec:screens}

\section{Styles}
\label{sec:styles}

\section{Utils}
\label{sec:utils}

\section{main.dart}
\label{sec:main}

% Listato delle varie classi principali implementate
% Fai riferimento allo storico dei commits su GitHub

% \section{Design Pattern utilizzati}

% \section{Codifica}
