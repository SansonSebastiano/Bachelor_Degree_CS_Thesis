\chapter{Implementazione}
\label{cap:implementazione}

\intro{In questo capitolo si discuterà dell'implementazione (o codifica) dell'applicazione in conseguenza alle scelte progettuali descritte nel capitolo precedente. Inoltre verranno descritte le librerie di terze parti utilizzate, motivandone la scelta.}\\

Seguendo quanto descritto nel capitolo \ref{cap:progettazione}, si è proceduto con l'implementazione del prodotto software.\\
Di seguito verrà illustratta l'effettiva struttura del progetto basata sull'approccio \emph{layer first} (vedi sezione \ref{sec:struttura-progetto}) implementata, descrivendone poi la varie classi contenute in ciascuna cartella. \\

\begin{verbatim}
    assets/
    lib/
        components/
        constants/
        data/
            model/
            service/
        provider/
        screens
        styles/
        utils/
        main.dart
\end{verbatim}

L'architettura menzionata nella sezione \ref{subsec:architettura-app} è stata implementata nel seguente modo: il \emph{data layer} e \emph{domain layer} sono stati implementati rispettivamente in \lstinline{service} e \lstinline{model}, contenute all'interno della cartella \lstinline{data}, mentre l'\emph{application layer} è stato implementato nella cartella \lstinline{provider} e il \emph{presentation layer} nella cartella \lstinline{screens}.\\
Le restanti cartelle sono servite come ausilio per l'implementazione delle funzionalità dei \emph{layer} sopraccitati, come ad esempio \lstinline{components} per la creazione di \emph{widget} personalizzati, \lstinline{constants} per la definizione di costanti, \lstinline{styles} per la definizione di stili e del tema dell'applicazione e \lstinline{utils} per la definizione di alcune funzioni di utilità.\\

\section{Components}
\label{sec:components}

INSERIRE LE FIGURE???\\
SPIEGARE O RIMANDARE ALLA DOCUMENTAZIONE DI COME FUNZIONANO I WIDGET\\

In questa cartella sono raccolti tutti i \emph{widget} personalizzati, utilizzati dalle schermate presenti in \emph{screens} (vedi sezione \ref{sec:screens}).\\
Di seguito verranno descritti i vari \emph{widget} implementati, suddivisi in base alla loro funzionalità.\\
Si specifica inoltre che la maggior parte di questi abbiano una visibilità pubblica, in quanto devono essere richiamabile dalle schermate, mentre alcuni hanno una visibilità privata, in quanto sono stati creati per essere utilizzati esclusivamente all'interno di altri \emph{widget}.\\

\subsection{Alert Dialogs}
\label{subsec:alert-dialogs}

Nel file denominato \lstinline{alert_dialogs.dart} sono stati implementati dei \emph{widget}, estendendo \lstinline{StatelessWidget}, che consentono di personalizzare un \emph{alert dialog}.\\

\subsubsection*{CustomBaseAlertDialog}
\label{subsubsec:custom-base-alert-dialog}

È una classe che implementa un \lstinline{AlertDialog}\cite{site:alert-dialog} e ne definisce l'aspetto base, fissando alcuni elementi, come \lstinline{Text}\cite{site:text} per il titolo, \lstinline{Widget} per l'icona posta sotto il titolo, il suo colore e \gls{padding}\glsoccur.\\
Mentre è possibile scegliere se aggiungere o meno un testo descrittivo e dei bottoni di conferma e/o annulla, in base all'operazione del contesto.\\

\subsubsection*{IconAlertDialog}
\label{subsubsec:icon-alert-dialog}

Classe che implementa \lstinline{CustomBaseAlertDialog} personalizzandolo ulteriormente impostando con dei valori fissi sia il \gls{padding}\glsoccur che la dimensione dell'icona.\\
Attraverso il costruttore è obbligatorio passare il \emph{widget} di tipo \lstinline{IconData}\cite{site:icon-data} che rappresenta l'icona, il suo colore e un titolo, mentre è opzionale passare un testo descrittivo, e se aggiungere o meno dei bottoni di conferma e/o annulla.\\

\subsubsection*{LoadingAlertDialog}
\label{subsubsec:loading-alert-dialog}

Personalizzazione di \lstinline{IconAlertDialog} in cui viene impostata come icona un \lstinline{CircularProgressIndicator}\cite{site:circular-progress-indicator}, che rappresenta un indicatore di caricamento.\\
È possibile decidere, attraveso il costruttore, il colore dell'indicatore e il titolo da visualizzare.\\
Lo scopo di questa classe è quella di essere utilizzata per mostrare un indicatore di caricamento durante l'esecuzione di un'operazione asincrona.\\

\subsubsection*{WarningAlertDialog}
\label{subsubsec:warning-alert-dialog}

Personalizzazione di \lstinline{IconAlertDialog} dove si richiede, nel costruttore, di passare un'icona e il suo colore, un titolo, il contenuto del testo descrittivo, l'azione che deve compiere il bottone di conferma alla sua pressione e il suo stile grafico (vedi sezione \ref{sec:styles}), mentre il pulsante di annullamento è stato fissato.\\
Lo scopo di questa classe è quella di essere utilizzata per mostrare un messaggio di avvertimento all'utente riguardante una scelta e la sua conferma.\\
\subsubsection*{ResponseDialog}
\label{subsubsec:response-dialog}

Personalizzazione di \lstinline{IconAlertDialog} dove si richiede, nel costruttore, di passare un'icona, il suo colore e un titolo.\\
Lo scopo di questa classe è quella di essere utilizzata per mostrare un messaggio di risposta all'utente riguardante l'esito di un'operazione.\\

\subsubsection*{NetworkAlertDialog}
\label{subsubsec:network-alert-dialog}

La sua implementazione è simile a \lstinline{WarningAlertDialog}, con la differenza che non è presente il pulsante di annullamento, in quanto il suo scopo è quello di notificare l'utente la mancanza di connessione ad internet e di cliccare il pulsante di conferma per chiudere l'\emph{alert dialog}.\\

\subsection{App Bars}
\label{subsec:app-bars}

Nel file denominato \lstinline{app_bars.dart} sono stati implementati dei widget, estendendo \lstinline{StatelessWidget}, che consentono di personalizzare un \emph{app bar}.\\

\subsubsection*{CustomAppBar}
\label{subsubsec:custom-app-bar}

Classe che implementa \lstinline{AppBar}\cite{site:app-bar} definendone il colore di background, applicato dal tema dell'applicazione (vedi sezione \ref{sec:styles}) e il titolo, in cui si tratta dell'applicazione di un'immagine vettoriale contenuta nella cartella \lstinline{assets} la quale rappresenta il logo dell'azienda.\\
Infine è possibile scegliere se aggiungere o meno un'icona, che rappresenta il pulsante di accesso alla schermata dell'account utente.\\
La motivazione di quest'ultima scelta è a causa del fatto che questa personalizzazione dell'\lstinline{AppBar} viene utilizzata per la quasi totalità delle viste, anche in quella dell'account utente, dove non si rende necessaria l'icona menzionata precedentemente poichè ci si trova già in tale vista.

\subsubsection*{LoginAppBar}
\label{subsubsec:login-app-bar}

Classe che implementa \lstinline{AppBar}, personalizzandolo appositamente per la schermata di \emph{login} (vedi sezione \ref{sec:screens}), che è simile a quella precedente, differenziandosi per la non presenza dell'icona che funge da pulsante di accesso alla schermata dell'account utente.

\subsection{Biomteric Switch}
\label{subsec:biometric-switch}

Nel file denominato \lstinline{biometric_switch.dart} è stato implementato un \emph{widget} che estende \lstinline{StatefulWidget} per costruire un componente in grado di permettere all'utente di abilitare il riconoscimento biometrico. \\
Questo \emph{widget} è composto da un \lstinline{Switch}\cite{site:switch} e un \lstinline{Text}\cite{site:text} che rappresenta il testo descrittivo.\\
Inoltre ne viene definito l'aspetto e il comportamento attraverso la definizione di un \lstinline{State} che estende \lstinline{StatefulWidget} implementando il metodo \lstinline{build} per la costruzione del \emph{widget} e il metodo \lstinline{onChanged} per la gestione dell'evento di cambiamento di stato del \lstinline{Switch}.\\
Stato che è rappresentato da due variabili di tipo \lstinline{bool}, una che si occupa di gestire l'abilitazione del componente e l'altro che indica se nel dispositivo in cui viene eseguita l'applicazione è supportato il riconoscimento biometrico.\\
La prima tra queste, ad ogni suo cambiamento di stato, viene salvata in locale attraverso una libreria di terze parti (di cui verrà discussa nella sezione \ref{sec:utils} - SPECIFICA LA PARTE PRECISA) per mantenere in memoria la preferenza dell'utente. \\
Il caso in cui la seconda variabile menzionata precedentemente sia \lstinline{false}, dunque non vi è il supporto per il riconoscimento biometrico, attraverso un \emph{widget} \lstinline{Text} si notifica l'utente di tale mancanza e si disabilita il \lstinline{Switch}.

\subsection{Date Picker}
\label{subsec:date-picker}

Nel file denominato \lstinline{date_picker.dart} sono stati implementati dei \emph{widget} che estendono sia \lstinline{StatelessWidget} che \lstinline{StatefulWidget} per costruire dei componenti in grado di permettere all'utente di selezionare una data o un intervallo di date.\\

\subsubsection*{DateButton}
\label{subsubsec:date-button}

Questo componente estende \lstinline{StatelessWidget} e si occupa di costruire un bottone che visualizza la data, o un intervallo di date, selezionata/e dall'utente.\\
Ha visibilità privata, poichè è stato creato per essere utilizzato esclusivamente all'interno dei \emph{widget} contenuti nello stesso file.\\
È composto da un \lstinline{TextButton}\cite{site:text-button}, in cui viene definito l'aspetto grafico, e un \lstinline{Text} che visualizza la data o un intervallo di date. \\
Attraverso il suo costruttore è possibile definire il testo da visualizzare e il comportamento del bottone alla sua pressione.\\

\subsubsection*{CustomDateRangePicker}
\label{subsubsec:custom-date-range-picker}

Questo componente estende \lstinline{StatefulWidget} e si occupa di costruire un \emph{widget} che permette all'utente di selezionare un intervallo di date.\\
È composto da \lstinline{DateButton}, al quale il primo parametro che viene passato è l'intervallo di date da visualizzare, di default o selezionato dall'utente, precedentemente formattato da un metodo privato \lstinline{dateFormatter}: riceve in input la data iniziale e finale dell'intervallo (nel caso in cui si intenda selezionare un singolo giorno è sufficiente impostarne con esso entrambi i cambi) e restituisce una stringa che rappresenta l'intervallo di date formattato opportunamente.\\
Mentre come secondo parametro viene passato un altro metodo privato \lstinline{show} che si occupa di visualizzare il calendario e di consentire all'utente di seleziona un intervallo. \\
Inoltre è presente un metodo pubblico \lstinline{onDateRangeSelected} che si occupa di gestire l'evento di selezione dell'intervallo di date, aggiornando lo stato del \emph{widget}. \\
Lo scopo di questo \emph{widget} è quello di fornire la possibilità all'utente di selezionare una singola data o intervallo di date per filtrare la lista di \emph{Meeting Note} (vedi requisiti \hyperref[RFN-18]{RFN-18} e \hyperref[RFN-19]{RFN-19}) e viene impiegato solamente all'interno del \lstinline{FilterPanel} (vedi sezione \ref{subsec:filter-panel}).

\subsubsection*{CustomDatePicker}
\label{subsubsec:custom-date-picker}

L'implementazione di questo componente è del tutto analoga a quella di \lstinline{CustomDateRangePicker}, con la differenza che permette all'utente di selezionare una singola data.\\
Mentre il suo scopo è quello di fornire all'utente, nel momento di revisione dei dati estratti dall'elaborazione del testo da parte di un algoritmo di \emph{intelligenza artificiale}, di modificare la data di una \emph{Meeting Note} da creare (vedi requisito \hyperref[RFN-53]{RFN-53}).

\subsection{Filter Panel}
\label{subsec:filter-panel}

Nel file denominato \lstinline{filter_panel.dart} è stato implementato un \emph{widget} che estende \lstinline{ConsumerStatefulWidget}\cite{site:reading-provider}, il cui comportamento è il medesimo di uno \lstinline{StatefulWidget}, con l'aggiunta che è in grado di leggere i dati forniti da un \emph{Provider} (vedi sezione \ref{subsec:riverpod}).
Tale componente è composto da quattro \emph{widget}, ciascuno dei quali consente di filtrare la lista di \emph{Meeting Note}  secondo i criteri definiti in fase di analisi dei requisiti (vedi requisiti dal \hyperref[RFN-15]{RFN-15} al \hyperref[RFN-19]{RFN-19} e dal \hyperref[RFN-31]{RFN-31} al \hyperref[RFN-33]{RFN-33}). \\
Di seguito verranno elencati i vari \emph{widget} che compongono il \emph{Filter Panel}: 
\begin{itemize}
    \item \lstinline{CustomObjectPicker} (componente la cui implementazione è discussa nella sezione \ref{subsec:object-picker});
    \item \lstinline{CustomAutocomplete} (componente la cui implementazione è discussa nella sezione \ref{subsubsec:custom-autocomplete});
    \item \lstinline{CustomDateRangePicker} (componente la cui implementazione è discussa nella sezione \ref{subsubsec:custom-date-range-picker});
    \item \lstinline{CustomToogleButtons} (componente la cui implementazione è discussa nella sezione \ref{subsec:toogle-buttons}).
\end{itemize}
Come da prassi, ne viene definito l'aspetto grafico e il comportamento che deve avere.\\
Per quanto riguarda il comportamento e dunque la gestione dello stato, che sostanzialmente consiste nel memorizzare e passare alla schermata dedicata (RIMANDARE ALLA SEZIONE PRECISA) quali filtri sono stati attivati dall'utente, tale operazione viene effettuata attraverso uno \lstinline{StateNotifierProvider} (vedi sezione \ref{subsec:riverpod}) e una classe \lstinline{FilteringOptions} (RIMANDARE ALLA SEZIONE PRECISA).\\

\subsection{Login Form}
\label{subsec:login-form}

Nel file denominato \lstinline{login_form.dart} è stato implementato un \emph{widget} che estende \lstinline{ConsumerStatefulWidget}.\\
Lo scopo di questo componente è quello di consentire all'utente di autenticarsi all'applicazione, per farlo ci sono due modi: inserendo manualmente le credenziali oppure utilizzando il riconoscimento biometrico.\\
Per soddisfare il primo caso (vedi requisito \hyperref[RFN-2]{RFN-2}), il componente è composto da due \lstinline{TextFormField}\cite{site:text-form-field} e un \lstinline{ElevatedButton}\cite{site:elevated-button} di cui vengono definiti l'aspetto grafico.\\
Il \lstinline{TextFormField} riguardante l'inserimento della password presenta anche un pulsante per mostrarla in chiaro o nasconderla.\\
Il pulsante di conferma è disabilitato fintanto che non vengono inserite entrambe le credenziali.\\
Successivamente attraverso l'ausilio del \emph{provider} \lstinline{authProvider} (SPECIFICA LA SEZIONE PRECISA) viene effettuata la richiesta di autenticazione, che in caso di successo porta alla schermata principale dell'applicazione, salvando il \emph{token} di autenticazione ricevuto (RIMANDA ALLA SEZIONE SPECIFICA), altrimenti viene mostrato un messaggio di errore (vedi \hyperref[RFN-4]{RFN-4}).\\ 
Mentre nel secondo caso (vedi \hyperref[RFN-3]{RFN-3}), si effettua un controllo per verificare se l'utente in precedenza aveva abilitato tale funzionalità, in caso affermativo viene eseguito il riconoscimento biometrico (RIMANDA ALLA SEZIONE SPECIFICA), altrimenti si prosegue con l'inserimento manuale delle credenziali.\\
L'operazione e l'esito dell'autenticazione attraverso il riconoscimento biometrico, che avviene sempre attraverso \lstinline{authProvider}, è simile a quello descritto per l'autenticazione manuale, con la differenza che quest'ultimo metodo preleva le credenziali dalla memoria locale, precedentemente salvate dall'operazione di abilitazione del riconoscimento biometrico da parte dell'utente (vedi sezione \ref{subsec:biometric-switch}).

\subsection{Meeting Note Card}
\label{subsec:meeting-note-card}

INSERIRE LE IMMAGINI \\

Nel file denominato \lstinline{meeting_note_card.dart} sono stati implementati dei \emph{widget} che estendono \lstinline{StatelessWidget} per costruire un componente in grado di visualizzare una \emph{Meeting Note} in una lista, e/o in dettaglio, con la possibilità di eliminarla e/o modificarla. \\
Per la costruzione di questo componente si sono creati vari \emph{widget} che ne definiscono l'aspetto grafico e di cui verranno illustratti di seguito.\\
Si specifica che tutti i dati necessari sono stati passati attraverso il costruttore.

\subsubsection*{MeetingNoteTitle}
\label{subsubsec:meeting-note-title}

Si tratta di un \emph{widget} che si occupa di disporre gli elementi che compongono il titolo di una 
\emph{Meeting Note}, ovvero l'identificativo del cliente e la data dell'incontro, in modo tale che entrambi siano posizionati sulla stessa riga, con il primo che occupa la maggior parte dello spazio e il secondo che viene posizionato a destra.\\
Inoltre è stato aggiunto un \lstinline{Divider}\cite{site:divider} che funge da separatore tra il titolo e il contenuto parziale della \emph{Meeting Note}, visualizzabile direttamente nella lista di \emph{Meeting Note}.

\subsubsection*{MeetingNoteItem}
\label{subsubsec:meeting-note-item}

\emph{Widget} che effettivamente costruisce l'\emph{item} della lista di \emph{Meeting Note} (vedi requisiti da \hyperref[RFN-8]{RFN-8} a \hyperref[RFN-11]{RFN-11}), composto da un \lstinline{ListTile}\cite{site:list-tile}, in cui nella proprietà \lstinline{title} viene posto un \lstinline{MeetingNoteTitle} e in \lstinline{subtitle} viene posto il contenuto della \emph{Meeting Note} stroncato ad un massimo di due righe di testo.\\
All'evento \lstinline{onTap} del componente, viene mostrato a schermo una modale che appare dal basso contenente i dettagli della \emph{Meeting Note}.\\
Si vuole precisare che contrariamente a quanto pensato durante la realizzazione del \gls{mockup}\glsoccur, si è deciso di apportare una modifica per quanto riguarda la visualizzazione dei dettagli della \emph{Meeting Note}, in quanto essendo l'applicazione usata da utenti in mobilità, è più ergonomico mostrare i dettagli in una modale che appare dal basso, piuttosto che nell'espansione di un item.

\subsubsection*{MeetingNoteCard}
\label{subsubsec:meeting-note-card}

Componente contenuto nella modale menzionata precedentemente, il quale si occupa di disporre tutti gli elementi di cui è composta una \emph{Meeting Note} (vedi requisiti dal \hyperref[RFN-35]{RFN-35} al \hyperref[RFN-42]{RFN-42}).\\
È composto da un \lstinline{MeetingNoteTitle}, il contenuto della \emph{Meeting Note}, l'autore e i due pulsanti che permettono di eliminarla e/o modificarla, il comportamento della prima azione viene passata per parametro, mentre per la seconda viene definita direttamente in quanto è stato sufficiente rimandare alla schermata del \emph{wizard} per poi effettuare le modifiche (RIMANDA ALLA SEZIONE SPECIFICA).

\subsection{Object Picker}
\label{subsec:object-picker}

Nel file denominato \lstinline{object_picker.dart} è stato implementato \lstinline{CustomObjectPicker} un \emph{widget} che estende \lstinline{StatefulWidget} per costruire un componente in grado di permettere all'utente di selezionare la categoria dei clienti (vedi requisiti \hyperref[RFN-16]{RFN-16}, \hyperref[RFN-66]{RFN-66}). \\
Viene utilizzato all'interno del \emph{Filter Panel} (vedi sezione \ref{subsec:filter-panel}), del \emph{wizard} di creazione/modifica di una \emph{Meeting Note} (vedi sezione DA SPECIFICARE) e nella schermata di revisione per la creazione automatica di una \emph{Meeting Note} (vedi sezione DA SPECIFICARE).\\
Anche per questo componente si è pensato di rivisitarlo rispetto al \gls{mockup}\glsoccur, in quanto si è deciso di utilizzare un \lstinline{CupertinoPicker}\cite{site:cupertino-picker}, che visualizza le categorie dei clienti selezionabili, passate in input attraverso il costruttore, al posto di un \lstinline{DropdownMenu}\cite{site:dropdown-menu}, questo per favorire l'utilizzo dell'applicazione da parte di utenti in mobilità, poichè il primo appare dal basso e dunque diventa più facilmente raggiungibile per il suo utilizzo. \\
Inoltre possiede dei metodi che permettono di ottenere la categoria selezionata attraverso l'indice e di aggiornare lo stato del \emph{widget} al cambiamento di quest'ultima.\\


\subsection{Recording Button}
\label{subsec:recording-button}

Nel file denominato \lstinline{recording_button.dart} è stato implementato un \emph{widget} che estende \lstinline{StatelessWidget} per costruire un componente in grado di permettere all'utente di attivare la dettatura vocale (vedi requisito \hyperref[RFN-73]{RFN-73}).\\
È composto da un \lstinline{Text}\cite{site:text} che esplicita all'utente lo scopo del pulsante, da un \lstinline{IconButton}\cite{site:icon-button} e da un \lstinline{AvatarGlow}\cite{site:avatar-glow} che rappresenta l'animazione che viene visualizzata quando la dettatura vocale è attiva.\\
L'attivazione avviene in base al valore della variabile \emph{booleana} \lstinline{isRecording} che viene passata attraverso il costruttore, come anche il comportamento del pulsante, che viene definito attraverso il metodo \lstinline{onPressed}.

\subsection{Screens Template}
\label{subsec:screens-template}

Nel file denominato \lstinline{screens_template.dart} sono stati implementati dei \emph{widget} che estendono \lstinline{StatelessWidget} per costruire diversi componenti che rappresentano gli elementi comuni delle schermate dell'applicazione.

\subsubsection*{BaseScreen}
\label{subsubsec:base-screen}

Classe in cui viene definito la struttura e l'aspetto base che dovranno avere tutte le schermate dell'applicazione.\\
È composto da un \lstinline{Scaffold}\cite{site:scaffold}, contenitore principale di tutti gli elementi grafici, in cui viene applicato il tema dell'applicazione (vedi sezione \ref{sec:styles}). \\
Sono state definite poi varie proprietà di base dello \lstinline{Scaffold}, tra cui \lstinline{appBar} con \lstinline{CustomAppBar} (vedi sezione \ref{subsubsec:custom-app-bar}) e un \lstinline{body}, passato per parametro del costruttore, in quanto ogni schermata ha il suo contenuto.\\
È stata inoltre definita la proprietà \lstinline{bottomNavigationBar} che non viene utilizzata da tutte le schermate, ma per ciascuna vi è la possibilità di implementarla nel caso cui si voglia aggiungere dei pulsanti di navigazione nella parte inferiore della schermata per avanzare o retrocedere tra le varie schermate dell'applicazione.\\
Per abilitare tali pulsanti è sufficiente passare, a seconda delle necessità, un \emph{widget} \lstinline{forwardButton} per la progressione e/o \lstinline{backButton} per la regressione.\\

\subsubsection*{WizardScreen}
\label{subsubsec:wizard-screen}

Classe che implementa \lstinline{BaseScreen} e che si occupa di fornire un \emph{template} per le schermate che compongono il \gls{wizard}\glsoccur di creazione/modifica di una \emph{Meeting Note}.\\
Imposta a \lstinline{true} la proprietà \lstinline{enableIcon} per mantenere il pulsante di accesso alla schermata dell'account utente.\\
Attraverso i parametri del costruttore è obbligatorio passare il \lstinline{body}, mentre sono opzionali \lstinline{forwardButton} e \lstinline{backButton}.\\

\subsubsection*{WizardHeader}
\label{subsubsec:wizard-header}

Classe che definisce l'aspetto grafico dell'\emph{header} del \gls{wizard}\glsoccur di creazione/modifica di una \emph{Meeting Note}.\\
É composto da un \lstinline{Text}\cite{site:text} che rappresenta il titolo della schermata, da un \lstinline{IconButton}\cite{site:icon-button} e da un \emph{widget} privato \lstinline{WizardStepper} (vedi sezione \ref{subsubsec:wizard-stepper}) che aiuta all'utente a capire in quale punto del processo di creazione/modifica si trova.\\
Il pulsante menzionato ha lo scopo di mostrare un \lstinline{WarningAlertDialog} (vedi sezione \ref{subsubsec:warning-alert-dialog}) chiedendo all'utente se vuole abbandonare il \gls{wizard}\glsoccur, in caso affermativo viene mostrata la schermata principale dell'applicazione, altrimenti viene chiusa la modale.\\    

\subsubsection*{WizardStepper}
\label{subsubsec:wizard-stepper}

Classe che è composta da tre \lstinline{Step} (vedi sezione \ref{subsubsec:step}), il numero massimo di passi pensato per il \gls{wizard}\glsoccur, inoltre riceve in input il numero di passo corrente in modo da evidenziare a che punto l'utente si trova nella progressione.
Si specifica che \lstinline{WizardStepper} è stato creato per essere utilizzato esclusivamente all'interno di questo \emph{widget}.\\

\subsubsection*{Step}
\label{subsubsec:step}

Classe che definisce l'aspetto grafico di un passo del \gls{wizard}\glsoccur di creazione/modifica di una \emph{Meeting Note}.\\
Sostanzialmente è composto da un \lstinline{Divider}\cite{site:divider} che rappresenta un passo, definendo due colori diversi per evidenziare il passo in cui l'utente si trova da quelli precedenti e/o successivi.

\subsection{Scroll Date Picker}
\label{subsec:scroll-date-picker}

Nel file denominato \lstinline{scroll_date_picker.dart} è stato definito la classe \lstinline{CustomScrollDatePicker} che estende \lstinline{StatefulWidget} per costruire un componente in grado di permettere all'utente di selezionare una data.\\
È composto da un \lstinline{Text}\cite{site:text} che visualizza la data selezionata, un \lstinline{ScrollDatePicker}\cite{site:scroll-date-picker} e un pulsante che reimposta la data selezionata a quella corrente.\\
Espone dei metodi che permettono di ottenere la data selezionata con \lstinline{selectedDate}, e di aggiornare lo stato del \emph{widget} al cambiamento di quest'ultima con \lstinline{onDateChanged}.\\
Lo scopo di tale componente è quello di fornire all'utente, nel momento di creazione/modifica di una \emph{Meeting Note}, la possibilità di selezionare una data (vedi requisito \hyperref[RFN-67]{RFN-67}).

\subsection{Search Bar}
\label{subsec:search-bar}

Nel file denominato \lstinline{search_bar.dart} sono stati implementati dei \emph{widget} che estendono \lstinline{StatelessWidget} per costruire dei componenti in grado di permettere all'utente di effettuare una ricerca.

\subsubsection*{CustomSearchBar}
\label{subsubsec:custom-search-bar}

Classe che implementa una \lstinline{SearchBar}\cite{site:search-bar} in cui viene definito pressochè solo l'aspetto grafico in quanto attraverso il costruttore è obbligatorio passare un \lstinline{TextEditingController}\cite{site:text-editing-controller}, che si occupa di gestire il testo inserito dall'utente, e un metodo \lstinline{onChanged} che si occupa di gestire l'evento di cambiamento del testo inserito dall'utente, mentre è opzionale passare il colore del \emph{background} del componente. \\
Lo scopo è quello di fornire all'utente la possibilità di effettuare una ricerca all'interno della lista dei clienti (vedi requisito \hyperref[RFN-65]{RFN-65}).

\subsubsection*{CustomAutocomplete}
\label{subsubsec:custom-autocomplete}

Classe che implementa \lstinline{Autocomplete}\cite{site:autocomplete}, consente di fornire dei suggerimenti per l'autocompletamento nella ricerca, in cui viene anche qui solo l'aspetto grafico. \\
Attraverso il costruttore si rende necessario passare dei metodi fondamentali: \lstinline{optionsBuilder} che si occupa di fornire i suggerimenti per l'autocompletamento, \lstinline{displayStringForOption} che si occupa di fornire la stringa da visualizzare per ogni suggerimento e \lstinline{onSelected} che si occupa di gestire l'evento di selezione di un suggerimento.\\
Viene impiegato nel \emph{widget} \lstinline{FilterPanel} (vedi sezione \ref{subsec:filter-panel}) per fornire all'utente la possibilità di ricercare un cliente (vedi requisito \hyperref[RFN-16]{RFN-16}).

\subsection{Text Box}
\label{subsec:text-box}

Nel file denominato \lstinline{text_box.dart} è stato implementato un \emph{widget} che estende \lstinline{StatefulWidget} per costruire un componente in grado di permettere all'utente di inserire del testo.\\
È composto da semplicemente un \lstinline{TextField}\cite{site:text-field} in cui viene personalizzato nell'aspetto grafico, viene richiesto dal costruttore di passare l'altezza di questo componente, in modo da poterlo utilizzare in diverse situazioni, poi è necessatio passare un \lstinline{TextEditingController}\cite{site:text-editing-controller} e un metodo \lstinline{onChanged} che si occupa di gestire l'evento di cambiamento del testo inserito dall'utente.\\
Viene utilizzato per la creazione/modifica di una \emph{Meeting Note} (vedi requisito \hyperref[RFN-72]{RFN-72}) e per la creazione automatica (vedi requisito \hyperref[RFD-58]{RFD-58}).

\subsection{Toogle Buttons}
\label{subsec:toogle-buttons}

Nel file denominato \lstinline{toogle_buttons.dart} è stato implementato un \emph{widget} che estende \lstinline{StatefulWidget} per costruire un componente in grado di permettere all'utente di selezionare una delle due opzioni disponibili.\\
È composto da un \lstinline{ToggleButtons}\cite{site:toggle-buttons} di cui viene definito l'aspetto grafico e il comportamento.\\
Attraverso il costruttore è necessario passare un \lstinline{List<Widget>} contenente le due opzioni disponibili, un \lstinline{List<bool>} che indica quale opzione è stata selezionata e un metodo \lstinline{onChoiceSelected} che si occupa di gestire l'evento di selezione di un'opzione.\\
Viene impiegato nel \lstinline{FilterPanel} (vedi sezione \ref{subsec:filter-panel}) per fornire all'utente la possibilità di selezionare una delle due opzioni disponibili per filtrare la lista di \emph{Meeting Note} (vedi requisiti \hyperref[RFN-31]{RFN-31} e \hyperref[RFN-32]{RFN-32}).

\subsection{Warning Alert}
\label{subsec:warning-alert}

Nel file denominato \lstinline{warning_alert.dart} è stato implementato un \emph{widget} che estende \lstinline{StatelessWidget} per costruire un componente in grado di mostrare un messaggio di avvertimento all'utente, i casi in cui viene richiesto si faccia riferimento ai seguenti requisiti: \hyperref[RFN-12]{RFN-12}, \hyperref[RFN-13]{RFN-13}, \hyperref[RFN-20]{RFN-20}, \hyperref[RFN-30]{RFN-30}, \hyperref[RFN-34]{RFN-34}, \hyperref[RFN-64]{RFN-64}. \\
È composto da un \lstinline{Text}\cite{site:text} e un \lstinline{Icon}\cite{site:icon}, il loro scopo è quello di visualizzare il messaggio di avvertimento, attraverso i parametri del costruttore viene passato il messaggio e il colore che devono avere il testo e l'icona. 

\section{Constants}
\label{sec:constants}

\section{Data}
\label{sec:data}

\section{Provider}
\label{sec:provider}

\section{Screens}
\label{sec:screens}

\section{Styles}
\label{sec:styles}

% SPIEGARE COME FUNZIONANO I TEMI, OVVERO CHE UNA VOLTA DEFINITO ADEGUATAMENTE, AI WIDGET IMPLEMENTATI VIENE, generalmente, APPLICATO in maniera AUTOMATICA, altre volte, se si vuole una ulteriore personalizzazione, si usa Theme.of(context).<nome_elemento>.<colore>\\

\section{Utils}
\label{sec:utils}

\section{main.dart}
\label{sec:main}

% Listato delle varie classi principali implementate
% Fai riferimento allo storico dei commits su GitHub

% \section{Design Pattern utilizzati}
