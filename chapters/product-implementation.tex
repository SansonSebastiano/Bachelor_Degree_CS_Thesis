\chapter{Implementazione}
\label{cap:implementazione}

\intro{In questo capitolo si discuterà dell'implementazione (o codifica) dell'applicazione in conseguenza alle scelte progettuali descritte nel capitolo precedente. Inoltre verranno descritte le librerie di terze parti utilizzate, motivandone la scelta.}\\

Seguendo quanto descritto nel capitolo \ref{cap:progettazione}, si è proceduto con l'implementazione del prodotto software.\\
Di seguito verrà illustratta l'effettiva struttura del progetto basata sull'approccio \emph{layer first} (vedi sezione \ref{sec:struttura-progetto}) implementata, descrivendone poi la varie classi contenute in ciascuna cartella. \\

\begin{verbatim}
    assets/
    lib/
        components/
        constants/
        data/
            model/
            service/
        provider/
        screens
        styles/
        utils/
        main.dart
\end{verbatim}

L'architettura menzionata nella sezione \ref{subsec:architettura-app} è stata implementata nel seguente modo: il \emph{data layer} e \emph{domain layer} sono stati implementati rispettivamente in \lstinline{service} e \lstinline{model}, contenute all'interno della cartella \lstinline{data}, mentre l'\emph{application layer} è stato implementato nella cartella \lstinline{provider} e il \emph{presentation layer} nella cartella \lstinline{screens}.\\
Le restanti cartelle sono servite come ausilio per l'implementazione delle funzionalità dei \emph{layer} sopracitati, come ad esempio \lstinline{components} per la creazione di \emph{widget} personalizzati, \lstinline{constants} per la definizione di costanti, \lstinline{styles} per la definizione di stili grafici e del tema dell'applicazione e \lstinline{utils} per la definizione di alcune funzioni di utilità.\\

\section{Components}
\label{sec:components}

INSERIRE LE FIGURE???\\
SPIEGARE O RIMANDARE ALLA DOCUMENTAZIONE DI COME FUNZIONANO I WIDGET\\

In questa cartella sono raccolti tutti i \emph{widget} personalizzati, utilizzati dalle schermate presenti in \emph{screens} (vedi sezione \ref{sec:screens}).\\
Di seguito verranno descritti i vari \emph{widget} implementati, suddivisi in base alla loro funzionalità.\\
Si specifica inoltre che la maggior parte di questi abbiano una visibilità pubblica, in quanto devono essere richiamabile dalle schermate, mentre alcuni hanno una visibilità privata, in quanto sono stati creati per essere utilizzati esclusivamente all'interno di altri \emph{widget}.\\

\subsection{Alert Dialogs}
\label{subsec:alert-dialogs}

Nel file denominato \lstinline{alert_dialogs.dart} sono stati implementati dei \emph{widget}, estendendo \lstinline{StatelessWidget}, che consentono di personalizzare un \emph{alert dialog}.\\

\subsubsection*{CustomBaseAlertDialog}
\label{subsubsec:custom-base-alert-dialog}

È una classe che implementa un \lstinline{AlertDialog}\cite{site:alert-dialog} e ne definisce l'aspetto base, fissando alcuni elementi, come \lstinline{Text}\cite{site:text} per il titolo, \lstinline{Widget} per l'icona posta sotto il titolo, il suo colore e \gls{padding}\glsoccur.\\
Mentre è possibile scegliere se aggiungere o meno un testo descrittivo e dei bottoni di conferma e/o annulla, in base all'operazione del contesto.\\

\subsubsection*{IconAlertDialog}
\label{subsubsec:icon-alert-dialog}

Classe che implementa \lstinline{CustomBaseAlertDialog} personalizzandolo ulteriormente impostando con dei valori fissi sia il \gls{padding}\glsoccur che la dimensione dell'icona.\\
Attraverso il costruttore è obbligatorio passare il \emph{widget} di tipo \lstinline{IconData}\cite{site:icon-data} che rappresenta l'icona, il suo colore e un titolo, mentre è opzionale passare un testo descrittivo, e se aggiungere o meno dei bottoni di conferma e/o annulla.\\

\subsubsection*{LoadingAlertDialog}
\label{subsubsec:loading-alert-dialog}

Personalizzazione di \lstinline{IconAlertDialog} in cui viene impostata come icona un \lstinline{CircularProgressIndicator}\cite{site:circular-progress-indicator}, che rappresenta un indicatore di caricamento.\\
È possibile decidere, attraveso il costruttore, il colore dell'indicatore e il titolo da visualizzare.\\
Lo scopo di questa classe è quella di essere utilizzata per mostrare un indicatore di caricamento durante l'esecuzione di un'operazione asincrona.\\

\subsubsection*{WarningAlertDialog}
\label{subsubsec:warning-alert-dialog}

Personalizzazione di \lstinline{IconAlertDialog} dove si richiede, nel costruttore, di passare un'icona e il suo colore, un titolo, il contenuto del testo descrittivo, l'azione che deve compiere il bottone di conferma alla sua pressione e il suo stile grafico (vedi sezione \ref{sec:styles}), mentre il pulsante di annullamento è stato fissato.\\
Lo scopo di questa classe è quella di essere utilizzata per mostrare un messaggio di avvertimento all'utente riguardante una scelta e la sua conferma.\\
\subsubsection*{ResponseDialog}
\label{subsubsec:response-dialog}

Personalizzazione di \lstinline{IconAlertDialog} dove si richiede, nel costruttore, di passare un'icona, il suo colore e un titolo.\\
Lo scopo di questa classe è quella di essere utilizzata per mostrare un messaggio di risposta all'utente riguardante l'esito di un'operazione.\\

\subsubsection*{NetworkAlertDialog}
\label{subsubsec:network-alert-dialog}

La sua implementazione è simile a \lstinline{WarningAlertDialog}, con la differenza che non è presente il pulsante di annullamento, in quanto il suo scopo è quello di notificare l'utente la mancanza di connessione ad internet e di cliccare il pulsante di conferma per chiudere l'\emph{alert dialog}.\\

\subsection{App Bars}
\label{subsec:app-bars}

Nel file denominato \lstinline{app_bars.dart} sono stati implementati dei widget, estendendo \lstinline{StatelessWidget}, che consentono di personalizzare un \emph{app bar}.\\

\subsubsection*{CustomAppBar}
\label{subsubsec:custom-app-bar}

Classe che implementa \lstinline{AppBar}\cite{site:app-bar} definendone il colore di background, applicato dal tema dell'applicazione (vedi sezione \ref{sec:styles}) e il titolo, in cui si tratta dell'applicazione di un'immagine vettoriale contenuta nella cartella \lstinline{assets} la quale rappresenta il logo dell'azienda.\\
Infine è possibile scegliere se aggiungere o meno un'icona, che rappresenta il pulsante di accesso alla schermata dell'account utente.\\
La motivazione di quest'ultima scelta è a causa del fatto che questa personalizzazione dell'\lstinline{AppBar} viene utilizzata per la quasi totalità delle viste, anche in quella dell'account utente, dove non si rende necessaria l'icona menzionata precedentemente poichè ci si trova già in tale vista.

\subsubsection*{LoginAppBar}
\label{subsubsec:login-app-bar}

Classe che implementa \lstinline{AppBar}, personalizzandolo appositamente per la schermata di \emph{login} (vedi sezione \ref{sec:screens}), che è simile a quella precedente, differenziandosi per la non presenza dell'icona che funge da pulsante di accesso alla schermata dell'account utente.

\subsection{Biomteric Switch}
\label{subsec:biometric-switch}

Nel file denominato \lstinline{biometric_switch.dart} è stato implementato un \emph{widget} che estende \lstinline{StatefulWidget} per costruire un componente in grado di permettere all'utente di abilitare il riconoscimento biometrico. \\
Questo \emph{widget} è composto da un \lstinline{Switch}\cite{site:switch} e un \lstinline{Text}\cite{site:text} che rappresenta il testo descrittivo.\\
Inoltre ne viene definito l'aspetto e il comportamento attraverso la definizione di un \lstinline{State} che estende \lstinline{StatefulWidget} implementando il metodo \lstinline{build} per la costruzione del \emph{widget} e il metodo \lstinline{onChanged} per la gestione dell'evento di cambiamento di stato del \lstinline{Switch}.\\
Stato che è rappresentato da due variabili di tipo \lstinline{bool}, una che si occupa di gestire l'abilitazione del componente e l'altro che indica se nel dispositivo in cui viene eseguita l'applicazione è supportato il riconoscimento biometrico.\\
La prima tra queste, ad ogni suo cambiamento di stato, viene salvata in locale attraverso una libreria di terze parti (di cui verrà discussa nella sezione \ref{sec:utils} - SPECIFICA LA PARTE PRECISA) per mantenere in memoria la preferenza dell'utente. \\
Il caso in cui la seconda variabile menzionata precedentemente sia \lstinline{false}, dunque non vi è il supporto per il riconoscimento biometrico, attraverso un \emph{widget} \lstinline{Text} si notifica l'utente di tale mancanza e si disabilita lo \lstinline{Switch}.

\subsection{Date Picker}
\label{subsec:date-picker}

Nel file denominato \lstinline{date_picker.dart} sono stati implementati dei \emph{widget} che estendono sia \lstinline{StatelessWidget} che \lstinline{StatefulWidget} per costruire dei componenti in grado di permettere all'utente di selezionare una data o un intervallo di date.\\

\subsubsection*{DateButton}
\label{subsubsec:date-button}

Questo componente estende \lstinline{StatelessWidget} e si occupa di costruire un bottone che visualizza la data, o un intervallo di date, selezionata/e dall'utente.\\
Ha visibilità privata, poichè è stato creato per essere utilizzato esclusivamente all'interno dei \emph{widget} contenuti nello stesso file.\\
È composto da un \lstinline{TextButton}\cite{site:text-button}, in cui viene definito l'aspetto grafico, e un \lstinline{Text} che visualizza la data o un intervallo di date. \\
Attraverso il suo costruttore è possibile definire il testo da visualizzare e il comportamento del bottone alla sua pressione.\\

\subsubsection*{CustomDateRangePicker}
\label{subsubsec:custom-date-range-picker}

Questo componente estende \lstinline{StatefulWidget} e si occupa di costruire un \emph{widget} che permette all'utente di selezionare un intervallo di date.\\
È composto da \lstinline{DateButton}, al quale il primo parametro che viene passato è l'intervallo di date da visualizzare, di default o selezionato dall'utente, precedentemente formattato da un metodo privato \lstinline{dateFormatter}: riceve in input la data iniziale e finale dell'intervallo (nel caso in cui si intenda selezionare un singolo giorno è sufficiente impostarne con esso entrambi i cambi) e restituisce una stringa che rappresenta l'intervallo di date formattato opportunamente.\\
Mentre come secondo parametro viene passato un altro metodo privato \lstinline{show} che si occupa di visualizzare il calendario e di consentire all'utente di seleziona un intervallo. \\
Inoltre è presente un metodo pubblico \lstinline{onDateRangeSelected} che si occupa di gestire l'evento di selezione dell'intervallo di date, aggiornando lo stato del \emph{widget}. \\
Lo scopo di questo \emph{widget} è quello di fornire la possibilità all'utente di selezionare una singola data o intervallo di date per filtrare la lista di \emph{Meeting Note} (vedi requisiti \hyperref[RFN-18]{RFN-18} e \hyperref[RFN-19]{RFN-19}) e viene impiegato solamente all'interno del \lstinline{FilterPanel} (vedi sezione \ref{subsec:filter-panel}).

\subsubsection*{CustomDatePicker}
\label{subsubsec:custom-date-picker}

L'implementazione di questo componente è del tutto analoga a quella di \lstinline{CustomDateRangePicker}, con la differenza che permette all'utente di selezionare una singola data.\\
Mentre il suo scopo è quello di fornire all'utente, nel momento di revisione dei dati estratti dall'elaborazione del testo da parte di un algoritmo di \emph{intelligenza artificiale}, di modificare la data di una \emph{Meeting Note} da creare (vedi requisito \hyperref[RFN-53]{RFN-53}).

\subsection{Filter Panel}
\label{subsec:filter-panel}

Nel file denominato \lstinline{filter_panel.dart} è stato implementato un \emph{widget} che estende \lstinline{ConsumerStatefulWidget}\cite{site:reading-provider}, il cui comportamento è il medesimo di uno \lstinline{StatefulWidget}, con l'aggiunta che è in grado di leggere i dati forniti da un \emph{Provider} (vedi sezione \ref{subsec:riverpod}).
Tale componente è composto da quattro \emph{widget}, ciascuno dei quali consente di filtrare la lista di \emph{Meeting Note}  secondo i criteri definiti in fase di analisi dei requisiti (vedi requisiti dal \hyperref[RFN-15]{RFN-15} al \hyperref[RFN-19]{RFN-19} e dal \hyperref[RFN-31]{RFN-31} al \hyperref[RFN-33]{RFN-33}). \\
Di seguito verranno elencati i vari \emph{widget} che compongono il \emph{Filter Panel}: 
\begin{itemize}
    \item \lstinline{CustomObjectPicker} (componente la cui implementazione è discussa nella sezione \ref{subsec:object-picker});
    \item \lstinline{CustomAutocomplete} (componente la cui implementazione è discussa nella sezione \ref{subsubsec:custom-autocomplete});
    \item \lstinline{CustomDateRangePicker} (componente la cui implementazione è discussa nella sezione \ref{subsubsec:custom-date-range-picker});
    \item \lstinline{CustomToogleButtons} (componente la cui implementazione è discussa nella sezione \ref{subsec:toogle-buttons}).
\end{itemize}
Come da prassi, ne viene definito l'aspetto grafico e il comportamento che deve avere.\\
Per quanto riguarda il comportamento e dunque la gestione dello stato, che sostanzialmente consiste nel memorizzare e passare alla schermata dedicata (RIMANDARE ALLA SEZIONE PRECISA) quali filtri sono stati attivati dall'utente, tale operazione viene effettuata attraverso uno \lstinline{StateNotifierProvider} (vedi sezione \ref{subsec:riverpod}) e una classe \lstinline{FilteringOptions} (RIMANDARE ALLA SEZIONE PRECISA).\\

\subsection{Login Form}
\label{subsec:login-form}

Nel file denominato \lstinline{login_form.dart} è stato implementato un \emph{widget} che estende \lstinline{ConsumerStatefulWidget} (discusso nella sezione \ref{subsec:riverpod}).\\
Lo scopo di questo componente è quello di consentire all'utente di autenticarsi all'applicazione, per farlo ci sono due modi: inserendo manualmente le credenziali oppure utilizzando il riconoscimento biometrico.\\
Per soddisfare il primo caso (vedi requisito \hyperref[RFN-2]{RFN-2}), il componente è composto da due \lstinline{TextFormField}\cite{site:text-form-field} e un \lstinline{ElevatedButton}\cite{site:elevated-button} di cui vengono definiti l'aspetto grafico.\\
Il \lstinline{TextFormField} riguardante l'inserimento della password presenta anche un pulsante per mostrarla in chiaro o nasconderla.\\
Il pulsante di conferma è disabilitato fintanto che non vengono inserite entrambe le credenziali.\\
Successivamente attraverso l'ausilio del \emph{provider} \lstinline{authProvider} (vedi sezione \ref{subsec:authentication-provider}) viene effettuata la richiesta di autenticazione, che in caso di successo porta alla schermata principale dell'applicazione, salvando il \emph{token} di autenticazione ricevuto (RIMANDA ALLA SEZIONE SPECIFICA), altrimenti viene mostrato un messaggio di errore (vedi \hyperref[RFN-4]{RFN-4}).\\ 
Mentre nel secondo caso (vedi \hyperref[RFN-3]{RFN-3}), si effettua un controllo per verificare se l'utente in precedenza aveva abilitato tale funzionalità, in caso affermativo viene eseguito il riconoscimento biometrico (RIMANDA ALLA SEZIONE SPECIFICA), altrimenti si prosegue con l'inserimento manuale delle credenziali.\\
L'operazione e l'esito dell'autenticazione attraverso il riconoscimento biometrico, che avviene sempre attraverso \lstinline{authProvider}, è simile a quello descritto per l'autenticazione manuale, con la differenza che quest'ultimo metodo preleva le credenziali dalla memoria locale, precedentemente salvate dall'operazione di abilitazione del riconoscimento biometrico da parte dell'utente (vedi sezione \ref{subsec:biometric-switch}). \\
Inoltre è presente una variabile di stato \lstinline{isVerification} che se impostata a \lstinline{true} indica che questa \emph{form} viene impiegata come verifica delle credenziali per confermare l'abilitazione del riconscimento biometrico (vedi sezione \ref{subsec:account-screen}).

\subsection{Meeting Note Card}
\label{subsec:meeting-note-card}

INSERIRE LE IMMAGINI \\
RICHIAMARE I REQUISITI CORRELATI \\

Nel file denominato \lstinline{meeting_note_card.dart} sono stati implementati dei \emph{widget} che estendono \lstinline{StatelessWidget} per costruire un componente in grado di visualizzare una \emph{Meeting Note} in una lista, e/o in dettaglio, con la possibilità di eliminarla e/o modificarla. \\
Per la costruzione di questo componente si sono creati vari \emph{widget} che ne definiscono l'aspetto grafico e di cui verranno illustratti di seguito.\\
Si specifica che tutti i dati necessari sono stati passati attraverso il costruttore.

\subsubsection*{MeetingNoteTitle}
\label{subsubsec:meeting-note-title}

Si tratta di un \emph{widget} che si occupa di disporre gli elementi che compongono il titolo di una 
\emph{Meeting Note}, ovvero l'identificativo del cliente e la data dell'incontro, in modo tale che entrambi siano posizionati sulla stessa riga, con il primo che occupa la maggior parte dello spazio e il secondo che viene posizionato a destra.\\
Inoltre è stato aggiunto un \lstinline{Divider}\cite{site:divider} che funge da separatore tra il titolo e il contenuto parziale della \emph{Meeting Note}, visualizzabile direttamente nella lista di \emph{Meeting Note}.

\subsubsection*{MeetingNoteItem}
\label{subsubsec:meeting-note-item}

\emph{Widget} che effettivamente costruisce l'\emph{item} della lista di \emph{Meeting Note} (vedi requisiti da \hyperref[RFN-8]{RFN-8} a \hyperref[RFN-11]{RFN-11}), composto da un \lstinline{ListTile}\cite{site:list-tile}, in cui nella proprietà \lstinline{title} viene posto un \lstinline{MeetingNoteTitle} e in \lstinline{subtitle} viene posto il contenuto della \emph{Meeting Note} stroncato ad un massimo di due righe di testo.\\
All'evento \lstinline{onTap} del componente, viene mostrato a schermo una modale che appare dal basso contenente i dettagli della \emph{Meeting Note}.\\
Si vuole precisare che contrariamente a quanto pensato durante la realizzazione del \gls{mockup}\glsoccur, si è deciso di apportare una modifica per quanto riguarda la visualizzazione dei dettagli della \emph{Meeting Note}, in quanto essendo l'applicazione usata da utenti in mobilità, è più ergonomico mostrare i dettagli in una modale che appare dal basso, piuttosto che nell'espansione di un item.

\subsubsection*{MeetingNoteCard}
\label{subsubsec:meeting-note-card}

Componente contenuto nella modale menzionata precedentemente, il quale si occupa di disporre tutti gli elementi di cui è composta una \emph{Meeting Note} (vedi requisiti dal \hyperref[RFN-35]{RFN-35} al \hyperref[RFN-42]{RFN-42}).\\
È composto da un \lstinline{MeetingNoteTitle}, il contenuto della \emph{Meeting Note}, l'autore e i due pulsanti che permettono di eliminarla e/o modificarla, il comportamento della prima azione viene passata per parametro, mentre per la seconda viene definita direttamente in quanto è stato sufficiente rimandare alla schermata del \emph{wizard} per poi effettuare le modifiche (RIMANDA ALLA SEZIONE SPECIFICA).

\subsection{Object Picker}
\label{subsec:object-picker}

Nel file denominato \lstinline{object_picker.dart} è stato implementato \lstinline{CustomObjectPicker} un \emph{widget} che estende \lstinline{StatefulWidget} per costruire un componente in grado di permettere all'utente di selezionare la categoria dei clienti (vedi requisiti \hyperref[RFN-16]{RFN-16}, \hyperref[RFN-66]{RFN-66}). \\
Viene utilizzato all'interno del \emph{Filter Panel} (vedi sezione \ref{subsec:filter-panel}), del \emph{wizard} di creazione/modifica di una \emph{Meeting Note} (vedi sezione DA SPECIFICARE) e nella schermata di revisione per la creazione automatica di una \emph{Meeting Note} (vedi sezione DA SPECIFICARE).\\
Anche per questo componente si è pensato di rivisitarlo rispetto al \gls{mockup}\glsoccur, in quanto si è deciso di utilizzare un \lstinline{CupertinoPicker}\cite{site:cupertino-picker}, che visualizza le categorie dei clienti selezionabili, passate in input attraverso il costruttore, al posto di un \lstinline{DropdownMenu}\cite{site:dropdown-menu}, questo per favorire l'utilizzo dell'applicazione da parte di utenti in mobilità, poichè il primo appare dal basso e dunque diventa più facilmente raggiungibile per il suo utilizzo. \\
Inoltre possiede dei metodi che permettono di ottenere la categoria selezionata attraverso l'indice e di aggiornare lo stato del \emph{widget} al cambiamento di quest'ultima.\\


\subsection{Recording Button}
\label{subsec:recording-button}

Nel file denominato \lstinline{recording_button.dart} è stato implementato un \emph{widget} che estende \lstinline{StatelessWidget} per costruire un componente in grado di permettere all'utente di attivare la dettatura vocale (vedi requisito \hyperref[RFN-73]{RFN-73}).\\
È composto da un \lstinline{Text}\cite{site:text} che esplicita all'utente lo scopo del pulsante, da un \lstinline{IconButton}\cite{site:icon-button} e da un \lstinline{AvatarGlow}\cite{site:avatar-glow} che rappresenta l'animazione che viene visualizzata quando la dettatura vocale è attiva.\\
L'attivazione avviene in base al valore della variabile \emph{booleana} \lstinline{isRecording} che viene passata attraverso il costruttore, come anche il comportamento del pulsante, che viene definito attraverso il metodo \lstinline{onPressed}.

\subsection{Screens Template}
\label{subsec:screens-template}

Nel file denominato \lstinline{screens_template.dart} sono stati implementati dei \emph{widget} che estendono \lstinline{StatelessWidget} per costruire diversi componenti che rappresentano gli elementi comuni delle schermate dell'applicazione.

\subsubsection*{BaseScreen}
\label{subsubsec:base-screen}

Classe in cui viene definito la struttura e l'aspetto base che dovranno avere tutte le schermate dell'applicazione.\\
È composto da un \lstinline{Scaffold}\cite{site:scaffold}, contenitore principale di tutti gli elementi grafici, in cui viene applicato il tema dell'applicazione (vedi sezione \ref{sec:styles}). \\
Sono state definite poi varie proprietà di base dello \lstinline{Scaffold}, tra cui \lstinline{appBar} con \lstinline{CustomAppBar} (vedi sezione \ref{subsubsec:custom-app-bar}) e un \lstinline{body}, passato per parametro del costruttore, in quanto ogni schermata ha il suo contenuto.\\
È stata inoltre definita la proprietà \lstinline{bottomNavigationBar} che non viene utilizzata da tutte le schermate, ma per ciascuna vi è la possibilità di implementarla nel caso cui si voglia aggiungere dei pulsanti di navigazione nella parte inferiore della schermata per avanzare o retrocedere tra le varie schermate dell'applicazione.\\
Per abilitare tali pulsanti è sufficiente passare, a seconda delle necessità, un \emph{widget} \lstinline{forwardButton} per la progressione e/o \lstinline{backButton} per la regressione.\\

\subsubsection*{WizardScreen}
\label{subsubsec:wizard-screen}

Classe che implementa \lstinline{BaseScreen} e che si occupa di fornire un \emph{template} per le schermate che compongono il \gls{wizard}\glsoccur di creazione/modifica di una \emph{Meeting Note}.\\
Imposta a \lstinline{true} la proprietà \lstinline{enableIcon} per mantenere il pulsante di accesso alla schermata dell'account utente.\\
Attraverso i parametri del costruttore è obbligatorio passare il \lstinline{body}, mentre sono opzionali \lstinline{forwardButton} e \lstinline{backButton}.\\

\subsubsection*{WizardHeader}
\label{subsubsec:wizard-header}

Classe che definisce l'aspetto grafico dell'\emph{header} del \gls{wizard}\glsoccur di creazione/modifica di una \emph{Meeting Note}.\\
É composto da un \lstinline{Text}\cite{site:text} che rappresenta il titolo della schermata, da un \lstinline{IconButton}\cite{site:icon-button} e da un \emph{widget} privato \lstinline{WizardStepper} (vedi sezione \ref{subsubsec:wizard-stepper}) che aiuta all'utente a capire in quale punto del processo di creazione/modifica si trova.\\
Il pulsante menzionato ha lo scopo di mostrare un \lstinline{WarningAlertDialog} (vedi sezione \ref{subsubsec:warning-alert-dialog}) chiedendo all'utente se vuole abbandonare il \gls{wizard}\glsoccur, in caso affermativo viene mostrata la schermata principale dell'applicazione, altrimenti viene chiusa la modale.\\    

\subsubsection*{WizardStepper}
\label{subsubsec:wizard-stepper}

Classe che è composta da tre \lstinline{Step} (vedi sezione \ref{subsubsec:step}), il numero massimo di passi pensato per il \gls{wizard}\glsoccur, inoltre riceve in input il numero di passo corrente in modo da evidenziare a che punto l'utente si trova nella progressione.
Si specifica che \lstinline{WizardStepper} è stato creato per essere utilizzato esclusivamente all'interno di questo \emph{widget}.\\

\subsubsection*{Step}
\label{subsubsec:step}

Classe che definisce l'aspetto grafico di un passo del \gls{wizard}\glsoccur di creazione/modifica di una \emph{Meeting Note}.\\
Sostanzialmente è composto da un \lstinline{Divider}\cite{site:divider} che rappresenta un passo, definendo due colori diversi per evidenziare il passo in cui l'utente si trova da quelli precedenti e/o successivi.

\subsection{Scroll Date Picker}
\label{subsec:scroll-date-picker}

Nel file denominato \lstinline{scroll_date_picker.dart} è stato definito la classe \lstinline{CustomScrollDatePicker} che estende \lstinline{StatefulWidget} per costruire un componente in grado di permettere all'utente di selezionare una data.\\
È composto da un \lstinline{Text}\cite{site:text} che visualizza la data selezionata, un \lstinline{ScrollDatePicker}\cite{site:scroll-date-picker} e un pulsante che reimposta la data selezionata a quella corrente.\\
Espone dei metodi che permettono di ottenere la data selezionata con \lstinline{selectedDate}, e di aggiornare lo stato del \emph{widget} al cambiamento di quest'ultima con \lstinline{onDateChanged}.\\
Lo scopo di tale componente è quello di fornire all'utente, nel momento di creazione/modifica di una \emph{Meeting Note}, la possibilità di selezionare una data (vedi requisito \hyperref[RFN-67]{RFN-67}).

\subsection{Search Bar}
\label{subsec:search-bar}

Nel file denominato \lstinline{search_bar.dart} sono stati implementati dei \emph{widget} che estendono \lstinline{StatelessWidget} per costruire dei componenti in grado di permettere all'utente di effettuare una ricerca.

\subsubsection*{CustomSearchBar}
\label{subsubsec:custom-search-bar}

Classe che implementa una \lstinline{SearchBar}\cite{site:search-bar} in cui viene definito pressochè solo l'aspetto grafico in quanto attraverso il costruttore è obbligatorio passare un \lstinline{TextEditingController}\cite{site:text-editing-controller}, che si occupa di gestire il testo inserito dall'utente, e un metodo \lstinline{onChanged} che si occupa di gestire l'evento di cambiamento del testo inserito dall'utente, mentre è opzionale passare il colore del \emph{background} del componente. \\
Lo scopo è quello di fornire all'utente la possibilità di effettuare una ricerca all'interno della lista dei clienti (vedi requisito \hyperref[RFN-65]{RFN-65}).

\subsubsection*{CustomAutocomplete}
\label{subsubsec:custom-autocomplete}

Classe che implementa \lstinline{Autocomplete}\cite{site:autocomplete}, consente di fornire dei suggerimenti per l'autocompletamento nella ricerca, in cui viene anche qui solo l'aspetto grafico. \\
Attraverso il costruttore si rende necessario passare dei metodi fondamentali: \lstinline{optionsBuilder} che si occupa di fornire i suggerimenti per l'autocompletamento, \lstinline{displayStringForOption} che si occupa di fornire la stringa da visualizzare per ogni suggerimento e \lstinline{onSelected} che si occupa di gestire l'evento di selezione di un suggerimento.\\
Viene impiegato nel \emph{widget} \lstinline{FilterPanel} (vedi sezione \ref{subsec:filter-panel}) per fornire all'utente la possibilità di ricercare un cliente (vedi requisito \hyperref[RFN-16]{RFN-16}).

\subsection{Text Box}
\label{subsec:text-box}

Nel file denominato \lstinline{text_box.dart} è stato implementato un \emph{widget} che estende \lstinline{StatefulWidget} per costruire un componente in grado di permettere all'utente di inserire del testo.\\
È composto da semplicemente un \lstinline{TextField}\cite{site:text-field} in cui viene personalizzato nell'aspetto grafico, viene richiesto dal costruttore di passare l'altezza di questo componente, in modo da poterlo utilizzare in diverse situazioni, poi è necessatio passare un \lstinline{TextEditingController}\cite{site:text-editing-controller} e un metodo \lstinline{onChanged} che si occupa di gestire l'evento di cambiamento del testo inserito dall'utente.\\
Viene utilizzato per la creazione/modifica di una \emph{Meeting Note} (vedi requisito \hyperref[RFN-72]{RFN-72}) e per la creazione automatica (vedi requisito \hyperref[RFD-58]{RFD-58}).

\subsection{Toogle Buttons}
\label{subsec:toogle-buttons}

Nel file denominato \lstinline{toogle_buttons.dart} è stato implementato un \emph{widget} che estende \lstinline{StatefulWidget} per costruire un componente in grado di permettere all'utente di selezionare una delle due opzioni disponibili.\\
È composto da un \lstinline{ToggleButtons}\cite{site:toggle-buttons} di cui viene definito l'aspetto grafico e il comportamento.\\
Attraverso il costruttore è necessario passare un \lstinline{List<Widget>} contenente le due opzioni disponibili, un \lstinline{List<bool>} che indica quale opzione è stata selezionata e un metodo \lstinline{onChoiceSelected} che si occupa di gestire l'evento di selezione di un'opzione.\\
Viene impiegato nel \lstinline{FilterPanel} (vedi sezione \ref{subsec:filter-panel}) per fornire all'utente la possibilità di selezionare una delle due opzioni disponibili per filtrare la lista di \emph{Meeting Note} (vedi requisiti \hyperref[RFN-31]{RFN-31} e \hyperref[RFN-32]{RFN-32}).

\subsection{Warning Alert}
\label{subsec:warning-alert}

Nel file denominato \lstinline{warning_alert.dart} è stato implementato un \emph{widget} che estende \lstinline{StatelessWidget} per costruire un componente in grado di mostrare un messaggio di avvertimento all'utente, i casi in cui viene richiesto si faccia riferimento ai seguenti requisiti: \hyperref[RFN-12]{RFN-12}, \hyperref[RFN-13]{RFN-13}, \hyperref[RFN-20]{RFN-20}, \hyperref[RFN-30]{RFN-30}, \hyperref[RFN-34]{RFN-34}, \hyperref[RFN-64]{RFN-64}. \\
È composto da un \lstinline{Text}\cite{site:text} e un \lstinline{Icon}\cite{site:icon}, il loro scopo è quello di visualizzare il messaggio di avvertimento, attraverso i parametri del costruttore viene passato il messaggio e il colore che devono avere il testo e l'icona. 

\section{Constants}
\label{sec:constants}

In questa cartella sono stati definiti dei file che contengono delle costanti utilizzate all'interno dell'applicazione.\\


\subsection{App constants}
\label{subsec:app-constants}

Nel file denominato \lstinline{app_constants.dart} sono state definite delle costanti di tipo \lstinline{String} che rappresentano i nomi dei vari componenti dell'applicazione come messaggi di errore, nomi delle schermate, nomi dei campi di una \emph{Meeting Note} e così via.

\subsection{API constants}
\label{subsec:api-constants}

Nel file denominato \lstinline{api_constants.dart} sono state definite due classi che rappresentano le costanti utilizzate per effettuare le richieste all'\gls{apig}\glsoccur.\\
Nella prima classe \lstinline{ApiUrls} sono presenti le costanti che si riferiscono agli \gls{endpoint}\glsoccur dell'\gls{apig}\glsoccur.\\
Nella seconda \lstinline{ApiFields} sono presenti costanti che si riferiscono al nome dei parametri che devono essere passati all'\gls{urig}\glsoccur di alcune richieste all'\gls{apig}\glsoccur. \\
L'utilizzo di queste costanti avviene esclusivamente all'interno delle classi contenute nella cartella \emph{Services} (vedi sezione \ref{subsec:services}).\\

\section{Data}
\label{sec:data}

Questa cartella contiene due cartelle rappresentanti i \emph{data layer} e \emph{domain layer} dell'applicazione, rispettivamente

\subsection{Model}
\label{subsec:model}

Sono contenuti i file che rappresentano il modello dei dati utilizzato all'interno dell'applicazione.\\
Si specifica, per evitare rindondanze, che alcune classi in questione sono state implementate seguendo un \emph{pattern} specifico: oltre alla definizione degli attributi opportuni, ovvero quelli specificati, e che sono vincolanti, nelle \gls{apig}\glsoccur del sistema, sono presenti un costruttore \emph{factory}\cite{site:factory} \lstinline{fromJson}, come si intuisce si occupa di convertire da \gls{jsong}\glsoccur all'oggetto interessato, dei metodi che ritornano il valore di ciascun attributo e un metodo \lstinline{toString} che ritorna una stringa che rappresenta l'oggetto.\\
Nel corso di questa sezione verrà specificato quando una classe è stata implementata seguendo tale \emph{pattern}, mentre per le altre verrà descritta specificatamente l'implementazione. \\
Inoltre in quasi tutti i file è presente una classe, il cui nome termina con suffisso \lstinline{Response}, che estende \lstinline{BaseResponse} (vedi sezione \ref{subsubsec:status-response}) e rappresenta l'esito di una richiesta \gls{httpg}\glsoccur fatta all'\gls{apig}\glsoccur, nel caso in cui l'esito sia positivo contiene anche il valore da restituire.\\
Infine, per ogni classe è sottinteso la presenza del costruttore con tutti gli attributi presenti, che non viene riportato per evitare ridondanze. \\ 
Di seguito tali classi verranno discusse nel dettaglio.

\subsubsection*{Authentication}
\label{subsubsec:authentication}

In questo file sono state definite tutte le classi necessari per eseguire l'operazione di autenticazione all'applicazione, di seguito verrà descritta l'implementazione di ciascuna.

\paragraph*{AuthArgs} ~ \\
\label{par:auth-args}

\noindent Classe che incapsula i parametri necessari per effettuare la richiesta di autenticazione all'\gls{apig}\glsoccur, è composta da due attributi di tipo \lstinline{String} che rappresentano lo \emph{username} e la \emph{password} dell'utente.\\
Inoltre espone due metodi che si occupano di effettuare degli \gls{overriding}\glsoccur dei metodi \lstinline{operator ==}\cite{site:operator-equals} e \lstinline{hashCode}\cite{site:hascode-property} per poter effettuare il confronto tra due oggetti di tipo \lstinline{AuthArgs}.\\
Infine è presente un metodo \lstinline{toString} che ritorna una stringa che rappresenta l'oggetto.

\paragraph*{AuthValues} ~ \\
\label{par:auth-values}

\noindent Classe che incapsula il valore restituito dall'\gls{apig}\glsoccur in seguito alla richiesta di autenticazione, è composta da un attributo di tipo \lstinline{String} che rappresenta il \emph{token} di autenticazione.\\
La sua implementazione è stata effettuata seguendo il \emph{pattern} nell'introduzione di questa sezione.

\paragraph*{AuthResponse} ~ \\
\label{par:auth-response}

\noindent Classe che estende \lstinline{BaseResponse}, nella quale viene aggiunto un attributo di tipo \lstinline{AuthValues} che ne rappresenta il valore restituito.

\subsubsection*{Meeting Note}
\label{subsubsec:meeting-note}

In questo file sono state definite tutte le classi necessari per gestire i dati riguardanti le \emph{Meeting Note} all'interno dell'applicazione, di seguito verrà descritta l'implementazione di ciascuna.

\paragraph*{RawMeetingNote} ~ \\
\label{par:raw-meeting-note}

\noindent Classe che si occupa di incapsulare i dati di una \emph{Meeting Note} ricevuti da una chiamata \gls{httpg}\glsoccur all'\gls{apig}\glsoccur, però non sono ancora utilizzabili all'interno dell'applicazione, la quasi totalità degli attributi di tipo \lstinline{String} ed alcuni di essi dunque non sono del tipo opportuno per la loro elaborazione all'interno dell'applicazione.\\
Gli attributi di tipo \lstinline{String} in questione sono:
\begin{itemize}
    \item \lstinline{uuid}, utilizzato per l'identificazione di una \emph{Meeting Note};
    \item \lstinline{meetingDate}, la data dell'incontro;
    \item \lstinline{note}, il contenuto della \emph{Meeting Note};
    \item \lstinline{userCreator}, l'autore della \emph{Meeting Note}.
\end{itemize}
Inoltre sono presenti tre attributi opzionali di tipo \lstinline{MeetingNoteObject}, ciascuno rappresentante un \gls{cliente}\glsoccur delle tre categorie: tra i quali solo uno di loro viene restituito da una chiamata al \emph{backend} e dunque non è nullo, poichè una \emph{Meeting Note} può essere associata ad un solo \gls{cliente}\glsoccur.\\
Una considerazione necessaria da fare è per l'attributo \lstinline{meetingDate} che non è di tipo \lstinline{DateTime}\cite{site:date-time}, dunque non è possibile effettuare ad esempio un confronto tra due date, situazione analoga per l'attributo \lstinline{userCreator} che è di tipo \lstinline{String} e non \lstinline{User}.\\
Per questo motivo è stata definita la classe \lstinline{MeetingNote} (discussa di seguito) che si occupa di effettuare la conversione da \lstinline{RawMeetingNote} a \lstinline{MeetingNote} e permette quindi di operare più facilmente su questi dati. \\
La sua implementazione è stata effettuata seguendo il \emph{pattern} nell'introduzione di questa sezione, con un ulteriore metodo \lstinline{toJson} che si occupa di convertire l'oggetto in \gls{jsong}\glsoccur per poterlo inviare al \gls{backend}\glsoccur.

\paragraph*{PaginatedMeetingNote} ~ \\
\label{par:paginated-meeting-note}

\noindent Classe che si occupa di incapsulare una lista di \lstinline{RawMeetingNote} paginata, con l'ulteriore presenza di due attributi di tipo \lstinline{String} che rappresentano rispettivamente il \emph{link} alla pagina precedente e successiva.\\
La sua implementazione è stata effettuata seguendo il \emph{pattern} nell'introduzione di questa sezione.

\paragraph*{MeetingNote} ~ \\
\label{par:meeting-note}

\noindent Classe che si occupa di incapsulare i dati di una \emph{Meeting Note} pronta per essere utilizzata all'interno dell'applicazione ed è composta dai seguenti attributi:
\begin{itemize}
    \item \lstinline{uuid} di tipo \lstinline{String}, utilizzato per l'identificazione di una \emph{Meeting Note};
    \item \lstinline{meetingDate} di tipo \lstinline{DateTime}\cite{site:date-time}, la data dell'incontro;
    \item \lstinline{note} di tipo \lstinline{String}, il contenuto della \emph{Meeting Note};
    \item \lstinline{user} di tipo \lstinline{User}, l'autore della \emph{Meeting Note};
    \item \lstinline{object} di tipo \lstinline{MeetingNoteObject}, il \gls{cliente}\glsoccur associato alla \emph{Meeting Note}.
\end{itemize}
La sua implementazione, molto semplice, consiste nei metodi per ottenere i valori degli attributi e un metodo \lstinline{toString} che ritorna una stringa che rappresenta l'oggetto.

\paragraph*{MeetingNoteResponse} ~ \\
\label{par:meeting-note-response}

\noindent Classe che estende \lstinline{BaseResponse}, nella quale viene aggiunto un attributo di tipo \lstinline{List<MeetingNote>} che ne rappresenta il valore restituito.

\paragraph*{SmartCreationValue} ~ \\
\label{par:smart-creation-value}

\noindent Classe che si occupa di incapsulare i dati, risultati dell'elaborazione da parte di un algoritmo di \gls{ai}\glsoccur per la creazione automatica di una \emph{Meeting Note}, è composta dai seguenti attributi:
\begin{itemize}
    \item \lstinline{customer} di tipo \lstinline{String}, il nome del \gls{cliente}\glsoccur elaborato dall'algoritmo;
    \item \lstinline{date} di tipo \lstinline{DateTime}, la data dell'incontro elaborato dall'algoritmo;
    \item \lstinline{category} di tipo \lstinline{String}, la categoria del \gls{cliente}\glsoccur elaborato dall'algoritmo;
    \item \lstinline{note} di tipo \lstinline{String}, il contenuto della \emph{Meeting Note} elaborato dall'algoritmo.
\end{itemize}
La sua implementazione è stata effettuata seguendo il \emph{pattern} nell'introduzione di questa sezione, con un ulteriore metodo \lstinline{toJson} che si occupa di convertire l'oggetto in \gls{jsong}\glsoccur per poterlo inviare al \gls{backend}\glsoccur e un altro metodo \lstinline{getObjectCategory} che converte l'attributo \lstinline{category} in un oggetto di tipo \lstinline{ObjectCategory}.

\paragraph*{SmartCreationResponse} ~ \\
\label{par:smart-creation-response}

\noindent Classe che estende \lstinline{BaseResponse}, nella quale viene aggiunto un attributo di tipo \lstinline{SmartCreationValue} che ne rappresenta il valore restituito.

\subsubsection*{Meeting Note Object}
\label{subsubsec:meeting-note-object}

In questo file sono state definite tutte le classi necessarie per gestire i dati riguardanti i \glspl{cliente} all'interno dell'applicazione, di seguito verrà descritta l'implementazione di ciascuna.

\paragraph*{MeetingNoteObject} ~ \\
\label{par:meeting-note-object}

\noindent Classe che rappresenta il modello di un \gls{cliente} relativo ad una \emph{Meeting Note}, ed è composta dai seguenti attributi:
\begin{itemize}
    \item \lstinline{uuid}, di tipo \lstinline{String}, utilizzato per l'identificazione di un \gls{cliente};
    \item \lstinline{name}, di tipo \lstinline{String}, il nome del \gls{cliente};
    \item \lstinline{vatNumber}, di tipo \lstinline{String}, il numero di partita IVA del \gls{cliente} ed è presente solo per i \emph{contraenti};
    \item \lstinline{category}, di tipo \lstinline{ObjectCategory}, la categoria del \gls{cliente}.
\end{itemize}
La sua implementazione è stata effettuata seguendo il \emph{pattern} nell'introduzione di questa sezione.

\paragraph*{PaginatedObject} ~ \\
\label{par:paginated-object}

\noindent Dalle specifiche dell'\gls{apig}\glsoccur, solamente i clienti di tipo \emph{broker} e \emph{contraente} vengono resituiti in modo paginato, per questo motivo è stata definita questa classe che rappresenta un \emph{wrapper} di una lista di \lstinline{MeetingNoteObject} e di due altri attributi di tipo \lstinline{String} che rappresentano rispettivamente il \emph{link} alla pagina precedente e successiva.\\
La sua implementazione è stata effettuata seguendo il \emph{pattern} nell'introduzione di questa sezione.

\paragraph*{ObjectValue} ~ \\
\label{par:object-value}

\noindent É una classe \emph{sealed}\cite{site:sealed-class} in cui è stato definito un costruttore di \emph{default} e un metodo \lstinline{toString} che ritorna una stringa che rappresenta l'oggetto.\\

\paragraph*{ObjectListValue} ~ \\
\label{par:object-list-value}

\noindent Classe che estende \lstinline{ObjectValue} e rappresenta il valore restituito dall'\gls{apig}\glsoccur in seguito alla richiesta di ottenere la lista dei clienti, indipendentemente a quale categoria appartengono e se sono paginati o meno.\\
Ridefinisce il costruttore e il metodo \lstinline{toString}, mentre viene aggiunto un metodo per ottenere la lista dei clienti.

\paragraph*{SingleObjectValue} ~ \\
\label{par:single-object-value}

\noindent Classe che estende \lstinline{ObjectValue} e rappresenta il valore restituito dall'\gls{apig}\glsoccur in seguito alla richiesta di ottenere un singolo cliente, la sua implementazione è analoga a quella di \lstinline{ObjectListValue}.

\paragraph*{MeetingNoteObjectResponse} ~ \\
\label{par:meeting-note-object-response}

\noindent Classe che estende \lstinline{BaseResponse}, nella quale viene aggiunto un attributo di tipo \lstinline{ObjectValue} che ne rappresenta il valore restituito.

\subsubsection*{Object Category}
\label{subsubsec:object-category}

Si tratta semplicemente di una enumerazione che rappresenta le tre categorie dei \glspl{cliente}\glsoccur:
\begin{itemize}
    \item \lstinline{ObjectCategory.broker}: rappresenta la categoria dei broker;
    \item \lstinline{ObjectCategory.contractor}: rappresenta la categoria dei contraenti;
    \item \lstinline{ObjectCategory.delegatedInsurer}: rappresenta la categoria delle compagnie assicurative.
\end{itemize}

\subsubsection*{Status Response}
\label{subsubsec:status-response}

La classe astratta \lstinline{BaseResponse} modella l'esito di una qualsiasi richiesta \gls{httpg}\glsoccur fatta all'\gls{apig}\glsoccur. \\
Ha un attributo di tipo \lstinline{StatusResponse}, una enumerazione che rappresenta tre possibili stati di una richiesta \gls{httpg}\glsoccur:
\begin{itemize}
    \item \lstinline{StatusResponse.success}: la richiesta è andata a buon fine;
    \item \lstinline{StatusResponse.unauthorizedException}: la richiesta non è andata a buon fine a causa del \emph{token} di autenticazione scaduto;
    \item \lstinline{StatusResponse.genericException}: la richiesta non è andata a buon fine per un errore generico.
\end{itemize}
Inoltre espone due metodi, il primo si occupa di restituire il valore dell'attributo \lstinline{StatusResponse} e il secondo di restituire una stringa che rappresenta l'oggetto.

\subsubsection*{User}
\label{subsubsec:user}

In questo file sono state definite tutte le classi necessari per gestire l'utente all'interno dell'applicazione, di seguito verrà descritta l'implementazione di ciascuna.

\paragraph*{User} ~ \\
\label{par:user}

\noindent Classe che modella l'utente all'interno dell'applicazione, gli attributi che la compongono sono: \emph{email}, \emph{firstName}, \emph{lastName}, di tipo \lstinline{String}, e \emph{profile} di tipo \lstinline{Profile}. \\
La sua implementazione è stata effettuata seguendo il \emph{pattern} specificato nell'introduzione di questa sezione.

\paragraph*{Profile} ~ \\
\label{par:profile}

\noindent Classe che modella il profilo dell'utente all'interno dell'applicazione, gli attributi che la compongono sono: \emph{uuid}, un identificativo univoco assegnato ad ogni utente, e l'\emph{avatar}, contiene il link dell'immagine profilo, entrambi di tipo \lstinline{String}. \\
La sua implementazione è stata effettuata seguendo il \emph{pattern} specificato nell'introduzione di questa sezione.

\paragraph*{UserResponse} ~ \\
\label{par:user-response}

\noindent Classe che estende \lstinline{BaseResponse}, nella quale viene aggiunto un attributo di tipo \lstinline{User} che ne rappresenta il valore restituito.

\subsection{Services}
\label{subsec:services}

In questa sezione verranno discusse le classi che si occupano di effettuare le richieste \gls{httpg}\glsoccur al \gls{backend}\glsoccur attraverso le \gls{apig}\glsoccur, per evitare rindondanze, ogni servizio, eccetto \lstinline{BaseService}, gestisce le eventuali eccezioni (RIMANDA A SEZIONE SPECIFICA) in caso una chiamata non vada a buon fine.

\subsubsection*{Authentication Service}
\label{subsubsec:authentication-service}

In questo file è definita una classe che estende \lstinline{BaseService} (classe discussa di seguito) con la composizione di \lstinline{PostRequest} per definire il metodo \lstinline{getToken} che si occupa di effettuare una richiesta \lstinline{POST} al \gls{backend}\glsoccur. \\
Riceve in input un oggetto di tipo \lstinline{AuthArgs} (vedi sezione \ref{subsec:model}) che incapsula lo \emph{username} e la \emph{password} dell'utente per poi restituire eventualmente un oggetto di tipo \lstinline{AuthResponse}.

\subsubsection*{Base Service}
\label{subsubsec:base-service}

In questo file sono state definite tutte le classi necessarie per fornire un'astrazione per le chiamate \gls{httpg}\glsoccur all'\gls{apig}\glsoccur, di seguito verrà descritta l'implementazione di ciascuna.

\paragraph*{BaseService} ~ \\
\label{par:base-service}

\noindent Classe astratta che si occupa di fornire un'interfaccia per gestire le chiamate \gls{httpg}\glsoccur e provede a definire i metodi fondamentali per effettuare una richiesta:
\begin{itemize}
    \item \lstinline{getUrl}, che si occupa di restituire l'\gls{urig}\glsoccur della risorsa data una stringa in input;
    \item \lstinline{handleResponse}, invocata per ogni tipologia di richiesta, che si occupa di determinare se una richiesta è avvenuta con successo o meno, in caso di quest'ultimo viene lanciata un'eccezione (RIMANDA A SEZIONE SPECIFICA).
\end{itemize}
Le seguenti classi sono di tipo \lstinline{mixin}\cite{site:mixins} e aggiungono funzionalità alla classe \lstinline{BaseService}: ogni classe \lstinline{mixin} si occupa di fornire un implementazione per una chiamata \gls{httpg}\glsoccur specifica. \\
Successivamente, quando verranno utilizzate, la classe in questione dovrà estendere \lstinline{BaseService} e includere alcune di queste classi \lstinline{mixin}, in modo da aggiungere ad un servizio le chiamate necessarie.\\
Di seguito verranno descritte le classi \lstinline{mixin}, per ciascuna si specificheranno i parametri necessari per la configurazione dell'\emph{header} o del \emph{body} della richiesta, a seconda delle necessità. \\

\paragraph*{GetRequest} ~ \\
\label{par:get-request}

\noindent Classe \lstinline{mixin} che si occupa di fornire un'implementazione per effettuare una richiesta \lstinline{GET}, i parametri necessari sono l'\emph{url} della risorsa e il \emph{token} di autenticazione.

\paragraph*{PatchRequest} ~ \\
\label{par:patch-request}

\noindent Classe \lstinline{mixin} che si occupa di fornire un'implementazione per effettuare una richiesta \lstinline{PATCH}, i parametri necessari sono l'\emph{url} della risorsa, il \emph{token} di autenticazione e il \emph{body} della richiesta.

\paragraph*{DeleteRequest} ~ \\
\label{par:delete-request}

\noindent Classe \lstinline{mixin} che si occupa di fornire un'implementazione per effettuare una richiesta \lstinline{DELETE}, i parametri necessari sono l'\emph{url} della risorsa e il \emph{token} di autenticazione.

\paragraph*{PostRequest} ~ \\
\label{par:post-request}

\noindent Classe \lstinline{mixin} che si occupa di fornire un'implementazione per effettuare una richiesta \lstinline{POST}. \\
Vi sono presenti due metodi, il primo si occupa di effettuare una richiesta \lstinline{POST} con lo scopo di salvare dei dati sul \emph{backend}, i parametri necessari sono l'\emph{url} della risorsa, il \emph{token} di autenticazione e il \emph{body} della richiesta. \\
Il secondo si occupa di effettuare una richiesta \lstinline{POST} con lo scopo di ottenere il \emph{token} di autenticazione, l'unico parametro necessario è di tipo \lstinline{AuthArgs} (vedi precedentemente).

\subsubsection*{Meeting Note Object Service}
\label{subsubsec:meeting-note-object-service}

In questo file è definita una classe che estende \lstinline{BaseService} (classe discussa in precedenza) con la composizione di \lstinline{GetRequest} per definire i seguenti metodi:
\begin{itemize}
    \item \lstinline{getMeetingNoteObject}, ritorna \lstinline{MeetingNoteObjectResponse} (vedi sezione \ref{subsec:model}) in seguito alla richiesta di ottenere un singolo \gls{cliente}\glsoccur per la creazione di una \emph{Meeting Note}, per fare ciò è necessario passare il \emph{token} di autenticazione, l'\emph{url} della risorsa e la categoria del cliente;
    \item \lstinline{getContractors}, ritorna \lstinline{MeetingNoteObjectResponse} (vedi sezione \ref{subsec:model}) in seguito alla richiesta di ottenere la lista paginata dei \glspl{cliente}\glsoccur di tipo \emph{contraente}, per fare ciò è necessario passare obbligatoriamente il \emph{token} di autenticazione, \lstinline{pageSize} e \lstinline{pageNumber} che rappresentano rispettivamente la dimensione della pagina e il numero della pagina, opzionalmente è possibile passare anche il nome del \gls{cliente} da ricercare;
    \item \lstinline{getBrokers}, implementazione analoga a \lstinline{getContractors} ma per i \glspl{cliente}\glsoccur di tipo \emph{broker};
    \item \lstinline{getDelegatedInsurers}, ritorna \lstinline{MeetingNoteObjectResponse} (vedi sezione \ref{subsec:model}) in seguito alla richiesta di ottenere la lista non paginata dei \glspl{cliente}\glsoccur di una \emph{compagnia assicurativa}, per fare ciò è necessario passare il \emph{token} di autenticazione ed opzionalmente il nome del \gls{cliente} da ricercare;
    \item \lstinline{buildUrl}, si occupa di costruire l'\gls{urig}\glsoccur della risorsa in base ai parametri passati, che sono: l'\emph{url} della risorsa, la dimensione della pagina, il numero della pagina e il nome del \gls{cliente} da ricercare, esso viene utilizzato dai metodi \lstinline{getContractors} e \lstinline{getBrokers}.
\end{itemize}

\subsubsection*{Meeting Note Service}
\label{subsubsec:meeting-note-service}

In questo file è definita una classe che estende \lstinline{BaseService} (classe discussa in precedenza) con la composizione di \lstinline{GetRequest}, \lstinline{PostRequest}, \lstinline{PatchRequest} e \lstinline{DeleteRequest} per definire i seguenti metodi:
\begin{itemize}
    \item \lstinline{getMeetingNotes}, ritorna \lstinline{MeetingNoteResponse} (vedi sezione \ref{subsec:model}) in seguito alla richiesta di ottenere la lista paginata delle \emph{Meeting Note}, per fare ciò è necessario passare obbligatoriamente il \emph{token} di autenticazione, \lstinline{pageSize} e \lstinline{pageNumber} che rappresentano rispettivamente la dimensione della pagina e il numero della pagina, opzionalmente è possibile passare anche un oggetto \lstinline{FilteringOptions} (RIMANDARE A SEZIONE SPECIFICA) che rappresenta le opzioni di filtraggio. \\
    Nel dettaglio, ad ogni chiamata riceve la lista di \lstinline{RawMeetingNote} e la converte in una lista di \lstinline{MeetingNote} (vedi sezione \ref{subsec:model});
    \item \lstinline{postMeetingNote}, ritorna \lstinline{MeetingNoteResponse} (vedi sezione \ref{subsec:model}) in seguito alla richiesta di creare una \emph{Meeting Note}, per fare ciò è necessario passare il \emph{token} di autenticazione e un oggetto di tipo \lstinline{MeetingNote} (vedi sezione \ref{subsec:model}), convertendo quest'ultimo in \gls{jsong}\glsoccur attraverso un metodo dedicato;
    \item \lstinline{patchMeetingNote}, ritorna \lstinline{MeetingNoteResponse} (vedi sezione \ref{subsec:model}) in seguito alla richiesta di modificare una \emph{Meeting Note}, per fare ciò è necessario passare il \emph{token} di autenticazione e un oggetto di tipo \lstinline{MeetingNote} (vedi sezione \ref{subsec:model}), convertendo quest'ultimo in \gls{jsong}\glsoccur attraverso un metodo dedicato;
    \item \lstinline{deleteMeetingNote}, ritorna \lstinline{MeetingNoteResponse} (vedi sezione \ref{subsec:model}) in seguito alla richiesta di eliminare una \emph{Meeting Note}, per fare ciò è necessario passare il \emph{token} di autenticazione e l'\emph{uuid} della \emph{Meeting Note};
    \item \lstinline{postSmartCreationMeetingNote}, ritorna \lstinline{SmartCreationResponse} (vedi sezione \ref{subsec:model}) in seguito alla richiesta di creare una \emph{Meeting Note} in modo automatico, per fare ciò è necessario passare il \emph{token} di autenticazione e il testo da elaborare;;
    \item \lstinline{buildUrl}, si occupa di costruire l'\gls{urig}\glsoccur della risorsa in base ai parametri passati, che sono: la dimensione della pagina, il numero della pagina ed eventualmente le opzioni di filtraggio, esso viene utilizzato dal metodo \lstinline{getMeetingNotes}.
\end{itemize}

\subsubsection*{User Service}
\label{subsubsec:user-service}

In questo file è definita una classe che estende \lstinline{BaseService} (classe discussa in precedenza) con la composizione di \lstinline{GetRequest} per definire il metodo \lstinline{getUser} che si occupa di effettuare una richiesta \lstinline{GET} al \gls{backend}\glsoccur per ottenere i dati dell'utente, ricevendo in input il \emph{token} di autenticazione e restitornando un oggetto di tipo \lstinline{UserResponse} (vedi sezione \ref{subsec:model}).

\section{Provider}
\label{sec:provider}

In questa cartella sono presenti le classi che si occupano di gestire la \emph{logica di business} dell'applicazione, si tratta dunque dell'\emph{application layer} (vedi sezione \ref{subsec:architettura-app}).\\
In ciascun file viene definita una classe al cui interno sono stati implementati dei metodi opportuni per effettuare delle operazioni e al di fuori di essa si istanzia un \lstinline{FutureProvider}\cite{site:future-provider} (discusso nella sezione \ref{subsec:riverpod}) per esporre questi metodi al \emph{presentation layer}.\\
Inoltre ciascuna classe per effettuare le operazioni necessarie si avvale dei corrispettivi servizi (vedi sezione \ref{subsec:services}) e modelli (vedi sezione \ref{subsec:model}) precedentemente discussi. \\
Si vuole infine precisare che, eccetto per \lstinline{AuthenticationHelper} (discussa nella sezione \ref{subsec:authentication-provider}), nelle restanti classi, ogni metodo che effettua una richiesta include l'operazione di lettura del \emph{token} per autenticare quest'ultime.

\subsection{Authentication Provider}
\label{subsec:authentication-provider}

In questo file è definita la classe \lstinline{AuthenticationHelper} che si occupa di gestire la logica dell'autenticazione all'applicazione, in particolare espone i seguenti metodi:
\begin{itemize}
    \item \lstinline{verifyCredentials}, si occupa di verificare la validità delle credenziali, passate per parametro, inserite dall'utente;
    \item \lstinline{authenticate}, verifica le credenziali dell'utente, avvalendosi del metodo \lstinline{verifyCredentials}, e in caso di successo ritorna un \emph{token} di autenticazione attraverso l'oggetto \lstinline{AuthResponse} (vedi sezione \ref{subsec:model}) salvandolo poi in locale con l'ausilio della libreria \lstinline{shared_preferences}\cite{site:shared-preferences} (discussa nella sezione \ref{sec:utils});
    \item \lstinline{logout}, si occupa di effettuare il \emph{logout} dall'applicazione, eliminando il \emph{token} di autenticazione salvato in locale e riportando l'applicazione alla schermata di \emph{login};
    \item \lstinline{isAuthenticated}, verifica se l'utente è autenticato, controllando se il \emph{token} di autenticazione è salvato in locale;
    \item \lstinline{isUnauthorized}, verifica se l'utente è autorizzato ad effettuare una chiamata \gls{httpg}\glsoccur, controllandone l'esito e in caso negativo effettua il \emph{logout} dall'applicazione, poichè significa che il \emph{token} di autenticazione è scaduto;
    \item \lstinline{isBiometricEnabled}, verifica se l'utente ha abilitato l'autenticazione biometrica all'interno dell'applicazione, controllando se il \emph{flag} è salvato in locale;
    \item \lstinline{readCredentials}, si occupa di leggere le credenziali cifrate salvate in locale, attraverso la libreria \lstinline{FlutterSecureStorage}\cite{site:flutter-secure-storage}, restituendo un oggetto di tipo \lstinline{AuthArgs} (vedi sezione \ref{subsec:model});
    \item \lstinline{saveCredentials}, si occupa di salvare le credenziali cifrate in locale, ricevendo in input un oggetto di tipo \lstinline{AuthArgs} (vedi sezione \ref{subsec:model});
    \item \lstinline{removeCredentials}, si occupa di eliminare le credenziali cifrate salvate in locale;
\end{itemize}

\subsection{Meeting Note Object Provider}
\label{subsec:meeting-note-object-provider}

In questo file è definita la classe \lstinline{MeetingNoteObjectHelper} che si occupa di gestire la logica dei \glspl{cliente}\glsoccur all'interno dell'applicazione, in particolare espone i seguenti metodi:
\begin{itemize}
    \item \lstinline{getContractors} si occupa di ottenere la lista paginata di tutti i \glspl{cliente}\glsoccur di tipo \emph{contraente} oppure solo di quelli ricercati dall'utente attraverso il passaggio del parametro \lstinline{searchTerm}, dati in input il numero della pagina e la dimensione della pagina, restituendo un oggetto di tipo \lstinline{MeetingNoteObjectResponse} (vedi sezione \ref{subsec:model});
    \item \lstinline{getBrokers} implementazione analoga a \lstinline{getContractors} ma per i \glspl{cliente}\glsoccur di tipo \emph{broker};
    \item \lstinline{getDelegatedInsurers} implementazione analoga ai due metodi precedenti ma per i \glspl{cliente}\glsoccur di tipo \emph{compagnia assicurativa} e senza la possibilità di paginazione.
\end{itemize}

\subsection{Meeting Note Provider}
\label{subsec:meeting-note-provider}

In questo file è definita la classe \lstinline{MeetingNoteHelper} che si occupa di gestire la logica delle \emph{Meeting Note} all'interno dell'applicazione, in particolare espone i seguenti metodi:
\begin{itemize}
    \item \lstinline{fecthMeetingNotes}, si occupa di ottenere la lista paginata delle \emph{Meeting Note} oppure solo di quelle filtrate dall'utente attraverso il passaggio del parametro \lstinline{filteringOptions} (RIMANDARE ALLA SEZIONE SPECIFICA), dati in input il numero della pagina e la dimensione della pagina, restituendo un oggetto di tipo \lstinline{MeetingNoteResponse} (vedi sezione \ref{subsec:model});
    \item \lstinline{createMeetingNote}, si occupa di memorizzare la \emph{Meeting Note} creata e passata in input come oggetto di tipo \lstinline{MeetingNote} nel \gls{backend}\glsoccur, restituendo un oggetto di tipo \lstinline{StatusResponse} (vedi sezione \ref{subsec:model}) in quanto è sufficiente ritornare l'esito della richiesta;
    \item \lstinline{deleteMeetingNote}, si occupa di eliminare la \emph{Meeting Note} passata in input lo \lstinline{uuid} come parametro, restituendo un oggetto di tipo \lstinline{StatusResponse} (vedi sezione \ref{subsec:model}) in quanto è sufficiente ritornare l'esito della richiesta;
    \item \lstinline{modifyMeetingNote}, si occupa di modificare la \emph{Meeting Note} passata in input come oggetto di tipo \lstinline{MeetingNote}, restituendo un oggetto di tipo \lstinline{StatusResponse} (vedi sezione \ref{subsec:model}) in quanto è sufficiente ritornare l'esito della richiesta;
    \item \lstinline{smartCreation}, si occupa di creare una \emph{Meeting Note} in modo automatico, passando in input il testo da elaborare, restituendo un oggetto di tipo \lstinline{SmartCreationResponse} (vedi sezione \ref{subsec:model}).
\end{itemize}

\subsection{User Provider}
\label{subsec:user-provider}

In questo file è definita la classe \lstinline{UserHelper} che si occupa di gestire la logica dell'utente all'interno dell'applicazione, in particolare espone il metodo \lstinline{initUser} che si occupa di ottenere i dati dell'utente, restituendo un oggetto di tipo \lstinline{UserResponse} (vedi sezione \ref{subsec:model}).

\section{Screens}
\label{sec:screens}

In questa cartella sono presenti tutte le schermate dell'applicazione, le quali verranno descritte di seguito.

\subsection{Account Screen}
\label{subsec:account-screen}

Schermata che permette all'utente di visualizzare i propri dati, letti attraverso il \lstinline{UserProvider} (vedi sezione \ref{subsec:user-provider}), si ricorda che per poter effettuare una lettura dei \emph{provider} è necessario che la classe estenda \lstinline{ConsumerWidget} (discussa nella sezione \ref{subsec:riverpod}). \\
Per la definizione dell'aspetto grafico questa schermata implementa \lstinline{BaseScreen} (vedi sezione \ref{subsec:screens-template}) e prima di renderizzare ulteriori componenti effettua un controllo sullo stato della connessione ad internet, ottenuto attraverso il \lstinline{NetworkAwareProvider} (RIMANDARE ALLA SEZIONE SPECIFICA), in caso di assenza di connessione viene mostrato un \lstinline{WarningAlert} (vedi sezione \ref{subsec:warning-alert}) con il relativo messaggio di errore. \\
Altrimenti viene mostrato il contenuto della schermata servendosi di \lstinline{FutureBuilder}\cite{site:future-builder} per invocare il metodo \lstinline{initUser} del \lstinline{UserProvider} e ottenere i dati dell'utente, in particolare ci possono essere vari esiti:
\begin{itemize}
    \item \emph{waiting}, viene mostrato un \lstinline{CircularProgressIndicator}\cite{site:circular-progress-indicator} per indicare che la richiesta è in corso;
    \item \emph{unauthorizedException}, l'utente non è autorizzato ad effettuare la richiesta, viene reindirizzato alla schermata di \emph{login};
    \item \emph{genericException}, la richiesta non è andata a buon fine per un errore generico e viene mostrato un \lstinline{WarningAlert} con il relativo messaggio di errore;
    \item \emph{success}, la richiesta è andata a buon fine e viene mostrato il contenuto della schermata.
\end{itemize}
In quest'ultimo caso vengono renderizzati i seguenti componenti:
\begin{itemize}
    \item \lstinline{TextButton}\cite{site:text-button}, per effettuare il \emph{logout} dall'applicazione, che mostra un \lstinline{WarningAlertDialog} (vedi sezione \ref{subsec:warning-alert}) che informa l'utente che ne deve confermare o meno l'operazione;
    \item \lstinline{CircleAvatar}\cite{site:circle-avatar}, per mostrare l'immagine profilo dell'utente;
    \item \lstinline{AccountCard}, \emph{StatelessWidget} privato che si occupa di mostrare il nome, cognome e la mail dell'utente;
    \item \lstinline{BiometricOption}, \emph{StatefulWidget} privato che implementa \lstinline{BiometricSwitch} (vedi sezione \ref{subsec:biometric-switch}), si ricorda che lo stato di questo componente è salvato in locale attraverso la libreria \lstinline{shared_preferences}\cite{site:shared-preferences} (discussa nella sezione \ref{sec:utils}); per la sua attivazione verrà richiesto all'utente di confermare l'operazione attraverso l'inserimento delle credenziali di autenticazione, anch'esse salvate in locale attraverso la libreria \lstinline{FlutterSecureStorage}\cite{site:flutter-secure-storage} (discussa nella sezione \ref{sec:utils}).
\end{itemize}

\subsection{Home Screen}
\label{subsec:home-screen}

Schermata principale dell'applicazione, si ricorda che per poter effettuare una lettura dei \emph{provider} è necessario che la classe estenda \lstinline{ConsumerStatefulWidget} (discussa nella sezione \ref{subsec:riverpod}). \\
Prima della renderizzazione dell'aspetto grafico viene effettuata l'inizializzazione delle variabili di stato, in particolare:
\begin{itemize}
    \item \lstinline{filteringOptions}, oggetto di tipo \lstinline{FilteringOptions} (RIMANDARE ALLA SEZIONE SPECIFICA) che rappresenta le opzioni di filtraggio attive;
    \item \lstinline{isFiltering}, \lstinline{bool} che indica se l'utente ha attivato o meno almeno un filtro;
    \item \lstinline{pagingController}, oggetto di tipo \lstinline{PagingController}\cite{site:infinite-scroll-pagination} che si occupa di gestire la paginazione dei dati.
\end{itemize}
Per la definizione dell'aspetto grafico questa schermata implementa \lstinline{BaseScreen} (vedi sezione \ref{subsec:screens-template}) e il suo contenuto è strutturato come segue:
\begin{itemize}
    \item \lstinline{HomeHeader}, \emph{StatelessWidget} privato che mostra il titolo della schermata e due pulsanti, uno per eliminare i filtri attivati e l'altro per aprire il \lstinline{FilterPanel} (vedi sezione \ref{subsec:filter-panel});
    \item \lstinline{SpeedDial}, componente che mostra un \lstinline{FloatingActionButton}\cite{site:fab} che consente all'utente di scegliere la modalità di creazione di una \emph{Meeting Note}, in base alla scelta l'utente verrà reindirizzato alla schermata corrispondente;
    \item \lstinline{PagedListView}\cite{site:infinite-scroll-pagination}, componente che si occupa di mostrare la lista paginata delle \emph{Meeting Note}, in particolare si avvale del \lstinline{PagingController} per gestire la paginazione dei dati, inoltre per ogni elemento della lista viene mostrato un \lstinline{MeetingNoteItem} (vedi sezione \ref{subsec:meeting-note-card}) che rappresenta la \emph{Meeting Note} e permette all'utente di modificarla o eliminarla.
\end{itemize}
Nel dettaglio, prima della visualizzazione della lista paginata, si effettua un controllo dello stato della connessione ad internet attraverso il \lstinline{NetworkAwareProvider} (RIMANDARE ALLA SEZIONE SPECIFICA), in caso di assenza di connessione viene mostrato un \lstinline{WarningAlert} (vedi sezione \ref{subsec:warning-alert}) con il relativo messaggio di errore. \\
Altrimenti viene mostrato il contenuto della schermata servendosi del metodo privato \lstinline{fetchPage} che si occupa di effettuare la richiesta al \gls{backend}\glsoccur attraverso il \lstinline{MeetingNoteProvider} (vedi sezione \ref{subsec:meeting-note-provider}), controllando se l'utente è autorizzato ad effettuare la richiesta attraverso il metodo \lstinline{isUnauthorized} fornito da \lstinline{authProvider} (vedi sezione \ref{subsec:authentication-provider}), per poi restituire la lista paginata delle \emph{Meeting Note} e aggiornare lo stato del \lstinline{PagingController}. \\
È presente un ulteriore metodo privato \lstinline{showWarningDialog} che si occupa di mostrare un \lstinline{WarningAlertDialog} (vedi sezione \ref{subsec:warning-alert}) chiedendo conferma l'utente per la eliminazione di una \emph{Meeting Note} e in base alla scelta dell'utente e all'esito della richiesta, vengono gestiti i seguenti casi {i medesimi illustrati nella sezione \ref{subsec:model}, vedi paragrafo \emph{StatusResponse}}:
\begin{itemize}
    \item \emph{success}, viene mostrato un \lstinline{SuccessAlertDialog} (vedi sezione \ref{subsec:success-alert}) con il relativo messaggio di successo e di conseguenza viene aggiornato lo stato del \lstinline{PagingController};
    \item \emph{unauthorizedException}, l'utente non è autorizzato ad effettuare la richiesta, viene reindirizzato alla schermata di \emph{login};
    \item \emph{genericException}, la richiesta non è andata a buon fine per un errore generico e viene mostrato un \lstinline{WarningAlert} con il relativo messaggio di errore.
\end{itemize}

\subsection{Login Screen}
\label{subsec:login-screen}

Schermata che permette all'utente di effettuare l'autenticazione all'applicazione. \\
Definisce semplicemente l'aspetto grafico della schermata: implementando \lstinline{BaseScreen} (vedi sezione \ref{subsec:screens-template}), impostando \lstinline{LoginAppBar} (vedi sezione \ref{subsec:app-bars}) e nel \lstinline{body} viene definito il titolo e richiamato il \lstinline{LoginForm} (vedi sezione \ref{subsec:login-form}), che si occupa di gestire il comportamento e per questo motivo tale classe viene estesa con \emph{StatelessWidget}.

\subsection{Smart Creation Screen}
\label{subsec:smart-creation-screen}

Schermata che permette all'utente di creare una \emph{Meeting Note} in modo automatico e data la sua complessità, essa è stata suddivisa in tre classi, di seguito verranno descritte nel dettaglio. \\

\subsubsection*{SmartCreationPage}
\label{subsubsec:smart-creation-page}

Classe che si occupa di ricevere in input il contenuto testuale dall'utente che verrà elaborato da un algoritmo di \emph{intelligenza artificiale} per creare una \emph{Meeting Note} in modo automatico. \\
Prima della renderizzazione dell'aspetto grafico viene effettuata l'inizializzazione delle variabili di stato, in particolare:
\begin{itemize}
    \item \lstinline{textController}, \lstinline{TextEditingController} che si occupa di gestire il contenuto testuale inserito dall'utente;
    \item \lstinline{isButtonEnabled}, \lstinline{bool} che indica se l'utente ha inserito del testo oppure no e abilita di conseguenza il pulsante che avvia l'elaborazione del testo;
    \item \lstinline{speechToText}, \lstinline{SpeechToText} che si occupa di gestire la dettatura vocale;
    \item \lstinline{listenedWords}, \lstinline{String} che memorizza quanto detto dall'utente.
\end{itemize}
Per la definizione dell'aspetto grafico questa schermata implementa \lstinline{WizardScreen} (vedi sezione \ref{subsec:screens-template}) e il suo contenuto è strutturato come segue:
\begin{itemize}
    \item \lstinline{WizardHeader}, imposta il titolo della schermata (vedi sezione \ref{subsec:screens-template});
    \item \lstinline{CustomTextBox}, componente che si occupa di ricevere in input il contenuto testuale inserito dall'utente, che può avvenire in due modalità differenti come specificato nei requisiti \hyperref[RFN-72]{RFN-72} e \hyperref[RFN-73]{RFN-73} (vedi sezione \ref{subsec:text-box});
    \item \lstinline{RecordingButton}, componente che si occupa attivare la dettatura vocale (vedi sezione \ref{subsec:recording-button});
    \item \lstinline{ElevatedButton}\cite{site:elevated-button}, componente che si occupa di avviare l'elaborazione del testo inserito dall'utente.
\end{itemize}
Nel dettaglio quest'ultima operazione viene effettuata servendosi di \lstinline{meetingNoteProvider} per effettuare l'elaborazione e di \lstinline{authProvider} per autorizzare la richiesta attraverso il metodo \lstinline{isUnauthorized}, inoltre viene mostrato un \lstinline{CircularProgressIndicator}\cite{site:circular-progress-indicator} per indicare che la richiesta è in corso. \\
Al termine dell'elaborazione, con i tre possibili esiti (i medesimi illustrati nella sezione \ref{subsec:model}, vedi sezione \emph{Status Response}), e in caso di successo viene mostrato \lstinline{ConfirmCreationPopup} attraverso una modale che appare dal basso.

\subsubsection*{ConfirmCreationPopup}
\label{subsubsec:confirm-creation-popup}

Classe che si occupa di mostrare all'utente i dati estratti dall'elaborazione del testo, passati input dal costruttore dall'oggetto di tipo \lstinline{SmartCreationValue} (vedi sezione \ref*{subsec:model}) e di permettergli di confermare la creazione della \emph{Meeting Note} o di eventualmente apportare delle modifiche ad alcuni campi. \\
Prima della renderizzazione dell'aspetto grafico viene effettuata l'inizializzazione delle variabili di stato, in particolare:
\begin{itemize}
    \item \lstinline{selectedCategory}, \lstinline{ObjectCategory} che rappresenta la categoria del \gls{cliente}\glsoccur estratto dall'elaborazione del testo;
    \item \lstinline{selectedCategoryIndex}, \lstinline{int} che rappresenta l'indice della categoria del \gls{cliente}\glsoccur;
    \item \lstinline{initialValue}, \lstinline{String} che viene inzializzato con il nome del \gls{cliente}\glsoccur estratto dall'elaborazione del testo;
    \item \lstinline{debouncedSearch} (RIMANDA ALL SEZIONE SPECIFICA), si occupa di gestire la ricerca del \gls{cliente}\glsoccur attraverso il componente \lstinline{CustomAutocomplete};
    \item \lstinline{selectedMeetingNoteObject}, \lstinline{MeetingNoteObject} che si occupa di gestire eventuali modifiche sulla ricerca dell'identificativo del cliente apportate dall'utente ;
    \item \lstinline{isButtonEnabled}, \lstinline{bool} che abilita il pulsante di creazione se tutti i campi sono correttamente compilati;
    \item \lstinline{isNameFound}, \lstinline{bool} che indica se l'identificativo del \gls{cliente}\glsoccur, estratto dall'algoritmo sia stato trovato o meno;
    \item \lstinline{selectedDate}, \lstinline{DateTime} che rappresenta la data estratta dall'elaborazione del testo;
    \item \lstinline{textController}, \lstinline{TextEditingController} che si occupa di gestire il contenuto testuale inserito dall'utente.;
\end{itemize}
La struttura di questa modale è la seguente:
\begin{itemize}
    \item \lstinline{TitleLabel}, mostra il titolo della \emph{label} per ogni campo elencato di seguito;
    \item \lstinline{CustomObjectPicker}, componente che si occupa di mostrare la lista delle categorie dei \glspl{cliente}\glsoccur e di permettere eventualmente all'utente di apportare modifiche, il cui funzionamento è supportato dalle variabili di stato \lstinline{selectedCategory} e \lstinline{selectedCategoryIndex} (vedi sezione \ref{subsec:object-picker});
    \item \lstinline{CustomAutoComplete}, componente che visualizza l'identificativo del cliente e permette all'utente di apportare modifiche effettuando una ricerca, il suo funzionamento è supportato dalle variabili di stato \lstinline{initialValue}, \lstinline{debouncedSearch} e \lstinline{selectedMeetingNoteObject} (vedi sezione \ref{subsec:autocomplete}), imposta inoltre il valore di \lstinline{isNameFound} in base all'esito della ricerca;
    \item \lstinline{CustomDatePicker}, componente che visualizza la data estratta dall'elaborazione del testo e permette all'utente di apportare modifiche, il suo funzionamento è supportato dalla variabile di stato \lstinline{selectedDate} (vedi sezione \ref{subsec:date-picker});
    \item \lstinline{CustomTextBox}, componente che visualizza il contenuto dell'incontro estratto dall'elaborazione del testo e permette all'utente di apportare modifiche, il suo funzionamento è supportato dalla variabile di stato \lstinline{textController} (vedi sezione \ref{subsec:text-box});
    \item \lstinline{OutlineButton}\cite{site:outline-button}, componente che si occupa di confermare la creazione della \emph{Meeting Note} richiamando il metodo \lstinline{onConfirmation}.
\end{itemize}
Vi sono presenti inoltre due metodi privati, \lstinline{search} e \lstinline{onConfirmation}. \\
Il primo si occupa di effettuare la ricerca dell'identificativo del \gls{cliente}\glsoccur ricevuto input come parametro, con l'ausilio di \lstinline{meetingNoteObjectProvider} (vedi sezione \ref{subsec:meeting-note-object-provider}) e di controllare l'autorizzazione della richiesta con \lstinline{authProvider}, prima di tutto ciò viene assicurata la presenza di connessione ad internet da \lstinline{networkAwareProvider} (RIMANDARE ALLA SEZIONE SPECIFICA). \\
Vengono gestiti inoltre i vari esiti della richiesta (i medesimi illustrati nella sezione \ref{subsec:model}, vedi paragrafo \emph{Status Response}) e solamente in caso di successo viene restituito il risultato e aggiornato lo stato di \lstinline{selectedMeetingNoteObject}. \\
Mentre il secondo mostra all'utente una finestra di dialogo attraverso la quale si chiede all'utente di confermare la creazione della \emph{Meeting Note} e in caso affermativo viene effettuata la richiesta al \gls{backend}\glsoccur attraverso \lstinline{meetingNoteProvider} (vedi sezione \ref{subsec:meeting-note-provider}), controllandone l'autorizzazione con \lstinline{authProvider} e gestendo i vari esiti della richiesta (i medesimi illustrati nella sezione \ref{subsec:model}, vedi paragrafo \emph{Status Response}).

\subsubsection*{TitleLabel}
\label{subsubsec:title-label}

Classe privata che estende \lstinline{StatelessWidget} e definisce solamente l'aspetto grafico del titolo per la label di ogni campo della \emph{Meeting Note}.

\subsection{Wizard Screen 1}
\label{subsec:wizard-screen-1}

% Classi
    % - WizardPage1 (ConsumerStatefulWidget)
        % - Parametri costruttore
            % - title (String)
            % - meetingNote (MeetingNote)
        % - Variabili di stato -> inizializzate via initState
            % - selectedCategory (ObjectCategory)
            % - selectedCategoryIndex (int)
            % - searchController (TextEditingController)
            % - selectedMeetingNoteObject (MeetingNoteObject)
            % - pagingController (PagingController)
            % - searchTerm (String)
            % - isButtonEnabled (bool)
        % - Provider
            % - meetingNoteObjectProvider
            % - authProvider
            % - networkAwareProvider    
        % - Struttura
            % - WizardScreen
            % - WizardHeader
            % - CustomObjectPicker
            % - CustomSearchBar
            % - MeetingNoteObjectList
            % - ForwardButton
        % - Metodi
            % - initState
            % - dispose
            % - fetchPage
            % - isEditingMode
            % - showWizardPage2
    % - MeetingNoteObjectList (StatefulWidget)
        % - Parametri costruttore
            % - onObjectSelected (Function)
            % - pagingController (PagingController)
            % - isEditingMode (bool)
        % - Variabili di stato -> initState
            % - selectedIndex (int)
        % - Struttura
            % - PagedListView
            % - MeetingNoteObjectItem
        % - Metodi
            % - initState
    % - MeetingNoteObjectItem (StatefulWidget)
        % - Parametri costruttore
            % - isContractor (bool)
            % - vatNumber (String)
            % - name (String)
            % - index (int)
            % - selectedIndex (int)
            % - onTap (Function)
        % - Struttura
            % - ListTile
            % - title (Text) -> identificativo cliente
            % - subtitle (Text) -> se isContractor == true


Schermata inerente al primo \emph{step} per la creazione/modifica di una \emph{Meeting Note}, data la sua complessità, essa è stata suddivisa in classi, di cui verranno descritte nel dettaglio. 

\subsubsection*{WizardPage1}
\label{subsubsec:wizard-page-1}

Per comprendere in quale modalità viene richiamato il \gls{wizard}\glsoccur è presente un metodo privato \lstinline{isEditingMode} che restituisce \lstinline{true} se l'utente sta modificando una \emph{Meeting Note} e \lstinline{false} se sta creando una nuova \emph{Meeting Note}. \\
Questo avviene controllando l'oggetto \lstinline{MeetingNote} passato in input alla classe \lstinline{WizardPage1} se è vuoto, in tal caso l'utente sta creando una nuova \emph{Meeting Note}, altrimenti sta modificando una \emph{Meeting Note} esistente. \\
Prima della renderizzazione dell'aspetto grafico viene effettuata l'inizializzazione delle variabili di stato, in particolare:
\begin{itemize}
    \item \lstinline{selectedCategory}, \lstinline{ObjectCategory} che assume il valore della categoria del cliente della \emph{Meeting Note} passata in input se l'utente sta modificando, altrimenti viene inizializzato di \emph{deafult} con \lstinline{ObjectCategory.broker};
    \item \lstinline{selectedCategoryIndex}, \lstinline{int} che assume il valore dell'indice della categoria del cliente della \emph{Meeting Note} passata in input se l'utente sta modificando, altrimenti viene inizializzato di \emph{deafult} con \lstinline{0};
    \item \lstinline{searchController}, \lstinline{TextEditingController} che assume il valore dell'identificativo del cliente della \emph{Meeting Note} passata in input se l'utente sta modificando, altrimenti viene inizializzato di \emph{deafult} con una stringa vuota;
    \item \lstinline{selectedMeetingNoteObject}, \lstinline{MeetingNoteObject} che assume il valore del cliente della \emph{Meeting Note} passata in input se l'utente sta modificando, altrimenti viene inizializzato di \emph{deafult} con \lstinline{null};
    \item \lstinline{pagingController}, \lstinline{PagingController} che si occupa di gestire la paginazione dei dati;
    \item \lstinline{searchTerm}, \lstinline{String} che assume il valore dell'identificativo del cliente della \emph{Meeting Note} passata in input se l'utente sta modificando con il conseguente aggiornamento di \lstinline{pagingController}, altrimenti viene inizializzato di \emph{deafult} con una stringa vuota;
    \item \lstinline{isButtonEnabled}, \lstinline{bool} che abilita il pulsante di creazione se tutti i campi sono correttamente compilati e viene inizializzato a \lstinline{true} se l'utente sta modificando.
\end{itemize}
Per la definizione dell'aspetto grafico questa schermata implementa \lstinline{WizardScreen} (vedi sezione \ref{subsec:screens-template}) e il suo contenuto è strutturato come segue:
\begin{itemize}
    \item \lstinline{WizardHeader}, imposta il titolo della schermata (vedi sezione \ref{subsec:screens-template});
    \item \lstinline{CustomObjectPicker}, componente che si occupa di mostrare la lista delle categorie dei \glspl{cliente}\glsoccur e di permettere all'utente di effettuare una selezione, aggiornando di conseguenza \lstinline{pagingController} e filtrando la lista dei clienti con la categoria selezionata, il suo funzionamento è supportato dalle variabili di stato \lstinline{selectedCategory} e \lstinline{selectedCategoryIndex} (vedi sezione \ref{subsec:object-picker});
    \item \lstinline{CustomSearchBar}, componente che visualizza l'identificativo del cliente e permette all'utente di effettuare una ricerca, aggiornando eventualmente \lstinline{pagingController} e filtrando opportunamente la lista dei clienti, il suo funzionamento è supportato dalle variabili di stato \lstinline{searchController} e \lstinline{searchTerm} (vedi sezione \ref{subsec:search-bar});
    \item \lstinline{MeetingNoteObjectList}, componente che si occupa di mostrare la lista dei clienti opportunamente filtrata e paginata, il suo funzionamento è supportato dalle variabili di stato \lstinline{selectedMeetingNoteObject}, il cui scopo è di evidenziare quale cliente è stato selezionato e \lstinline{pagingController} (vedi sezione \ref{subsec:meeting-note-object-list});
    \item \lstinline{ElevatedButton}\cite{site:elevated-button}, componente che si occupa di confermare la selezione del cliente e di mostrare la schermata successiva del \gls{wizard}\glsoccur.
\end{itemize}
Nel dettaglio, prima della visualizzazione della lista paginata, si effettua un controllo dello stato della connessione ad internet attraverso il \lstinline{NetworkAwareProvider} (RIMANDARE ALLA SEZIONE SPECIFICA), in caso di assenza di connessione viene mostrato un \lstinline{WarningAlert} (vedi sezione \ref{subsec:warning-alert}) con il relativo messaggio di errore. \\
Sono presenti dei metodi privati a supporto del funzionamento del \gls{wizard}\glsoccur, che sono:
\begin{itemize}
    \item \lstinline{fetchPage}, si occupa di effettuare la richiesta al \gls{backend}\glsoccur attraverso il \lstinline{MeetingNoteObjectProvider} (vedi sezione \ref{subsec:meeting-note-object-provider}), controllando se l'utente è autorizzato ad effettuare la richiesta attraverso il metodo \lstinline{isUnauthorized} fornito da \lstinline{authProvider} (vedi sezione \ref{subsec:authentication-provider}), per poi restituire la lista dei clienti e aggiornare lo stato del \lstinline{PagingController};
    \item \lstinline{isEditingMode}, discusso all'inizio di questa sezione;
    \item \lstinline{showWizardPage2}, si occupa di mostrare la schermata successiva del \gls{wizard}\glsoccur, passando in input l'oggetto \lstinline{MeetingNote}, solamente in caso di modifica, e l'oggetto \lstinline{MeetingNoteObject} selezionato dall'utente.
\end{itemize}

\subsubsection*{MeetingNoteObjectList}
\label{subsubsec:meeting-note-object-list}

\indent Classe che si occupa di mostrare la lista dei clienti opportunamente filtrata e paginata. \\
Attraverso il costruttore riceve in input il \lstinline{PagingController}, la funzione \lstinline{onObjectSelected} che si occupa di aggiornare lo stato di \lstinline{selectedMeetingNoteObject} e la variabile \lstinline{isEditingMode} per evidenziare il cliente selezionato dall'utente. \\

\indent Prima della renderizzazione dell'aspetto grafico viene effettuata l'inizializzazione della variabile di stato \lstinline{selectedIndex} che assume il valore \lstinline{0} (si precisa che il conteggio degli elementi della lista parte da \lstinline{0} e non da \lstinline{1}) se l'utente sta modificando, in quanto la lista in questa modalità mostra solamente un elemento, altrimenti viene inizializzato a \lstinline{-1} per indicare che nessun elemento è stato ancora selezionato. \\
Per la definizione dell'aspetto grafico questa schermata implementa \lstinline{StatefulWidget} e il suo contenuto è strutturato sostanzialmente da \lstinline{PagedListView}\cite{site:infinite-scroll-pagination}, componente che si occupa di mostrare la lista paginata dei clienti, in particolare si avvale del \lstinline{PagingController} per gestire la paginazione dei dati, inoltre per ogni elemento della lista viene mostrato un \lstinline{MeetingNoteObjectItem} (vedi sezione \ref{subsubsec:meeting-note-object-item}) che rappresenta il cliente e permette all'utente di selezionarlo.

\subsubsection*{MeetingNoteObjectItem}
\label{subsubsec:meeting-note-object-item}

Classe che si occupa di mostrare il cliente nella lista e permette all'utente di selezionarlo. \\
Attraverso il costruttore riceve in input i seguenti parametri:
\begin{itemize}
    \item \lstinline{isContractor}, \lstinline{bool} che indica se il cliente è un \gls{cliente}\glsoccur appartiene alla categoria dei \emph{contraenti};
    \item \lstinline{vatNumber}, \lstinline{String} che rappresenta la partita iva del contraente;
    \item \lstinline{name}, \lstinline{String} che rappresenta l'identificativo del cliente;
    \item \lstinline{index}, \lstinline{int} che rappresenta l'indice del cliente nella lista;
    \item \lstinline{selectedIndex}, \lstinline{int} che rappresenta l'indice del cliente selezionato dall'utente;
    \item \lstinline{onTap}, \lstinline{Function} che si occupa di impostare \lstinline{selectedIndex} con l'indice del cliente selezionato dall'utente.
\end{itemize}
Per la definizione dell'aspetto grafico questa schermata implementa \lstinline{StatefulWidget} e il suo contenuto è strutturato sostanzialmente da \lstinline{ListTile}\cite{site:list-tile}, \emph{widget} che si occupa di mostrare il cliente nella lista, in particolare si avvale dei parametri \lstinline{title} e \lstinline{subtitle} per mostrare rispettivamente l'identificativo del cliente e la sua partita iva, inoltre evidenzia l'\emph{item} del cliente selezionato variandone il colore di sfondo e del testo.\\
Per capire quale cliente è stato selezionato dall'utente viene effettuato una comparazione del valore di \lstinline{selectedIndex} e \lstinline{index}.

\subsection{Wizard Screen 2}
\label{subsec:wizard-screen-2}

% Classi 
    % - WizardPage2 (StatefulWidget)
        % - Parametri costruttore
            % - title (String)
            % - meetingNoteObject (MeetingNoteObject)
            % - meetingNote (MeetingNote)
        % - Variabili di stato -> inizializzate via initState
            % - selectedDate (DateTime)
        % - Provider -> NESSUNO
        % - Struttura
            % - WizardScreen
            % - WizardHeader
            % - CustomScrollDatePicker
            % - ForwardButton
            % - BackButton
        % - Metodi 
            % - initState
            % - showWizardPage3

Schermata inerente al secondo \emph{step} per la creazione/modifica di una \emph{Meeting Note}. \\
Prima della renderizzazione dell'aspetto grafico viene effettuata l'inizializzazione della variabile di stato \lstinline{selectedDate} che assume il valore della data della \emph{Meeting Note} passata in input se l'utente sta modificando, altrimenti viene inizializzato di \emph{deafult} con la data corrente. \\
Per la definizione dell'aspetto grafico questa schermata implementa \lstinline{WizardScreen} (vedi sezione \ref{subsec:screens-template}) e il suo contenuto è strutturato come segue:
\begin{itemize}
    \item \lstinline{WizardHeader}, imposta il titolo della schermata (vedi sezione \ref{subsec:screens-template});
    \item \lstinline{CustomScrollDatePicker}, componente che si occupa di mostrare la data e permette all'utente di selezionarla, il suo funzionamento è supportato dalla variabile di stato \lstinline{selectedDate} (vedi sezione \ref{subsec:scroll-date-picker});
    \item \lstinline{ElevatedButton}\cite{site:elevated-button}, pulsante che si occupa di confermare la selezione della data e di renderizzare alla schermata successiva del \gls{wizard}\glsoccur invocando il metodo \lstinline{showWizardPage3};
    \item \lstinline{OutlinedButton}\cite{site:outline-button}, pulsante che si occupa di renderizzare alla schermata precedente del \gls{wizard}\glsoccur.
\end{itemize}
In questa schermata non è presente alcun \lstinline{Provider} in quanto non è necessario effettuare alcuna richiesta al \gls{backend}\glsoccur, poichè si tratta solamente di effettuare una selezione della data.


\subsection{Wizard Screen 3}
\label{subsec:wizard-screen-3}

% Classi
    % - WizardPage3 (ConsumerStatefulWidget)
        % - Parametri costruttore
            % - title (String)
            % - meetingNoteObject (MeetingNoteObject)
            % - meetingNoteDate (DateTime)
            % - meetingNote (MeetingNote)
        % - Variabili di stato -> inizializzate via initState
            % - textController (TextEditingController)
            % - isButtonEnabled (bool)
            % - speechToText (SpeechToText)
            % - listenedWords (String)
        % - Provider
            % - meetingNoteProvider
            % - networkAwareProvider
            % - authProvider
        % - Struttura
            % - WizardScreen
            % - WizardHeader
            % - SummaryCard
            % - CustomTextBox
            % - RecordingButton
            % - ConfirmButton
        % - Metodi
            % - initState
            % - dispose
            % - onConfirmation
    % - SummaryCard (privato) (StatelessWidget)     
        % - Parametri costruttore
            % - meetingNoteObject (MeetingNoteObject)
            % - meetingNoteDate (DateTime)
        % - Struttura
            % - Card
            % - Text -> identificativo cliente
            % - Text -> data  

Schermata inerente al terzo \emph{step} per la creazione/modifica di una \emph{Meeting Note}. \\
\lstinline{WizardPage3} è la classe principale che si occupa di ricevere in input il contenuto testuale dall'utente e di permettergli di confermare la creazione della \emph{Meeting Note}. \\
Prima della renderizzazione dell'aspetto grafico viene effettuata l'inizializzazione delle variabili di stato, in particolare:
\begin{itemize}
    \item \lstinline{textController}, \lstinline{TextEditingController} che si occupa di gestire il contenuto testuale inserito dall'utente, se l'utente sta modificando la \emph{Meeting Note} viene inizializzato con il contenuto testuale della \emph{Meeting Note} passata in input, altrimenti viene inizializzato con una stringa vuota;
    \item \lstinline{isButtonEnabled}, \lstinline{bool} che indica se l'utente ha inserito del testo e abilita di conseguenza il pulsante per confermare la creazione della \emph{Meeting Note};
    \item \lstinline{speechToText}, \lstinline{SpeechToText} che si occupa di gestire la dettatura vocale;
    \item \lstinline{listenedWords}, \lstinline{String} che memorizza quanto detto dall'utente.
\end{itemize}
Per la definizione dell'aspetto grafico questa schermata implementa \lstinline{WizardScreen} (vedi sezione \ref{subsec:screens-template}) e il suo contenuto è strutturato come segue:
\begin{itemize}
    \item \lstinline{WizardHeader}, imposta il titolo della schermata (vedi sezione \ref{subsec:screens-template});
    \item \lstinline{SummaryCard}, classe che si occupa di mostrare un riepilogo dei dati inseriti dall'utente, il suo funzionamento è supportato dalle variabili di stato \lstinline{meetingNoteObject} e \lstinline{meetingNoteDate}, verrà discussa la sua struttura nel dettaglio in seguito;
    \item \lstinline{CustomTextBox}, componente che si occupa di ricevere in input il contenuto testuale inserito dall'utente, che può avvenire in due modalità differenti come specificato nei requisiti \hyperref[RFN-72]{RFN-72} e \hyperref[RFN-73]{RFN-73} (vedi sezione \ref{subsec:text-box});
    \item \lstinline{RecordingButton}, componente che si occupa attivare la dettatura vocale (vedi sezione \ref{subsec:recording-button});
    \item \lstinline{OutlinedButton}\cite{site:outline-button}, pulsante che si occupa di renderizzare alla schermata precedente del \gls{wizard}\glsoccur;
    \item \lstinline{ElevatedButton}\cite{site:elevated-button}, pulsante che si occupa di confermare la creazione/modifica della \emph{Meeting Note} richiamando il metodo \lstinline{onConfirmation}.
\end{itemize}
Il metodo privato \lstinline{onConfirmation} si occupa di mostrare all'utente una finestra di dialogo attraverso la quale si chiede all'utente di confermare la creazione della \emph{Meeting Note} e in caso affermativo viene effettuata la richiesta al \gls{backend}\glsoccur attraverso \lstinline{meetingNoteProvider} (vedi sezione \ref{subsec:meeting-note-provider}), controllandone l'autorizzazione con \lstinline{authProvider} e gestendo i vari esiti della richiesta (i medesimi illustrati nella sezione \ref{subsec:model}, vedi paragrafo \emph{Status Response}). \\
Per differenziare il caso di creazione a quello di modifica si utilizza lo stesso approccio descritto nella sezione \ref{subsubsec:wizard-page-1}, ovvero si verifica se l'oggetto \lstinline{MeetingNote} passato in input è vuoto, in tal caso l'utente sta creando una nuova \emph{Meeting Note}, altrimenti sta modificando una \emph{Meeting Note} esistente, di conseguenza viene invocato la chiamata opportuna. \\
Prima della gestione degli esiti possibili della richiesta viene effettuato un controllo sullo stato della connessione ad internet attraverso il \lstinline{NetworkAwareProvider} (RIMANDARE ALLA SEZIONE SPECIFICA), in caso di assenza di connessione viene mostrato un \lstinline{WarningAlert} (vedi sezione \ref{subsec:warning-alert}) con il relativo messaggio di errore. \\
Infine durante l'elaborazione della richiesta viene mostrato una finestra di dialogo mostrando all'utente che l'operazione è in corso.

\section{Styles}
\label{sec:styles}

% SPIEGARE COME FUNZIONANO I TEMI, OVVERO CHE UNA VOLTA DEFINITO ADEGUATAMENTE E RICHIAMATO NEL MAIN.DART, AI WIDGET IMPLEMENTATI VIENE, generalmente, APPLICATO in maniera AUTOMATICA, altre volte, se si vuole una ulteriore personalizzazione, si usa Theme.of(context).<nome_elemento>.<colore>\\

\emph{Flutter}\cite{site:flutter} per applicare uno stile grafico all'interfaccia si serve della classe \lstinline{Theme}\cite{site:theme-class} in cui viene richiamata nel \lstinline{main.dart} (vedi sezione \ref{sec:main}), la conseguenza è che i \emph{widget} implementati erediteranno lo stile grafico definito. \\
Senza una ridefinizione del tema, verrà applicato quello di \emph{default}, per questo progetto invece, seguendo il \gls{mockup}\glsoccur è stato personalizzato. \\
In particolare è sufficiente ridefinire opportunamente la classe \lstinline{ThemeData}\cite{site:theme-data-class}, che contiene tutta la configurazione grafica dell'applicazione. \\
Per la ridefinizione del tema si sono create delle classi, raccolte in opportuni file, che si occupano di definire i colori, dimensioni e stili grafici utilizzati nell'applicazione. \\
Si specifica che le variabili utilizzate per la configurazione grafica sono definite come \lstinline{static const}, in quanto devono essere accessibili senza la necessità di istanziare la classe. \\
Di seguito verrà illustrato nel dettaglio il contenuto di ogni file.

\subsection{AppColors}
\label{subsec:app-colors}

\lstinline{AppColors} è una classe in cui viene ridefinita la \emph{palette} colori utilizzata nell'applicazione, in particolare:
\begin{itemize}
    \item \lstinline{primary}, rappresenta il colore primario, applicato ai compomenti principali dell'applicazione (es. pulsanti, icone, ecc.);
    \item \lstinline{secondary}, rappresenta il colore secondario dell'applicazione, applicato ai componenti secondari (es. \emph{filter chips}\cite{site:chips});
    \item \lstinline{error}, rappresenta il colore da applicare nelle eccezioni (es. validazione dei campi);
    \item \lstinline{surface}, rappresenta il colore di sfondo per una categoria di componenti (es. \emph{card}\cite{site:card});
    \item \lstinline{background}, rappresenta il colore di sfondo per l'intera applicazione.
\end{itemize}

\subsection{ButtonStyles}
\label{subsec:button-styles}

In questo file sono presenti differenti classi il cui scopo è di ridefinire lo stile grafico dei pulsanti utilizzati nell'applicazione. \\
Successivamente verrà discusso in dettaglio la configurazione dello stile grafico dei pulsanti di tipo \lstinline{ElevatedButton}\cite{site:elevated-button}, si precisa dunque che per le altre due tipologie di pulsanti, \lstinline{OutlinedButton}\cite{site:outline-button} e \lstinline{TextButton}\cite{site:text-button}, è stato seguito lo stesso approccio.

\subsubsection*{ButtonsColor}
\label{subsubsec:button-color}

\lstinline{ButtonsColor} è una classe in cui viene ridefinito il colore dei pulsanti, in particolare:
\begin{itemize}
    \item \lstinline{disabledButton}, colore applicato ai pulsanti disabilitati;
    \item \lstinline{backgroundButton}, colore di sfondo dei pulsanti abilitati;
    \item \lstinline{overlayButton}, colore di sfondo dei pulsanti abilitati quando vengono premuti;
    \item \lstinline{confirmButton}, colore di sfondo dei pulsanti di conferma;
    \item \lstinline{deleteButton}, colore di sfondo dei pulsanti di annullamento o eliminazione.
\end{itemize}

\subsubsection*{ButtonsPadding}
\label{subsubsec:button-padding}

Classe privata, infatti verrà utilizzata solamente all'interno del file per la configurazione dello stile grafico dei pulsanti, in cui ne viene ridefinito il \emph{padding}\cite{site:padding}. \\
Ne sono stati implementati due: \lstinline{normalButtonPadding} e \lstinline{smallButtonPadding}, il primo è stato applicato alla maggior parte dei pulsanti presenti nell'applicazione, mentre il secondo ai pulsanti contenuti in componenti di dimensioni ridotte (es. \lstinline{AlertDialog}).

\subsubsection*{ButtonsBorder}
\label{subsubsec:button-border}

Classe privata implementata per definire la proprietà \lstinline{border}\cite{site:border}, in particolare è stato semplicemente impostato il raggio degli angoli a \lstinline{20.0}.

\subsubsection*{ElevatedButton}
\label{subsubsec:elevated-button}

Classe pubblica che contiene tutte le impostazione grafiche per i pulsanti di tipo \lstinline{ElevatedButton}\cite{site:elevated-button}, in particolare i colori che deve assumere in base agli stati di attivazione e disattivazione, il \emph{padding}, precisamente \lstinline{normalButtonPadding}, e la \emph{border radius}\cite{site:border-radius}. \\
Questo viene fatto, oltre per lo stile del pulsante base, ma anche per queli di conferma e di annullamento/eliminazione.

\subsubsection*{AlertDialogButton}
\label{subsubsec:alert-dialog-button}

Classe che definisce lo stile grafico dei pulsanti impiegati nelle finestre di dialogo, in particolare ne vengono configurati solo due, uno per la conferma e uno per l'annullamento. \\
Entrambi condividono lo stesso \emph{padding}, precisamente \lstinline{smallButtonPadding}, e la \emph{border radius}\cite{site:border-radius}, mentre differiscono per il colore di sfondo, \lstinline{confirmButton} e \lstinline{deleteButton} rispettivamente e per la tipologia (FORSE, DA CONFERMARE).

\subsection{TextStyles}
\label{subsec:text-styles}

In questo file sono presenti differenti classi il cui scopo è di ridefinire lo stile grafico dei testi utilizzati nell'applicazione. \\
La classe principale \lstinline{TextStyles} contiene varie stili di testo, in particolare per ciascuno di essi è stato definito: \lstinline{fontSize}\cite{site:font-size} e \lstinline{fontWeight}\cite{site:font-weight}. \\
Gli stili di testo\cite{site:text-theme} che sono stati ridefiniti sono:
 \lstinline{displayLarge},
 \lstinline{displayMedium},
 \lstinline{displaySmall},
 \lstinline{titleLarge},
 \lstinline{titleMedium},
 \lstinline{titleSmall},
 \lstinline{bodyLarge},
 \lstinline{bodyMedium} e
 \lstinline{bodySmall}. \\
 È presente un ulteriore classe, \lstinline{AlertDialogTextStyles}, in cui viene ridefinito lo stile grafico dei testi utilizzati nelle finestre di dialogo, in particolare una configurazione per il titolo, \lstinline{alertDialogTitleStyle}, e una per il contenuto, \lstinline{alertDialogContentStyle}, richiamando per ciascuno, uno degli stili sopra elencati.

 \subsection{AppTheme}
\label{subsec:app-theme}

Classe che si occupa di ridefinire il tema dell'applicazione, richiamando per ciascuna delle proprietà di \lstinline{ThemeData}\cite{site:theme-data-class} le classi precedentemente descritte. \\
I colori precedentemente configurati sono stati richiamati nella proprietà \lstinline{colorScheme}, mentre lo stile dei testi in \lstinline{textTheme}. \\
Nelle proprietà \lstinline{elevatedButtonTheme} e \lstinline{outlinedButtonTheme} sono stati richiamati rispettivamente le configurazione grafiche illustrate in precedenza.\\
Infine è stato impostato \emph{ad hoc} lo stile grafico del \lstinline{floatingActionButton}\cite{site:fab}.

\section{Utils}
\label{sec:utils}

In questa cartella sono raccolti file in cui sono definite delle classi che espongono metodi ausiliari per la gestione di determinate funzionalità. \\
Di seguito, per ciascun file, verranno illustrate nel dettaglio le classi presenti.

\subsection{Biometric auth}
\label{subsec:biometric-auth}

È presente la classe \lstinline{BiometricAuth} che implementa la libreria \emph{local auth}\cite{site:local-auth} per la gestione dell'autenticazione attraverso il riconoscimento biometrico. \\
Sono presenti due metodi: \lstinline{isBiometricSupported} che ritorna \lstinline{true} se il dispositivo supporta il riconoscimento biometrico e \lstinline{authenticate} che si occupa di effettuare l'autenticazione attraverso il riconoscimento biometrico, se l'operazione va a buon fine viene restituito \lstinline{true}, altrimenti \lstinline{false} e in quest'ultimo caso è comunque possibile effettuare l'autenticazione attraverso l'inserimento delle credenziali manualmente.

\subsection{Debouncer}
\label{subsec:debouncer}

Classe che implementa il \gls{designpattern}\glsoccur \emph{debouncer}: tecnica per prevenire l'esecuzione di una funzione troppe volte. \\
In questo progetto è servito per la gestione della ricerca dell'identificativo dei clienti, evitando di effettuare una richiesta \gls{httpg}\glsoccur al \gls{backend}\glsoccur ad ogni carattere digitato dall'utente, inserendo dunque un intervallo di tempo per ritardarne l'esecuzione.

\subsection{Exception handler}
\label{subsec:exception-handler}

La classe \lstinline{UnauthorizedException} implementa l'interfaccia \lstinline{Exception}\cite{site:exception} per gestione delle eccezione che si verifica quando l'utente non è autorizzato ad effettuare una determinata richiesta al \gls{backend}\glsoccur. 

\subsection{Filtering}
\label{subsec:filtering}

La classe \lstinline{FilteringOptions} incapsula le informazioni necessarie per effettuare un filtraggio nella lista di \emph{Meeting Note}.\\
Le variabili di stato sono:
\begin{itemize}
    \item \lstinline{object}, \lstinline{MeetingNoteObject} che rappresenta il \gls{cliente}\glsoccur;
    \item \lstinline{dateRange}, \lstinline{DateTimeRange} che rappresenta l'intervallo di date;
    \item \lstinline{isChronologicalOrder}, \lstinline{bool} che indica se l'ordinamento deve avvenire in maniera cronologica o meno.
\end{itemize}
Inoltre sono presenti dei metodi per impostare il valore di ciasuna variabile di stato, inoltre è stato definito un metodo \lstinline{isNull} che restituisce \lstinline{false} se tutte le variabili di stato sono \lstinline{null}. \\
È stato creato poi un \emph{Provider} di tipo \lstinline{StateNotifierProvider}\cite{site:state-notifier-provider} per gestire lo stato di \lstinline{FilteringOptions} nella schermata principale dell'applicazione {vedi sezione \ref{subsec:home-screen}}.

\subsection{Network handler}
\label{subsec:network-handler}

In questo file è contenuto una enumerazione \lstinline{NetworkStatus} che rappresenta lo stato della connessione ad internet che può essere assunto dall'applicazione, in particolare: \lstinline{notDetermined}, \lstinline{connected} e \lstinline{disconnected}. \\
È presente la classe \lstinline{NetworkDetectorNotifier} in cui è stato implementato un costruttore che si occupa di inizializzare la variabile di stato \lstinline{networkStatus} con il valore \lstinline{notDetermined} e di aggiornarla quando avvengono dei cambiamenti. \\
È stato creato poi \lstinline{networkAwareProvider} un \emph{provider} di tipo \lstinline{StateNotifierProvider}\cite{site:state-notifier-provider} per leggere lo stato della connessione ad internet a livello globale.

\subsection{Shared preferences}
\label{subsec:shared-preferences}

In questo file vengono implemetate le librerie \lstinline{SharedPreferences}\cite{site:shared-preferences} e \lstinline{SecureStorage}\cite{site:flutter-secure-storage} per la memorizzazione di alcuni dati in locale, a supporto delle funzionalità di autenticazione e riconoscimento biometrico. \\
Si specifica che sono state definite due classi, per entrambe le due librerie, utilizzando il \gls{designpattern}\glsoccur \gls{singleton}\glsoccur, in quanto è necessario che vi sia una sola istanza per ciascuna classe. \\
La classe \lstinline{AuthPreferences} ha lo scopo di gestire il \emph{token} di autenticazione e lo stato di abilitazione del riconoscimento biometrico, esponendo i seguenti metodi:
\begin{itemize}
    \item \lstinline{saveToken}, salva il \emph{token} di autenticazione;
    \item \lstinline{getToken}, restituisce il \emph{token} di autenticazione;
    \item \lstinline{deleteToken}, elimina il \emph{token} di autenticazione.
    \item \lstinline{saveBiometricState}, salva una variabile \emph{booleana} che indica se l'utente ha abilitato il riconoscimento biometrico;
    \item \lstinline{getBiometricState}, restituisce il valore della variabile \emph{booleana} che indica se l'utente ha abilitato il riconoscimento biometrico.
\end{itemize}
Mentre la classe \lstinline{SecureStorage} ha lo scopo di gestire le credenziali di accesso, esponendo i seguenti metodi:
\begin{itemize}
    \item \lstinline{setUsername}, cifra e salva l'username;
    \item \lstinline{getUsername}, decifra e restituisce l'username;
    \item \lstinline{removeUsername}, elimina l'username;
    \item \lstinline{setPassword}, cifra e salva la password;
    \item \lstinline{getPassword}, decifra e restituisce la password;
    \item \lstinline{removePassword}, elimina la password;
    \item \lstinline{hasCredentials}, restituisce \lstinline{true} se sono salvate e cifrate le credenziali di accesso.
    \item \lstinline{removeAll}, elimina tutte le credenziali di accesso, invocando i metodi \lstinline{removeUsername} e \lstinline{removePassword};
    \item \lstinline{saveAll}, salva e cifra le credenziali di accesso, invocando i metodi \lstinline{setUsername} e \lstinline{setPassword}.
\end{itemize}

\subsection{Speech to text}
\label{subsec:speech-to-text}

In questo file è presente la classe \lstinline{AudioRecognizer} che implementa la libreria \lstinline{SpeechToText}\cite{site:speech-to-text} per la gestione della dettatura vocale. \\
In essa sono stati definiti i seguenti metodi:
\begin{itemize}
    \item \lstinline{initSpeech}, determina se il dispositivo supporta la dettatura vocale;
    \item \lstinline{startListening}, avvia la dettatura vocale, inoltre vengono impostati alcuni parametri, come la lingua e la durata massima della registrazione;
    \item \lstinline{stopListening}, interrompe la dettatura vocale;
    \item \lstinline{isListening}, restituisce \lstinline{true} se la dettatura vocale è in corso;
\end{itemize}

\section{main.dart}
\label{sec:main}

\section{Responsività}
\label{sec:responsivity}

\section{Usabilità}
\label{sec:usability}

% Listato delle varie classi principali implementate
% Fai riferimento allo storico dei commits su GitHub

% \section{Design Pattern utilizzati}
