\chapter{Implementazione}
\label{cap:implementazione}

\intro{In questo capitolo si discuterà dell'implementazione (o codifica) dell'applicazione in conseguenza alle scelte progettuali descritte nel capitolo precedente. Inoltre verranno descritte le librerie di terze parti utilizzate, motivandone la scelta.}\\

Seguendo quanto descritto nel capitolo \ref{cap:progettazione}, si è proceduto con l'implementazione del prodotto software.\\
Di seguito verrà illustratta l'effettiva struttura del progetto basata sull'approccio \emph{layer first} (vedi sezione \ref{sec:struttura-progetto}) implementata, descrivendone poi la varie classi contenute in ciascuna cartella. \\

\begin{verbatim}
    assets/
    lib/
        components/
        constants/
        data/
            model/
            service/
        provider/
        screens
        styles/
        utils/
        main.dart
\end{verbatim}

L'architettura menzionata nella sezione \ref{subsec:architettura-app} è stata implementata nel seguente modo: il \emph{data layer} e \emph{domain layer} sono stati implementati rispettivamente in \lstinline{service} e \lstinline{model}, contenute all'interno della cartella \lstinline{data}, mentre l'\emph{application layer} è stato implementato nella cartella \lstinline{provider} e il \emph{presentation layer} nella cartella \lstinline{screens}.\\
Le restanti cartelle sono servite come ausilio per l'implementazione delle funzionalità dei \emph{layer} sopraccitati, come ad esempio \lstinline{components} per la creazione di \emph{widget} personalizzati, \lstinline{constants} per la definizione di costanti, \lstinline{styles} per la definizione di stili e del tema dell'applicazione e \lstinline{utils} per la definizione di alcune funzioni di utilità.\\
\section{Components}
\label{sec:components}

INSERIRE LE FIGURE???\\
SPIEGARE O RIMANDARE ALLA DOCUMENTAZIONE DI COME FUNZIONANO I WIDGET\\

In questa cartella sono raccolti tutti i \emph{widget} personalizzati, utilizzati dalle schermate presenti in \emph{screens} (vedi sezione \ref{sec:screens}).\\

\subsection{Alert Dialogs}
\label{subsec:alert-dialogs}

Nel file denominato \lstinline{alert_dialogs.dart} sono stati implementati i \emph{widget} che consentono di personalizzare un \emph{alert dialog}.\\

\subsubsection*{CustomBaseAlertDialog}
\label{subsubsec:custom-base-alert-dialog}

È una classe che implementa un \lstinline{AlertDialog}\cite{site:alert-dialog} e ne definisce l'aspetto base, inserendo alcuni elementi fissi, come il titolo, un'icona posta sotto il titolo, il suo colore e \gls{padding}\glsoccur.\\
Mentre è possibile scegliere se aggiungere o meno un testo descrittivo e dei bottoni di conferma e/o annulla, in base all'operazione del contesto.\\

\subsubsection*{IconAlertDialog}
\label{subsubsec:icon-alert-dialog}

Classe che implementa \lstinline{CustomBaseAlertDialog} personalizzandolo ulteriormente impostando con dei valori fissi sia il \gls{padding}\glsoccur che la dimensione dell'icona.\\
Attraverso il costruttore è obbligatorio passare il \emph{widget} di tipo \lstinline{IconData}\cite{site:icon-data} che rappresenta l'icona, il suo colore e un titolo, mentre è opzionale passare un testo descrittivo, e se aggiungere o meno dei bottoni di conferma e/o annulla.\\

\subsubsection*{LoadingAlertDialog}
\label{subsubsec:loading-alert-dialog}

Personalizzazione di \lstinline{IconAlertDialog} in cui viene impostata come icona un \lstinline{CircularProgressIndicator}\cite{site:circular-progress-indicator}, che rappresenta un indicatore di caricamento.\\
È possibile decidere, attraveso il costruttore, il colore dell'indicatore e il titolo da visualizzare.\\
Lo scopo di questa classe è quella di essere utilizzata per mostrare un indicatore di caricamento durante l'esecuzione di un'operazione asincrona.\\

\subsubsection*{WarningAlertDialog}
\label{subsubsec:warning-alert-dialog}

Personalizzazione di \lstinline{IconAlertDialog} dove si richiede, nel costruttore, di passare un'icona e il suo colore, un titolo, il contenuto del testo descrittivo, l'azione che deve compiere il bottone di conferma alla sua pressione e il suo stile grafico (vedi sezione \ref{sec:styles}), mentre il pulsante di annullamento è stato fissato.\\
Lo scopo di questa classe è quella di essere utilizzata per mostrare un messaggio di avvertimento all'utente riguardante una scelta e di scegliere se confermarla.\\
\subsubsection*{ResponseDialog}
\label{subsubsec:response-dialog}

Personalizzazione di \lstinline{IconAlertDialog} dove si richiede, nel costruttore, di passare un'icona, il suo colore e un titolo.\\
Lo scopo di questa classe è quella di essere utilizzata per mostrare un messaggio di risposta all'utente riguardante l'esito di un'operazione.\\

\subsubsection*{NetworkAlertDialog}
\label{subsubsec:network-alert-dialog}

La sua implementazione è simile a \lstinline{WarningAlertDialog}, con la differenza che non è presente il pulsante di annullamento, in quanto il suo scopo è quello di notificare l'utente la mancanza di connessione ad internet e di cliccare il pulsante di conferma per chiudere l'\emph{alert dialog}.\\

\subsection{App Bars}
\label{subsec:app-bars}

Nel file denominato \lstinline{app_bars.dart} sono stati implementati i \emph{widget} che consentono di personalizzare un \emph{app bar}.\\

\subsubsection*{CustomAppBar}
\label{subsubsec:custom-app-bar}

Classe che implementa \lstinline{AppBar}\cite{site:app-bar} definendone il colore di background, applicato dal tema dell'applicazione (vedi sezione \ref{sec:styles}), il titolo, in cui si tratta dell'applicazione di un'immagine vettoriale contenuta nella cartella \lstinline{assets} la quale rapresenta il logo dell'azienda.\\
Infine è possibile scegliere se aggiungere o meno un'icona, che rappresenta il pulsante di accesso alla schermata dell'account utente.\\

\subsubsection*{LoginAppBar}
\label{subsubsec:login-app-bar}


\section{Constants}
\label{sec:constants}

\section{Data}
\label{sec:data}

\section{Provider}
\label{sec:provider}

\section{Screens}
\label{sec:screens}

\section{Styles}
\label{sec:styles}

SPIEGARE COME FUNZIONANO I TEMI, OVVERO CHE UNA VOLTA DEFINITO ADEGUATAMENTE, AI WIDGET IMPLEMENTATI VIENE, generalmente, APPLICATO in maniera AUTOMATICA, altre volte, se si vuole una ulteriore personalizzazione, si usa Theme.of(context).<nome_elemento>.<colore>\\

\section{Utils}
\label{sec:utils}

\section{main.dart}
\label{sec:main}

% Listato delle varie classi principali implementate
% Fai riferimento allo storico dei commits su GitHub

% \section{Design Pattern utilizzati}

% \section{Codifica}
