\chapter{Il progetto di stage}
\label{cap:descrizione-stage}

\intro{In questo capitolo viene descritto nel dettaglio il progetto, includendo gli obiettivi richiesti e la pianificazione del lavoro per la sua realizzazione. Infine viene motivata la scelta di tale progetto.}

\section{Descrizione del progetto}
\label{sec:descrizione-progetto}

Lo scopo del progetto di stage è quello di sviluppare un'applicazione \gls{crossplatform}\glsoccur per la gestione di \gls{meetingnote}\glsoccur e relativa documentazione.
Nello specifico, l'applicazione deve fornire le seguenti funzionalità:
\begin{itemize}
    \item \textbf{creazione} di una \gls{meetingnote}\glsoccur, inserendo le informazioni richieste, essa può avvenire in due modi:
    \begin{itemize}
        \item inserimento \textbf{manuale} dei dati attraverso un \gls{wizard}\glsoccur di creazione;
        \item \textbf{automatizzazion} del processo di creazione.
    \end{itemize}
    \item possibilità di \textbf{modificare} una \gls{meetingnote}\glsoccur già creata, attraverso un \gls{wizard}\glsoccur di modifica, analogo a quello di creazione;
    \item possibilità di \textbf{eliminare} una \gls{meetingnote}\glsoccur dalla lista di quelle create;
    \item \textbf{visualizzazione} di una lista di \gls{meetingnote}\glsoccur, con la possibilità di filtrarla per \gls{cliente}\glsoccur, data e ordine di creazione (dalla più recente o meno recente), inoltre, per ciascuna, è possibile visualizzarne in dettaglio il contenuto;
    \item integrazione con il sistema di \textbf{autenticazione} della piattaforma \emph{RiskAPP}, essa può avvenire in due modalità differenti:
    \begin{itemize}
        \item inserimento \textbf{manuale} delle credenziali;
        \item tramite \textbf{\gls{riconoscimentobiometrico}}\glsoccur.
    \end{itemize}
    \item possibilità di visualizzare i \textbf{dati dell'utente}.
\end{itemize}

\section{Obiettivi}
\label{sec:obiettivi}

In questa sezione verranno illustrati gli obiettivi, definiti nel \emph{Piano di Lavoro}, che si prevede di raggiungere durante lo svolgimento del progetto di stage.

\subsection{Notazione}
\label{subsec:notazione}

Si farà riferimento alla seguente notazione per indicare il grado di importanza degli obiettivi fissati:
\begin{itemize}
    \item \textbf{O}, identifica i requisiti obbligatori, vincolanti in quanto obiettivo primario richiesto dal committente;
    \item \textbf{D}, identifica i requisiti desiderabili, non vincolanti o strettamente necessari, ma dal riconoscibile valore aggiunto;
    \item \textbf{F}, identifica i requisiti facoltativi, rappresentanti valore aggiunto non strettamente competitivo.
\end{itemize}
Le sigle indicate saranno seguite da una coppia sequenziale di numeri, che identificano univocamente ogni obiettivo.

\subsection{Obiettivi fissati}
\label{subsec:obiettivi-fissati}

Si prevede lo svolgimento dei seguenti obiettivi:
\begin{itemize}
    \item \textbf{Obbligatori}
    \begin{itemize}
        \item \textbf{O01}\label{O01}: implementare un meccanismo di autenticazione attraverso delle chiamate all'\gls{apig}\glsoccur della piattaforma \emph{RiskAPP};
        \item \textbf{O02}\label{O02}: richiamare, attraverso richieste \gls{httpg}\glsoccur, le \gls{apig}\glsoccur per ottenere i dati necessari da visualizzare;
        \item \textbf{O03}\label{O03}: creazione delle \gls{meetingnote}\glsoccur;
        \item \textbf{O04}\label{O04}: compilazione del contenuto delle \gls{meetingnote}\glsoccur in maniera testuale e vocale;
        \item \textbf{O05}\label{O05}: sviluppo della \gls{uig}\glsoccur grafica dell'applicazione;
    \end{itemize}
    \item \textbf{Desiderabili}
    \begin{itemize}
        \item \textbf{D01}\label{D01}: integrazione di \gls{openai}\glsoccur per la creazione automatica delle \gls{meetingnote}\glsoccur;
        \item \textbf{D02}\label{D02}: \gls{deploy}\glsoccur dell'applicazione sulle piattaforme \emph{Android} e \emph{iOS};
    \end{itemize}
    \item \textbf{Facoltativi}
    \begin{itemize}
        \item \textbf{F01}\label{F01}: implementazione dei test per la verifica e validazione dell'applicazione.
    \end{itemize}
\end{itemize}

\section{Pianificazione}
\label{sec:pianificazione}

In questa sezione viene riportata la ripartizione settimanale delle attività da svolgere durante il periodo di stage. 
Tale pianificazione iniziale è stata definita anteriormente al periodo di stage e documentata nel \emph{Piano di Lavoro}.\\
Nella Tabella \ref{tab:preventivo-ore}, è stato riportato il preventivo della durata delle attività, espresso in ore, che si prevede di impiegare per la realizzazione del progetto di stage. \\

\subsubsection{Prima settimana: 24/07/23 - 28/07/23 (40 ore)}
    \begin{itemize}
        \item Incontro con il tutor aziendale per la presentazione dettagliata del progetto e comprensione dei requisiti richiesti;
        \item Visione del \emph{Tool trattative} e delle \gls{apig}\glsoccur, per la comprensione del contesto applicativo;
        \item Formazione sulle tecnologie e strumenti da utilizzare: \emph{Dart} \cite{site:dart}.
    \end{itemize}
\subsubsection{Seconda settimana: 31/07/23 - 04/08/23 (40 ore)}
    \begin{itemize}
        \item Formazione sulle tecnologie e strumenti da utilizzare: \emph{Figma} \cite{site:figma} e \emph{Flutter} \cite{site:flutter}.
    \end{itemize}
\subsubsection{Terza settimana: 07/08/23 - 11/08/23 (40 ore)}
    \begin{itemize}
        \item Realizzazione del \gls{mockup} dell'applicazione;
        \item Analisi dei requisiti;
        \item Fase di progettazione dell'applicazione e realizzazione del diagramma UML delle classi;
        \item Stesura della documentazione relativa alle attività di analisi dei requisiti e progettazione.
    \end{itemize}
\subsubsection{Quarta settimana: 14/08/23 - 18/08/23 (40 ore)}
    \begin{itemize}
        \item Inizio sviluppo dell'interfaccia utente (\gls{uig}\glsoccur) dell'applicazione (\textbf{O05});
        \item Implementazione del meccanismo di autenticazione (\textbf{O01});
        \item Chiamate alle \gls{apig}\glsoccur per ottenere i dati necessari da visualizzare (\textbf{O02});
        \item Stesura dei test per le funzionalità implementate (\textbf{F01}).
    \end{itemize}
\subsubsection{Quinta settimana: 21/08/23 - 25/08/23 (40 ore)}
    \begin{itemize}
        \item Implementazione della funzionalità di creazione delle \gls{meetingnote}\glsoccur (\textbf{O03});
        \item Implementazione della funzionalità di compilazione delle \gls{meetingnote}\glsoccur (\textbf{O04});
        \item Stesura dei test per le funzionalità implementate (\textbf{F01}).
    \end{itemize}
\subsubsection{Sesta settimana: 28/08/23 - 01/09/23 (40 ore)}
    \begin{itemize}
        \item Continuazione sviluppo \gls{uig}\glsoccur dell'applicazione (\textbf{O05});
        \item Integrazione di \gls{openai}\glsoccur per la creazione automatica delle \gls{meetingnote}\glsoccur (\textbf{D01}).
    \end{itemize}
\subsubsection{Settima settimana: 04/09/23 - 08/09/23 (40 ore)}
    \begin{itemize}
        \item Completamento dello sviluppo \gls{uig}\glsoccur dell'applicazione (\textbf{O05});
        \item Stesura dei test per le funzionalità implementate (\textbf{F01}).
    \end{itemize} 
\subsubsection{Ottava settimana: 11/09/23 - 15/09/23 (40 ore)}
    \begin{itemize}
        \item Esecuzione dei test per la verifica e validazione dell'applicazione (\textbf{F01});
        \item Stesura della documentazione relativa alle attività di codifica;
        \item \Gls{deploy}\glsoccur dell'applicazione sulle piattaforme \emph{Android} e \emph{iOS} (\textbf{D02}).
    \end{itemize}

    \begin{table}
        \centering
        \begin{tabularx}{\textwidth}{|c|X|}
            \hline
            \textbf{Ore pianificate} & \textbf{Descrizione dell'attività} \\\hline
            
            \textbf{80} & \textbf{Formazione sulle tecnologie} \\	 
            \hline
            
            \textbf{40} & \textbf{Definizione architettura di riferimento e relativa documentazione} \\ \hdashline 
            \multirow{3}{0cm}\\ 
            \textit{12} & 
            \textit{Analisi del problema e del dominio applicativo} \\
            \textit{22} & 
            \textit{Progettazione della piattaforma e relativi test} \\
            \textit{6} & 
            \textit{Stesura documentazione relativa ad analisi e progettazione} \\
            \hline
            
    
            \textbf{160} & \textbf{Sviluppo}  \\ \hdashline 
            \multirow{5}{0cm}\\ 
            \textit{26} & 
            \textit{Implementazione del meccanismo di autenticazione e integrazione con il sistema preesistente per il prelevamento dei dati da visualizzare nell'applicazione} \\
            \textit{34} & 
            \textit{Implementazione delle funzionalità di creazione e compilazione (testuale e vocale) delle meeting-note} \\
            \textit{28} & 
            \textit{Integrazione del servizio OpenAI} \\
            \textit{40} & 
            \textit{Attività di testing} \\
            \textit{32} & 
            \textit{Sviluppo UI} \\
            \hline
            
            \textbf{40} & \textbf{Verifica e Validazione finale}  \\ \hdashline 
            \multirow{3}{0cm}\\ 
            \textit{24} & 
            \textit{Esecuzione dei test per la verifica e collaudo dell'applicazione} \\
            \textit{8} & 
            \textit{Stesura documentazione finale} \\
            \textit{8} & 
            \textit{Deploy dell'applicazione} \\
            \hline
            
            \textbf{Totale ore} & \multicolumn{1}{|c|}{\textbf{320}} \\ 
            \hline
            
        \end{tabularx}
        \caption{Preventivo della durata delle attività}
        \label{tab:preventivo-ore}
    \end{table}

\section{Motivazione della scelta}
\label{sec:motivazione-scelta}

La scelta di tale progetto è stata dettata da varie motivazioni, alcune indipendenti da esso, ovvero la possibilità di mettersi in gioco e cercare di applicare le conoscenze acquisite durante il percorso di studi, in un contesto più professionale e reale. \\
Inoltre, come menzionato in precedenza, la mia esperienza pregressa con le tecnologie impiegate, mi ha semplificato la decisione, in quanto avere avuto la possibilità di utilizzarle per sviluppare un progetto in un contesto lavorativo è stato stimolante e gratificante, nonostante mi abbia anche permesso di saltare parzialmente la fase formativa, è stata comunque necessaria per approfondire le conoscenze e acquisirne di nuove.
