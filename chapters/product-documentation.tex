\chapter{Documentazione}
\label{cap:documentazione}

% \intro{In questo capitolo viene descritto }\\

\section{README}
\label{sec:readme}

Insieme al codice sorgente dell'applicazione è stata consegnata anche una breve documentazione, scritta in lingua inglese, secondo la richiesta dell'azienda, utilizzando il linguaggio di markup \gls{markdown}\glsoccur.\\
È stato redatto il file \lstinline{README.md}, il cui contenuto è riportato di seguito:
\begin{itemize}
    \item \textbf{Abstract}: breve descrizione dell'applicazione e del contesto applicativo;
    \item \textbf{Getting Started}: istruzioni per poter compilare ed eseguire l'applicazione;
    \item \textbf{Libraries Used}: elenco delle librerie utilizzate, necessarie per la compilazione;
    \item \textbf{Project Structure}: descrizione della struttura e architettura del progetto;
    \item \textbf{Features}: elenco delle funzionalità implementate.
\end{itemize}

\lstset{
    language=Java,
    basicstyle=\ttfamily,
    keywordstyle=\color{blue}\ttfamily,
    stringstyle=\color{red}\ttfamily,
    commentstyle=\color{green!60!black}\ttfamily,
}

\section{Documentazione del codice}
\label{sec:documentazione-codice}

Per quanto riguarda il codice sorgente, è stato scritto un commento per ogni classe, metodo e variabile, in modo da rendere più chiara la lettura del codice.\\
Lo standard seguito è quello proposto nella documentazione ufficiale di \emph{Dart}\cite{site:comment}. \\

In particolare, per i commenti alle classi, è stato utilizzato il seguente formato, spiegando anche il significato e scopo delle variabili di stato:
\begin{lstlisting}[language=Java, caption={Commento classe}, captionpos=b]
/// Class encapsulating the authentication's params,
/// given in input to the authentication request.
///
/// [_username], the user's username
///
/// [_password], the user's password
class AuthArgs {
    final String _username;
    final String _password;
    // ...
}
\end{lstlisting}

Per i commenti ai metodi, è stato utilizzato il medesimo formato utilizzato per i commenti alle classi, spiegando lo scopo del metodo, i parametri in input e il valore di ritorno:
\begin{lstlisting}[language=Java, caption={Commento metodo}, captionpos=b]
/// Authenticate the user with the given authentication [args]
/// retrieving the token and saving it in the local storage if the authentication request is successful.
///
/// Return [AuthResponse] object with:
///
/// - [AuthValues] object, containing the token if the request is successful
///
/// - [StatusResponse] object, containing the response's status
Future<AuthResponse> authenticate({required AuthArgs args}) async {
    log("credentials: ${args.toString()}");

    // ...

    log("auth response: ${result.toString()}");

    // ...
}
\end{lstlisting}