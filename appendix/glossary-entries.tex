% Acronyms
\newacronym[description={\glslink{apig}{Application Program Interface}}]
    {api}{API}{Application Program Interface}

\newacronym[description={\glslink{umlg}{Unified Modeling Language}}]
    {uml}{UML}{Unified Modeling Language}

% Glossary entries
\newglossaryentry{apig} {
    name=\glslink{api}{API},
    text=Application Program Interface,
    sort=api,
    description={in informatica con il termine \emph{Application Programming Interface API} (ing. interfaccia di programmazione di un'applicazione) si indica ogni insieme di procedure disponibili al programmatore, di solito raggruppate a formare un set di strumenti specifici per l'espletamento di un determinato compito all'interno di un certo programma. La finalità è ottenere un'astrazione, di solito tra l'hardware e il programmatore o tra software a basso e quello ad alto livello semplificando così il lavoro di programmazione}
}

\newglossaryentry{cliente}{
    name=Cliente,
    description={i clienti del contesto applicativo possono essere di tre tipologie:
    \begin{itemize}
        \item \textbf{Broker}: è un intermediario tra il cliente finale (in questo caso sempre aziende) e l'assicurazione;
        \item \textbf{Contraenti}: sono le aziende che stipulano un contratto della polizza assicurativa;
        \item \textbf{Compagnie assicurative}
    \end{itemize}
    }
}

\newglossaryentry{meetingnote}{
    name=Meeting Note,
    description={nota compilabile dall'utente per monitorare gli incontri con i clienti. Le informazioni che contiene sono:
    \begin{itemize}
        \item \textbf{\gls{cliente}}: nome del cliente;
        \item \textbf{Data}: data dell'incontro;
        \item \textbf{Contenuto}: contenuto dell'incontro;
        \item \textbf{Autore}: nome dell'utente che ha creato la nota;
    \end{itemize}}
}

\newglossaryentry{openai}{
    name=OpenAI \glslink{api}{API},
    description={insieme di servizi e risorse messi a disposizione da \emph{OpenAI}, che permettono agli sviluppatori di integrare modelli di intelligenza artificiale nei loro prodotti software}
}

\newglossaryentry{umlg} {
    name=\glslink{uml}{UML},
    text=UML,
    sort=uml,
    description={in ingegneria del software \emph{UML, Unified Modeling Language} (ing. linguaggio di modellazione unificato) è un linguaggio di modellazione e specifica basato sul paradigma object-oriented. L'\emph{UML} svolge un'importantissima funzione di ``lingua franca'' nella comunità della progettazione e programmazione a oggetti. Gran parte della letteratura di settore usa tale linguaggio per descrivere soluzioni analitiche e progettuali in modo sintetico e comprensibile a un vasto pubblico}
}
