% Acronyms
\newacronym[description={\glslink{apig}{Application Program Interface\glsoccur}}]
    {api}{API}{Application Program Interface}

    \newacronym[description={\glslink{clig}{Command Line Interface\glsoccur}}]
    {cli}{CLI}{Command Line Interface}

    \newacronym[description={\glslink{guig}{Graphic User Interface\glsoccur}}]
    {gui}{GUI}{Graphic User Interface}

    \newacronym[description={\glslink{iag}{Intelligenza Artificiale\glsoccur}}]
        {ia}{IA}{Intelligenza Artificiale}

\newacronym[description={\glslink{httpg}{HyperText Transfer Protocol\glsoccur}}]
    {http}{HTTP}{HyperText Transfer Protocol}

\newacronym[description={\glslink{jsong}{JavaScript Object Notation\glsoccur}}]
{json}{JSON}{JavaScript Object Notation}

\newacronym[description={\glslink{mvcg}{Model View Controller\glsoccur}}]
    {mvc}{MVC}{Model View Controller}

\newacronym[description={\glslink{uig}{User Interface\glsoccur}}]
    {ui}{UI}{User Interface}

\newacronym[description={\glslink{umlg}{Unified Modeling Language\glsoccur}}]
    {uml}{UML}{Unified Modeling Language}

\newacronym[description={\glslink{urig}{Uniform Resource Identifier\glsoccur}}]
    {uri}{URI}{Uniform Resource Identifier}

% Glossary entries
\newglossaryentry{apig} {
    name=\glslink{api}{API},
    text=API,
    sort=api,
    description={in informatica, con il termine \emph{Application Programming Interface API} (ing. interfaccia di programmazione di un'applicazione) si indica ogni insieme di procedure disponibili al programmatore, di solito raggruppate a formare un set di strumenti specifici per l'espletamento di un determinato compito all'interno di un certo programma. La finalità è ottenere un'astrazione, di solito tra l'hardware e il programmatore o tra software a basso e quello ad alto livello semplificando così il lavoro di programmazione}
}

\newglossaryentry{iag}{
    name=\glslink{ia}{Intelligenza Artificiale},
    text=IA,
    sort=ia,
    description={insieme di teorie e tecniche che permettono a un sistema di presentare un comportamento intelligente, ovvero di emulare l'intelligenza umana}
}

\newglossaryentry{attore}{
    name=Attore,
    text=attore,
    plural=attori,
    description={in informatica, con il termine \emph{attore} si indica un ruolo che l'utente o un sistema esterno ha nell'interazione con il sistema in esame}
}

\newglossaryentry{backend}{
    name=Back-end,
    text=back-end,
    description={in informatica,  si indica la parte di un sistema software non accessibile direttamente dall'utente, tipicamente responsabile della gestione dei dati e della loro elaborazione}
}

\newglossaryentry{baseline}{
    name=Baseline,
    text=baseline,
    plural=baseline,
    description={termine che viene utilizzato nella disciplina di \emph{ingegneria del software} per indicare un punto di avanzamento consolidato da cui ripartire per proseguire lo sviluppo del progetto}
}

\newglossaryentry{usecase}{
    name=Caso d'uso,
    text=caso d'uso,
    plural=casi d'uso,
    description={insieme di scenari (sequenza di azioni) che hanno in comune un obiettivo finale per un \gls{attore}\glsoccur.\\
        Sono rappresentabili attraverso dei diagrammi di tipo \gls{uml}, dedicati alla descrizione delle funzioni o servizi offerti da un sistema così come sono percepiti e utilizzati dagli attori che interagiscono col sistema stesso}
}

\newglossaryentry{clig}{
    name=\glslink{cli}{CLI},
    text=CLI,
    description={in informatica, con il termine \emph{CLI, Command Line Interface} (ing. interfaccia a riga di comando) si indica la parte di un sistema software che permette l'interazione con l'utente attraverso l'uso esclusivo di comandi testuali}
}

\newglossaryentry{cliente}{
    name=Cliente,
    text=cliente,
    plural=clienti,
    description={(nessuna fonte, fa parte del dominio applicativo) nel contesto applicativo della tesi, costituisce l'oggetto di una \emph{Meeting Note} e può essere di tre tipologie: \textbf{broker}, \textbf{contraenti} o \textbf{compagnia assicurativa}.\\
    Il primo tra questi è un intermediario tra il cliente finale e la compagnia assicurativa, mentre i secondi sono aziende che sottoscrivono un contratto della polizza assicurativa}
}

\newglossaryentry{crossplatform}{
    name=Cross-platform,
    text=cross-platform,
    description={in informatica, con il termine \emph{cross-platform} (ing. multi-piattaforma) si indica un software che può essere eseguito su più di un sistema hardware o software. Il termine si riferisce in particolare a sistemi operativi, ma anche a software applicativi, linguaggi di programmazione e strumenti di sviluppo}
}

\newglossaryentry{deploy}{
    name=Deploy,
    text=deploy,
    description={termine che indica la fase di distribuzione di un software, ovvero l'insieme di tutte le attività necessarie per rendere disponibile il prodotto finito all'utilizzatore finale}
}

\newglossaryentry{designpattern}{
    name=Design pattern,
    text=design pattern,
    description={in informatica, con il termine \emph{design pattern} (ing. schema progettuale) si indica una soluzione che può essere applicata in più contesti, per risolvere un problema ricorrente}
}

\newglossaryentry{endpoint}{
    name=Endpoint,
    text=endpoint,
    description={in informatica, con il termine \emph{endpoint} si indica un'interfaccia esposta da un servizio web, attraverso la quale è possibile interagire con esso}
}

\newglossaryentry{framework}{
    name=Framework,
    text=framework,
    description={in informatica, con il termine \emph{framework} si indica un'architettura logica di supporto su cui un software può essere progettato e realizzato, spesso facilitandone lo sviluppo da parte del programmatore}
}

\newglossaryentry{funnel}{
    name=Funnel commerciale,
    text=funnel commerciale,
    description={modello di marketing che consiste in un processo suddiviso in più fasi, attraverso le quali si cerca di convertire un utente generico in un cliente effettivo}
}

\newglossaryentry{gesture}{
    name=Gesture,
    text=gesture,
    description={si indica un insieme di azioni che l'utente può compiere attraverso l'interazione con il dispositivo mobile, come ad esempio, toccare, trascinare, pizzicare, ecc.}
}

\newglossaryentry{git}{
    name=Git,
    description={sistema di controllo versione distribuito, utilizzato per il tracciamento delle modifiche al codice sorgente durante lo sviluppo del software}
}

\newglossaryentry{github}{
    name=GitHub,
    description={piattaforma, basata sul sistema di controllo versione \gls{git}\glsoccur, che permette di controllare e versionare il codice sorgente.\\
    Offre molte altre funzionalità, tra cui la condivisibilita del codice e segnalazione di problematiche}
}

\newglossaryentry{guig}{
    name=\glslink{gui}{GUI},
    text=GUI,
    description={in informatica, con il termine \emph{GUI, Graphic User Interface} (ing. interfaccia utente grafica) si indica la parte del software che permette l'interazione con l'utente.\\
    La GUI è composta da elementi grafici, come finestre, icone, menu, ecc. che permettono l'interazione con l'utente attraverso dispositivi di input, come mouse e tastiera}
}

\newglossaryentry{httpg}{
    name=\glslink{http}{HTTP},
    text=HTTP,
    sort=http,
    description={con il termine \emph{HTTP, HyperText Transfer Protocol} (ing. protocollo di trasferimento ipertesto) si indica un protocollo a livello applicativo usato come principale sistema per la trasmissione d'informazioni sul web, ovvero in un'architettura tipica client-server.\\
    Le richieste sono composte da un \glslink{uri}{URI}, versione del protocollo e il metodo, che indica l'azione da compiere. Le risposte contengono un codice di stato, informazioni sul server e sulla risorsa richiesta e il contenuto della risorsa stessa}
}

\newglossaryentry{ingsw}{
    name=Ingegneria del software,
    text=ingegneria del software,
    description={disciplina che si occupa di tutti gli aspetti della produzione del software, dalla fase di progettazione a quella di manutenzione, passando per la codifica, la verifica e la validazione, fornendo principi e metodologie da applicare con lo scopo di garantire efficacia ed efficienza nel processo di sviluppo}
}

\newglossaryentry{issuetracking}{
    name=Issue tracking,
    text=issue tracking,
    description={si intende un sistema che permette di tenere traccia delle problematiche, riscontri, idee o attività durante lo sviluppo del software, in modo da poterle gestire in maniera efficiente}
}

\newglossaryentry{jsong}{
    name=\glslink{json}{JSON},
    text=JSON,
    sort=json,
    description={con il termine \emph{JSON} si indica un formato adatto all'interscambio di dati fra applicazioni client-server}
}

\newglossaryentry{markdown}{
    name=Markdown,
    description={linguaggio utilizzato per la formattazione di testo semplice, in particolare per la scrittura di documentazione tecnica}
}

\newglossaryentry{meetingnote}{
    name=Meeting Note,
    text=Meeting Note,
    description={(nessuna fonte, fa parte del dominio applicativo) nel contesto applicativo della tesi, si tratta di una nota compilabile dall'utente per monitorare gli incontri con i clienti. Le informazioni che contiene sono: il nome identificativo del \textbf{\Gls{cliente}\glsoccur}, la \textbf{Data} dell'incontro, il \textbf{Contenuto} e l'\textbf{Autore}}
}

\newglossaryentry{milestone}{
    name=Milestone,
    text=milestone,
    plural=milestones,
    description={termine che viene utilizzato nella disciplina di \emph{ingegneria del software} per indicare un punto di riferimento nel tempo che indica un obbiettivo da raggiungere, ovvero un evento che rappresenta un traguardo parziale nello svolgimento del progetto, deve essere consolidato dal raggiungimento di una o più \gls{baseline}\glsoccur}
}

\newglossaryentry{mockup}{
    name=Mockup,
    text=mockup,
    description={indica una realizzazione a scopo illustrativo di un prodotto, con il fine di fornire in anteprima una rappresentazione del prodotto stesso, mancante dell'implementazione effettiva delle sue funzionalità, ma che ne mostri ugualmente le caratteristiche salienti}
}

\newglossaryentry{mvcg}{
    name=\glslink{mvc}{MVC},
    text=MVC,
    sort=mvc,
    description={indica il \glslink{patternarchitetturale}{pattern architetturale}\glsoccur Model-View-Controll, separa la logica di presentazione dei dati dalla logica di business.\\
    Il \emph{modello} rappresenta la struttura dei dati dell'applicazione e le regole che governano la loro manipolazione.\\
    La \emph{vista} è la rappresentazione visuale del modello.\\
    Il \emph{controllore} è il componente che gestisce gli eventi, che possono essere generati sia dalla vista che dal modello, e aggiorna il modello e la vista}
}

\newglossaryentry{openai}{
    name=OpenAI \glslink{api}{API},
    description={\cite{site:openai} insieme di servizi e risorse messi a disposizione da \emph{OpenAI}, che permettono agli sviluppatori di integrare modelli di intelligenza artificiale nei loro prodotti software}
}

\newglossaryentry{open-source}{
    name=Open source,
    text=open source,
    description={in informatica, con il termine \emph{open source} si indica un software di cui il codice sorgente è pubblico, in modo che esso possa essere studiato, modificato e utilizzato da chiunque}
}

\newglossaryentry{overriding}{
    name=Overriding,
    text=overriding,
    description={in informatica, con il termine \emph{overriding} si indica la possibilità di ridefinire un metodo ereditato da una classe base, in modo da adattarlo alle esigenze della classe derivata}
}

\newglossaryentry{padding}{
    name=Padding,
    text=padding,
    description={\cite{site:padding} in \emph{Flutter}, indica un margine interno che viene applicato ad un \emph{widget}, definisce la distanza tra il contenuto e il suo bordo}
}

\newglossaryentry{patternarchitetturale}{
    name=Pattern architetturale,
    text=pattern architetturale,
    description={in informatica, con il termine \emph{pattern architetturale} si intende una soluzione, riutilizzabile, a un problema di progettazione architetturale, in un contesto definito.\\
    Delinea una struttura di base per un sistema software, specificando le sue parti, le loro responsabilità e le relazioni tra di esse}
}

\newglossaryentry{responsivita}{
    name=Responsività,
    text=responsività,
    description={in informatica, con il termine \emph{responsività} si indica la capacità di un sistema di adattarsi a diverse situazioni, in particolare, nel contesto applicativo, si intende la capacità di un'applicazione di adattarsi a diverse dimensioni dello schermo dei dispositivi mobile}
}

\newglossaryentry{uig}{
    name=\glslink{ui}{UI},
    text=UI,
    sort=ui,
    description={in informatica, con il termine \emph{UI, User Interface} (ing. interfaccia utente) si indica la parte del software che permette l'interazione con l'utente, attraverso l'uso di dispositivi hardware e software.\\ 
    L'interfaccia utente può essere grafica (\gls{guig}\glsoccur) o testuale (\gls{clig}\glsoccur)}
}

\newglossaryentry{umlg} {
    name=\glslink{uml}{UML},
    text=UML,
    sort=uml,
    description={in ingegneria del software \emph{UML, Unified Modeling Language} (ing. linguaggio di modellazione unificato) è un linguaggio di modellazione e specifica basato sul paradigma object-oriented. L'\emph{UML} svolge un'importantissima funzione di ``lingua franca'' nella comunità della progettazione e programmazione a oggetti. Gran parte della letteratura di settore usa tale linguaggio per descrivere soluzioni analitiche e progettuali in modo sintetico e comprensibile a un vasto pubblico}
}

\newglossaryentry{urig}{
    name=\glslink{uri}{URI},
    text=URI,
    sort=uri,
    description={in informatica, con il termine \emph{URI, Uniform Resource Identifier} (ing. identificatore uniforme di risorsa) si indica una sequenza di caratteri che identifica univocamente una risorsa generica che può essere un indirizzo Web, un documento, un'immagine, un file, un servizio, ecc.}
}

\newglossaryentry{riconoscimentobiometrico}{
    name=Riconoscimento biometrico,
    text=riconoscimento biometrico,
    description={si intende una particolare forma di autenticazione attraverso caratteristiche fisiologiche e/o comportamentali dell'utente, come ad esempio, tra quelle fisiologiche più comuni, l'impronta digitale e il riconoscimento facciale}
}

\newglossaryentry{singleton}{
    name=Singleton,
    text=singleton,
    description={si indica un \glslink{designpattern}{design pattern}\glsoccur creazionale con lo scopo di garantire che di una determinata classe venga creata una e una sola istanza, e di fornire un punto di accesso globale a tale istanza}
}

\newglossaryentry{wizard}{
    name=Wizard,
    text=wizard,
    description={in informatica, con il termine \emph{wizard} si indica un programma, solitamente contenuto in un'applicazione più complessa, che permette all'utente, attraverso una procedura passo-passo, di essere guidato per l'esecuzione di un compito, con lo scopo di semplificare l'interazione con il software}
}